\message{ !name(tarea2.tex)}\documentclass[12pt,a4paper]{article}
\usepackage{ugmath}
\usepackage{float}
\usepackage{placeins}
\usepackage{array}
\usepackage[labelfont=bf,labelsep=space]{caption}
\newcommand{\paren}[1]{\left( #1 \right)}
\newcommand{\alumno}{Ricardo León Martínez}
\newcommand{\materia}{Elementos de Probabilidad y Estadistica}
\newcommand{\profesor}{Claudia Reynoso Alcántara}
\newcommand{\tarea}{Tarea 2}
\newcommand{\fecha}{17/2/2026}

\begin{document}

\message{ !name(tarea2.tex) !offset(-3) }


    \begin{center}
        {\large\textbf{UNIVERSIDAD DE GUANAJUATO}}\\[0.3cm]
        {\normalsize\textbf{DIVISIÓN DE CIENCIAS NATURALES Y EXACTAS}}\\
        {\normalsize\textbf{CAMPUS GUANAJUATO}}\\[1cm]

        {\Large\textbf{\tarea\ (\materia)}}\\[1cm]
    \end{center}

    \noindent
    \textbf{Nombre:} \alumno 
    \hfill 
    \textbf{Fecha:} \fecha 
    \hfill 
    \textbf{Calificación:} \rule{3cm}{0.4pt}

    \vspace{0.3cm}

    \begin{mdframed}[style=mdbluebox,frametitle={Ejercicio 1}]
        En una tómbola se colocan $n$ pelotas identificadas cada una por un número del $1$ al $n$. 
        En cada una de $k$ ocasiones consecutivas, hacemos girar la tómbola, tomamos una pelota al azar 
        anotamos el número de la pelota que tomamos y, finalmente, la devolvemos a la tómbola.
        \begin{enumerate}
            \item[(a)] ¿Cuál es la probabilidad de que la sucesión de números obtenida sea estrictamente creciente?
            \item[(b)] ¿Cuál es la probabilidad de que la sucesión de números obtenida sea no decreciente?
        \end{enumerate}
    \end{mdframed}

    \begin{solucion}
        Sea $\Omega=\{1,\dots,n\}^{k}$ el espcio muestra de todas las suceciones posibles
        $(x_{1},\dots,x_{k})$ obtenidas al realizar $k$ extracciones con reemplazo.
        Como en cada extraccion todas las pelotas son independientes
        \begin{align*}
            \abs{\Omega}=n^{k},\qquad \mathbb{P}(\{(x_{1},\dots,x_{k})\})=\frac{1}{n^{k}}
        \end{align*}
        Por tanto, para cualquier evento $A\subseteq\Omega$,
        \begin{equation*}
            \mathbb{P}(A)=\frac{\abs{A}}{n^{k}}.
        \end{equation*}
        \begin{enumerate}
            \item[(a)] ¿Cuál es la probabilidad de que la sucesión de números obtenida sea estrictamente creciente?\\
            Queremos encontrar las sucesiones $x_{1}\lt x_{2}\lt\cdots\lt x_{k}$, con $k\in\{1,\dots,n\}$.
            Elegir una suceción estrictamente creciente equivale a elegir un subconjunto de $k$ elementos
            distintos de $\{1,\dots,n\}$. Por lo tanto, el número de tales sucesiones es
            \begin{equation*}
                \binom{n}{k}.
            \end{equation*}
            Asi la probabilidad es
            \begin{equation*}
                \mathbb{P}(x_{1}\lt\cdots\lt x_{k})=\frac{\binom{n}{k}}{n^{k}} \qquad (k\leq n),
            \end{equation*}
            y es 0 si $k\gt n$.
            \item[(b)] ¿Cuál es la probabilidad de que la sucesión de números obtenida sea no decreciente?\\
            Ahora queremos contar las sucesiones $x_{1}\leq x_{2}\leq\cdots\leq x_{k}$, con $i\in\{1,\dots,n\}$.
            Aqui se permiten repeticiones.
            Sea $a_{j}$ el numero de veces que aparece el valor $j$. Entonces
            \begin{equation*}
                a_{1}+a_{2}+\cdots+a_{n}=k,\qquad a_{j}\geq0.
            \end{equation*}
            Cada solución $(a_{1},\dots,a_{n})$ determina exactamnete una sucesión no decreciente.
            El numero de soluciones enteras no negativas de esa ecuación es el numero de combinaciones con
            repetición
            \begin{equation*}
                \binom{n+k-1}{k}.
            \end{equation*}
            Así, la probabilidad es
            \begin{equation*}
                \mathbb{P}(x_{1}\leq\cdots\leq x_{k})=\frac{\binom{n+k-1}{k}}{n^{k}}.
            \end{equation*}
        \end{enumerate}
    \end{solucion}
    
    \begin{mdframed}[style=mdbluebox,frametitle={Ejercicio 2}]
        ¿Tres personas están en el primer piso de un edificio de diez pisos y cada una elige de manera 
        aleatoria un piso de los nueve restantes al cual ir. ¿Cuál es la probabilidad de que las tres 
        personas vayan a pisos consecutivos?
        \begin{enumerate}
            \item[(a)] Si las elecciones son independientes, es decir, se permite que dos o tres vayan al mismo piso.
            \item[(b)] Si las elecciones no son independientes, en el sentido de que deben ir a pisos distintos.
        \end{enumerate}
    \end{mdframed}

    \begin{solucion}
        \begin{enumerate}
            \item[(a)] Si las elecciones son independientes, es decir, se permite que dos o tres vayan al mismo piso.\\
            Cada persona elige independientemente uno de los 9 pisos, por tanto
            \begin{equation*}
                \abs{\Omega}=9^{3}.
            \end{equation*}
            Cada resultado tiene probabilidad $\frac{1}{9^{3}}$. Fijemos una terna consecutiva $\{k,k+1,k+2\}$.
            Para que ocurra el evento $A$, las tres personas deben ocupar exactamente esos tres pisos, pues si
            hubiera repetición no habría tres pisos distintos consecutivos. Para cada bloque existen
            \begin{equation*}
                3!=6
            \end{equation*}
            formas de asignar esos pisos a las tres personas. Como existen 7 bloques consecutivos,
            \begin{equation*}
                \abs{A}=7\cdot3!=42.
            \end{equation*}
            Así, la probabilidad es
            \begin{equation*}
                \mathbb{P}_{a}(A)=\frac{42}{9^{3}}=\frac{42}{729}=\frac{14}{243}.
            \end{equation*}
            \item[(b)] Si las elecciones no son independientes, en el sentido de que deben ir a pisos distintos.\\
            Ahora el espacio muestral está formado por ternas ordenadas sin repetición. Número de maneras
            de asignar 3 pisos distintos a 3 personas
            \begin{equation*}
                \abs{\Omega^{\prime}}=P(9,3)=9\cdot8\cdot7=504.
            \end{equation*}
            El conteo favorable es el mismo que el anterior
            \begin{align*}
                \abs{A}=3\cdot3!=42.
            \end{align*}
            Así, la probabilidad es
            \begin{equation*}
                \mathbb{P}_{b}(A)=\frac{42}{504}=\frac{1}{12}.
            \end{equation*}
        \end{enumerate}
    \end{solucion}

    \begin{mdframed}[style=mdbluebox,frametitle={Ejercicio 3}]
        En un cajón de calcetines solamente $3$ son azules. Si al escoger dos calcetines al azar 
        tenemos que la probabilidad de que ambos sean azules es de $\frac{1}{2}$, 
        ¿cuántos calcetines en total tiene el cajón?
    \end{mdframed}

    \begin{solucion}
        Sea $N$ el número total de calcentines en el cajón. Se sabe que exactamente 3 son azules y que se extraen
        dos calcetines al azar sin reemplazo. El espacio muestral está formado por todas las parejas no ordenadas
        de calcetines
        \begin{equation*}
            \abs{\Omega}=\binom{N}{2}.
        \end{equation*}
        El evento $A$ es: ambos calcetines son azules. El numero de parejas favorables es
        \begin{equation*}
            \abs{A}=\binom{3}{2}
        \end{equation*}
        Por tanto
        \begin{equation*}
            \mathbb{P}(A)=\frac{\binom{3}{2}}{\binom{N}{2}}.
        \end{equation*}
        Sustituyendo los valores
        \begin{equation*}
            \binom{3}{2}=3, \qquad \binom{N}{2}=\frac{N(N-1)}{2}.
        \end{equation*}
        La condición del problema es
        \begin{equation*}
            \frac{\binom{3}{2}}{\binom{N}{2}}=\frac{1}{2}.
        \end{equation*}
        Sustituyendo
        \begin{equation*}
            \frac{3}{\frac{N(N-1)}{2}}=\frac{1}{2}
        \end{equation*}
        Luego,
        \begin{equation*}
            \frac{6}{N(N-1)}=\frac{1}{2}.
        \end{equation*}
        Por lo tanto
        \begin{align*}
            12=N(N-1)\\
            N^{2}-N-12=0.
        \end{align*}
        Factorizamos 
        \begin{equation*}
            (N-4)(N+3)=0.
        \end{equation*}
        Como $N\gt0$,
        \begin{equation*}
            N=4.
        \end{equation*}
        Así, el cajon tiene exactamente 4 calcetines en total.
    \end{solucion}

    \begin{mdframed}[style=mdbluebox,frametitle={Ejercicio 4}]
        Supongamos que tomamos $5$ cartas al azar de un mazo de póker estándar con $52$ cartas 
        (13 cartas numeradas por cada tipo entre corazones, tréboles, diamantes y picas). 
        Encuentre la probabilidad de obtener lo siguiente:
        \begin{enumerate}
            \item[(a)] Cinco cartas del mismo tipo (palo).
            \item[(b)] Exactamente un par (dos cartas con la misma numeración).
            \item[(c)] Exactamente dos pares distintos.
            \item[(d)] Exactamente una tercia (tres cartas con la misma numeración).
            \item[(e)] Póker (cuatro cartas con la misma numeración).
        \end{enumerate}
    \end{mdframed}

    \begin{solucion}
        El espacio muestral es
        \begin{align*}
            \Omega=\{\text{manos de 5 cartas}\},\qquad \abs{\Omega}=\binom{52}{5}=2598960.
        \end{align*}
        Por tanto, para cualquier evento $A$,
        \begin{equation*}
            \mathbb{P}(A)=\frac{\abs{A}}{\binom{52}{5}}.
        \end{equation*}
        \begin{enumerate}
            \item[(a)] Cinco cartas del mismo tipo\\
            Elegimos, el palo: 4 maneras. 5 cartas dentro de las 13 de ese palo $\binom{13}{5}$.
            Entonces
            \begin{align*}
                \abs{A}=4\binom{13}{4}.
            \end{align*}
            Así, la probabilidad es
            \begin{align*}
                \mathbb{P}(a)=\frac{4\binom{13}{5}}{\binom{52}{5}}=\frac{5148}{2598960}\approx0.00198.
            \end{align*}
            \item[(b)] Exactamente un par\\
            El valor del par: 13 maneras. Las 2 cartas de ese valor $\binom{4}{2}$. Los otros 3 valores distintos entre los 12 restantes: $\binom{12}{3}$.
            Para cada uno de esos valores, elegimos un palo: $4^{3}$.
            \begin{equation*}
                \abs{B}=13\binom{4}{2}\binom{12}{3}4^{3}.
            \end{equation*}
            Así, la probabilidad es
            \begin{equation*}
                \mathbb{P}(b)=\frac{13\binom{4}{2}\binom{12}{3}4^{3}}{\binom{52}{5}}=\frac{1098240}{2598960}\approx0.4226.
            \end{equation*}
            \item[(c)] Exactamente dos pares distintos\\
            Elegimos, los 2 valores de los pares: $\binom{13}{2}$. Para cada par: $\binom{4}{2}^{2}$.
            El valor de la quinta carta: 11. Su palo: 4.
            \begin{equation*}
                \abs{C}=\binom{13}{4}\binom{4}{2}^{2}\cdot11\cdot4.
            \end{equation*}
            Así, la probabilidad es
            \begin{equation*}
                \mathbb{P}(c)=\frac{123552}{2598960}\approx0.0475.
            \end{equation*}
            \item[(d)] Exactamente una tercia\\
            Valor de la tercia: 13. Las 3 cartas: $\binom{4}{3}$. 2 valores distintos entre los 12 restantes:
            $\binom{12}{2}$. Palo de cada una $4^{2}$.
            \begin{equation*}
                \abs{D}=13\binom{4}{3}\binom{12}{2}4^{2}.
            \end{equation*}
            Así, la probabilidad es
            \begin{equation*}
                \mathbb{P}(d)=\frac{54912}{2598960}\approx0.0211.
            \end{equation*}
            \item[(e)] Póker\\
            Valor del póker: 13. Las 4 cartas: $\binom{4}{4}=1$. La quinta carta: 12. Su palo: 4.
            \begin{equation*}
                \abs{E}=13\cdot12\cdot4=624.
            \end{equation*}
            Así, la probabilidad es
            \begin{equation*}
                \mathbb{P}(e)=\frac{624}{2598960}\approx0.00024.
            \end{equation*}
        \end{enumerate}
    \end{solucion}

    \begin{mdframed}[style=mdbluebox,frametitle={Ejercicio 5}]
        Cierta comunidad se compone de $20$ familias en total. De estas, 
        $4$ familias tienen exactamente un hij@, 
        $8$ familias tienen exactamente dos hijos, 
        $5$ familias tienen exactamente tres hijos, 
        $2$ familias tienen exactamente cuatro hijos y 
        $1$ familia tiene exactamente cinco hijos.

        \begin{enumerate}
            \item[(a)] Si escogemos una familia al azar, ¿cuál es la probabilidad de que dicha familia tenga $k$ hijos para $k=1,2,\dots,5$?
            \item[(b)] Si escogemos un niño al azar, ¿cuál es la probabilidad de que dicho niñ@ pertenezca a una familia con exactamente $k$ hijos para $k=1,2,\dots,5$?
        \end{enumerate}
    \end{mdframed}

    \begin{solucion}
        El número total de familias es
        \begin{equation*}
            4+8+5+2+1=20.
        \end{equation*}
        El número total de niños es
        \begin{equation*}
            4(1)+8(2)+5(3)+2(4)+1(15)=48.
        \end{equation*}
        \begin{enumerate}
            \item[(a)] Si escogemos una familia al azar, ¿cuál es la probabilidad de que dicha familia tenga $k$ hijos para $k=1,2,\dots,5$?\\
            Aquí el espacio muestral son las 20 familias. Por tanto,
            \begin{equation*}
                \mathbb{P}(k)=\frac{\text{familias con }k\text{ hijos}}{20}
            \end{equation*}
            Explicitamente,
            \begin{align*}
                \mathbb{P}(1)&=\frac{4}{20}=\frac{1}{5},\\
                \mathbb{P}(2)&=\frac{8}{20}=\frac{2}{5},\\
                \mathbb{P}(3)&=\frac{5}{20}=\frac{1}{4},\\
                \mathbb{P}(4)&=\frac{2}{20}=\frac{1}{10},\\
                \mathbb{P}(5)&=\frac{1}{20}.
            \end{align*}
            \item[(b)] Si escogemos un niño al azar, ¿cuál es la probabilidad de que dicho niñ@ pertenezca a una familia con exactamente $k$ hijos para $k=1,2,\dots,5$?\\
            Ahora el espacio muestral son los 48 niños. Los niños que pertenecen a las familias con $k$ hijos son:
            \begin{equation*}
                k\times(\text{familias con }k\text{ hijos}).
            \end{equation*}
            Por lo tanto,
            \begin{equation*}
                \mathbb{P}(k)=\frac{k\cdot(\text{familias con }k\text{ hijos})}{48}.
            \end{equation*}
            Explicitamente,
            \begin{align*}
                \mathbb{P}(1)&=\frac{1\cdot4}{48}=\frac{4}{48}=\frac{1}{12},\\
                \mathbb{P}(2)&=\frac{2\cdot8}{48}=\frac{16}{48}=\frac{1}{3},\\
                \mathbb{P}(3)&=\frac{3\cdot5}{48}=\frac{15}{48}=\frac{5}{16},\\
                \mathbb{P}(4)&=\frac{4\cdot2}{48}=\frac{8}{48}=\frac{1}{6},\\
                \mathbb{P}(5)&=\frac{5\cdot1}{48}=\frac{5}{48}.
            \end{align*}
        \end{enumerate}
    \end{solucion}
    
    \begin{mdframed}[style=mdbluebox,frametitle={Ejercicio 6}]
        Supongamos que tenemos $m$ pelotas distintas y $m$ cajas distintas. 
        Si colocamos al azar cada una de las pelotas en alguna de las cajas, 
        ¿cuál es la probabilidad de que exactamente una caja quede vacía?
    \end{mdframed}

    \begin{solucion}
        Sea $m\geq2$. Cada pelota puede colocarse en cualquiera de las $m$ cajas de manera independiente, por lo que
        el número total de configuraciones es $m^{m}$. Queremos que exactamente una caja quede vacía. Hay $m$
        formas de elegirla. Como tenemos $m$ pelotas y $m-1$ cajas, y ninguna puede quedar vacía, necesariamente:
        una caja contendrá exactamente 2 pelotas, las demás $m-2$ cajas contendrán exactamente 1 pelota.
        No hay otra posibilidad, pues
        \begin{align*}
            2+(m-2)\cdot1=m.
        \end{align*}
        Hay $\binom{m}{2}$ formas de escogerlas. Fijadas las dos pelotas que irán juntas: elegimos la caja donde irán esas
        dos pelotas, distribuimos las restantes $m-2$ pelotas, una en cada una de las otras $m-2$ cajas.
        Numero de formas
        \begin{equation*}
            (m-1)(m-2)!=(m-1)!
        \end{equation*}
        Multiplicando
        \begin{equation*}
            N=m\binom{m}{2}(m-1)!.
        \end{equation*}
        Así, la probabilidad es
        \begin{equation*}
            \mathbb{P}=\frac{m\binom{m}{2}(m-1)!}{m^{m}}.
        \end{equation*}
    \end{solucion}

    \begin{mdframed}[style=mdbluebox,frametitle={Ejercicio 7}]
        Cierta urna contiene $46$ pelotas en total: $12$ rojas, $16$ azules y $18$ verdes. 
        Si tomamos $7$ pelotas de la urna de manera aleatoria, 
        ¿cuál es la probabilidad de obtener cada uno de los siguientes escenarios?

        \begin{enumerate}
            \item[(a)] $3$ rojas, $2$ azules y $2$ verdes.
            \item[(b)] Al menos $2$ rojas.
            \item[(c)] Todas las pelotas del mismo color.
            \item[(d)] Exactamente $3$ rojas o exactamente $3$ azules.
        \end{enumerate}
    \end{mdframed}

    \begin{solucion}
        Sea el experimento consistente en extraer 7 pelotas sin reemplazo de una urna con
        \begin{align*}
            12\text{ rojas},\qquad16\text{ azules}\qquad18\text{ verdes},
        \end{align*}
        en total 46 pelotas. El espacio muestral tiene cardinalidad
        \begin{equation*}
            \binom{46}{7}.
        \end{equation*}
        \begin{enumerate}
            \item[(a)] $3$ rojas, $2$ azules y $2$ verdes.
            \begin{equation*}
                \mathbb{P}=\frac{\binom{12}{3}\binom{16}{2}\binom{18}{2}}{\binom{46}{7}}
            \end{equation*}
            \item[(b)] Al menos 2 rojas.\\
            Sea $X$ el número de pelotas rojas extraídas.
            \begin{align*}
                \mathbb{P}(X\geq2)=1-\mathbb{P}(X=0)-\mathbb{P}(X=1).\\
            \end{align*}
            Con
            \begin{align*}
                \mathbb{P}(X=0)=\frac{\binom{34}{7}}{\binom{46}{7}}
            \end{align*}
            y
            \begin{align*}
                \mathbb{P}(X=1)=\frac{\binom{12}{1}\binom{34}{6}}{\binom{46}{7}}.
            \end{align*}
            Por tanto,
            \begin{align*}
                \mathbb{P}(x\geq2)=1-\frac{\binom{34}{7}}{\binom{46}{7}}-\frac{12\binom{34}{6}}{\binom{46}{7}}.
            \end{align*}
            \item[(c)] Todas del mismo color\\
            Tres casos mutuamente excluyentes
            \begin{align*}
                7\text{ rojas: }\binom{12}{7},\\
                7\text{ azules: }\binom{16}{7},\\
                7\text{ verdes: }\binom{18}{7}.
            \end{align*}
            Entonces
            \begin{align*}
                \mathbb{P}=\frac{\binom{12}{7}+\binom{16}{7}+\binom{18}{7}}{\binom{46}{7}}.
            \end{align*}
            \item[(d)] Exactamente 3 rojas o exactamente 3 azules.\\
            Sea $R$ el evento “exactamente 3 rojas” y $B$ el evento “exactamente 3 azules”.
            Estos eventos no son disjuntos, pues pueden ocurrir simultaneamente.
            \begin{align*}
                \mathbb{P}(R\cup B)=\mathbb{P}(R)+\mathbb{P}(B)-\mathbb{P}(R\cap B).
            \end{align*}
            Calculamos cada termino
            \begin{align*}
                \mathbb{P}(R)&=\frac{\binom{12}{3}\binom{34}{4}}{\binom{46}{7}}\\
                \mathbb{P}(B)&=\frac{\binom{16}{3}\binom{30}{4}}{\binom{46}{7}}\\
                \mathbb{P}(R\cap B)=\frac{\binom{12}{3}\binom{16}{3}\binom{18}{1}}{\binom{46}{7}}.
            \end{align*}
            Así, la probabilidad es
            \begin{align*}
                \mathbb{P}(R\cup B)=\frac{\binom{12}{3}\binom{34}{4}+\binom{16}{3}\binom{30}{4}-\binom{12}{3}\binom{16}{3}\binom{18}{1}}{\binom{46}{7}}
            \end{align*}
        \end{enumerate}
    \end{solucion}

    \begin{mdframed}[style=mdbluebox,frametitle={Ejercicio 8}]
        Consideremos el siguiente reordenamiento aleatorio de $n$ elementos. 
        Iniciamos con los números $1,2,\dots,n$ en ese mismo orden. Luego, en el primer paso, 
        lanzamos una moneda balanceada con lados $S$ (sol) y $A$ (águila). 

        Si sale sol, dejamos al $1$ en su lugar y pasamos al siguiente número. 
        Si sale águila, colocamos $1$ al final de la fila y pasamos al siguiente número. 
        Repetimos este procedimiento hasta llegar al número $n$. 

        Por ejemplo, para $n=4$, si la sucesión de lanzamientos es $S A A S$, 
        el orden inicial $1,2,3,4$ se convierte en $1,4,2,3$.

        Calcule la probabilidad de que $1,2,\dots,n$ terminen en el mismo orden.
    \end{mdframed}

    \begin{solucion}
        Sea $\omega=(\omega_1,\dots,\omega_n)\in\{S,A\}^n$ la sucesión de resultados de los lanzamientos de la moneda. 
        Cada lanzamiento es independiente y balanceado, por lo que todas las $2^n$ sucesiones posibles son equiprobables y cada una ocurre con probabilidad $2^{-n}$.

        El procedimiento descrito tiene la siguiente estructura: al procesar los números $1,2,\dots,n$ en orden creciente, si en el paso $i$ ocurre $S$, el número $i$ permanece en su posición relativa actual; mientras que si ocurre $A$, el número $i$ se envía al final de la fila en ese instante. Obsérvese que los elementos que se envían al final conservan entre sí el orden creciente en el que fueron procesados.

        Si definimos
        \[
        K=\{\, i : \omega_i=S \,\}, 
        \qquad 
        L=\{\, i : \omega_i=A \,\},
        \]
        entonces el orden final de la fila queda determinado por la concatenación de los elementos de $K$ en orden creciente seguida de los elementos de $L$ también en orden creciente. Es decir, la permutación resultante siempre es de la forma
        \[
        (k_1<k_2<\dots<k_r,\;\ell_1<\ell_2<\dots<\ell_{n-r}).
        \]

        Para que el orden final coincida con $(1,2,\dots,n)$ es necesario y suficiente que no exista ningún par $i<j$ tal que $\omega_i=A$ y $\omega_j=S$, pues en tal caso el elemento $j$ aparecería antes que $i$ en la disposición final, produciendo una inversión. Por consiguiente, la única manera de conservar el orden natural es que la sucesión de lanzamientos tenga la forma
        \[
        \underbrace{S\,S\,\dots\,S}_{r\text{ veces}}
        \underbrace{A\,A\,\dots\,A}_{n-r\text{ veces}},
        \]
        para algún $r\in\{0,1,\dots,n\}$.

        Para cada valor posible de $r$ existe exactamente una sucesión de esta forma, de modo que el número total de sucesiones favorables es $n+1$. Dado que el espacio muestral tiene cardinal $2^n$, se concluye que
        \[
        \mathbb{P}\big((1,2,\dots,n)\text{ es el orden final}\big)
        =
        \frac{n+1}{2^n}.
        \]
    \end{solucion}

\end{document}

\message{ !name(tarea2.tex) !offset(-477) }
