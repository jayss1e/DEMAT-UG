\documentclass[12pt,a4paper]{article}
\usepackage{ugmath}
\usepackage{float}
\usepackage{placeins}
\usepackage{array}
\usepackage[labelfont=bf,labelsep=space]{caption}
\newcommand{\paren}[1]{\left( #1 \right)}
\newcommand{\alumno}{Ricardo León Martínez}
\newcommand{\materia}{Elementos de Probabilidad y Estadistica}
\newcommand{\profesor}{Claudia Reynoso Alcántara}
\newcommand{\tarea}{Tarea 2}
\newcommand{\fecha}{17/2/2026}

\begin{document}

    \begin{center}
        {\large\textbf{UNIVERSIDAD DE GUANAJUATO}}\\[0.3cm]
        {\normalsize\textbf{DIVISIÓN DE CIENCIAS NATURALES Y EXACTAS}}\\
        {\normalsize\textbf{CAMPUS GUANAJUATO}}\\[1cm]

        {\Large\textbf{\tarea\ (\materia)}}\\[1cm]
    \end{center}

    \noindent
    \textbf{Nombre:} \alumno 
    \hfill 
    \textbf{Fecha:} \fecha 
    \hfill 
    \textbf{Calificación:} \rule{3cm}{0.4pt}

    \vspace{0.3cm}

    \begin{mdframed}[style=mdbluebox,frametitle={Ejercicio 1}]
        En una tómbola se colocan $n$ pelotas identificadas cada una por un número del $1$ al $n$. 
        En cada una de $k$ ocasiones consecutivas, hacemos girar la tómbola, tomamos una pelota al azar, 
        anotamos el número de la pelota que tomamos y, finalmente, la devolvemos a la tómbola.
        \begin{enumerate}
            \item[(a)] ¿Cuál es la probabilidad de que la sucesión de números obtenida sea estrictamente creciente?
            \item[(b)] ¿Cuál es la probabilidad de que la sucesión de números obtenida sea no decreciente?
        \end{enumerate}
    \end{mdframed}
    
    \begin{mdframed}[style=mdbluebox,frametitle={Ejercicio 2}]
        ¿Tres personas están en el primer piso de un edificio de diez pisos y cada una elige de manera 
        aleatoria un piso de los nueve restantes al cual ir. ¿Cuál es la probabilidad de que las tres 
        personas vayan a pisos consecutivos?
        \begin{enumerate}
            \item[(a)] Si las elecciones son independientes, es decir, se permite que dos o tres vayan al mismo piso.
            \item[(b)] Si las elecciones no son independientes, en el sentido de que deben ir a pisos distintos.
        \end{enumerate}
    \end{mdframed}

    \begin{mdframed}[style=mdbluebox,frametitle={Ejercicio 3}]
        En un cajón de calcetines solamente $3$ son azules. Si al escoger dos calcetines al azar 
        tenemos que la probabilidad de que ambos sean azules es de $\frac{1}{2}$, 
        ¿cuántos calcetines en total tiene el cajón?
    \end{mdframed}

    \begin{mdframed}[style=mdbluebox,frametitle={Ejercicio 4}]
        Supongamos que tomamos $5$ cartas al azar de un mazo de póker estándar con $52$ cartas 
        (13 cartas numeradas por cada tipo entre corazones, tréboles, diamantes y picas). 
        Encuentre la probabilidad de obtener lo siguiente:
        \begin{enumerate}
            \item[(a)] Cinco cartas del mismo tipo (palo).
            \item[(b)] Exactamente un par (dos cartas con la misma numeración).
            \item[(c)] Exactamente dos pares distintos.
            \item[(d)] Exactamente una tercia (tres cartas con la misma numeración).
            \item[(e)] Póker (cuatro cartas con la misma numeración).
        \end{enumerate}
    \end{mdframed}

    \begin{mdframed}[style=mdbluebox,frametitle={Ejercicio 5}]
        Cierta comunidad se compone de $20$ familias en total. De estas, 
        $4$ familias tienen exactamente un hij@, 
        $8$ familias tienen exactamente dos hijos, 
        $5$ familias tienen exactamente tres hijos, 
        $2$ familias tienen exactamente cuatro hijos y 
        $1$ familia tiene exactamente cinco hijos.

        \begin{enumerate}
            \item[(a)] Si escogemos una familia al azar, ¿cuál es la probabilidad de que dicha familia tenga $k$ hijos para $k=1,2,\dots,5$?
            \item[(b)] Si escogemos un niño al azar, ¿cuál es la probabilidad de que dicho niñ@ pertenezca a una familia con exactamente $k$ hijos para $k=1,2,\dots,5$?
        \end{enumerate}
    \end{mdframed}
    
    \begin{mdframed}[style=mdbluebox,frametitle={Ejercicio 6}]
        Supongamos que tenemos $m$ pelotas distintas y $m$ cajas distintas. 
        Si colocamos al azar cada una de las pelotas en alguna de las cajas, 
        ¿cuál es la probabilidad de que exactamente una caja quede vacía?
    \end{mdframed}

    \begin{mdframed}[style=mdbluebox,frametitle={Ejercicio 7}]
        Cierta urna contiene $46$ pelotas en total: $12$ rojas, $16$ azules y $18$ verdes. 
        Si tomamos $7$ pelotas de la urna de manera aleatoria, 
        ¿cuál es la probabilidad de obtener cada uno de los siguientes escenarios?

        \begin{enumerate}
            \item[(a)] $3$ rojas, $2$ azules y $2$ verdes.
            \item[(b)] Al menos $2$ rojas.
            \item[(c)] Todas las pelotas del mismo color.
            \item[(d)] Exactamente $3$ rojas o exactamente $3$ azules.
        \end{enumerate}
    \end{mdframed}

    \begin{mdframed}[style=mdbluebox,frametitle={Ejercicio 7}]
        Consideremos el siguiente reordenamiento aleatorio de $n$ elementos. 
        Iniciamos con los números $1,2,\dots,n$ en ese mismo orden. Luego, en el primer paso, 
        lanzamos una moneda balanceada con lados $S$ (sol) y $A$ (águila). 

        Si sale sol, dejamos al $1$ en su lugar y pasamos al siguiente número. 
        Si sale águila, colocamos $1$ al final de la fila y pasamos al siguiente número. 
        Repetimos este procedimiento hasta llegar al número $n$. 

        Por ejemplo, para $n=4$, si la sucesión de lanzamientos es $S A A S$, 
        el orden inicial $1,2,3,4$ se convierte en $1,4,2,3$.

        Calcule la probabilidad de que $1,2,\dots,n$ terminen en el mismo orden.
    \end{mdframed}

\end{document}