\documentclass[12pt,a4paper]{article}
\usepackage{ugmath}
\usepackage{float}
\usepackage{placeins}
\usepackage[labelfont=bf,labelsep=space]{caption}
\newcommand{\alumno}{Ricardo León Martínez}
\newcommand{\materia}{Elementos de Probabilidad y Estadistica}
\newcommand{\profesor}{Josué Daniel Vázquez Becerra}
\newcommand{\tarea}{Tarea 1}
\newcommand{\fecha}{10/2/2026}

\begin{document}
    \begin{center}
    {\large\textbf{UNIVERSIDAD DE GUANAJUATO}}\\[0.3cm]
    {\normalsize\textbf{DIVISIÓN DE CIENCIAS NATURALES Y EXACTAS}}\\
    {\normalsize\textbf{CAMPUS GUANAJUATO}}\\[1cm]

    {\Large\textbf{\tarea\ (\materia)}}\\[1cm]
    \end{center}

    \textbf{Nombre:} \alumno \hfill 
    \textbf{Fecha:} \fecha \hfill 
    \textbf{Calificación:} \rule{3cm}{0.4pt} \\[0.3cm]

    Parte I. Conteo

    \begin{mdframed}[style=mdbluebox,frametitle={Ejercicio 1}]
        Cierta semana, Fernanda desea salir a cenar cada noche de lunes a viernes a algunos de sus
        siete restaurantes favoritos de Guanajuato.
        \begin{enumerate}
            \item[(a)] Si Fernanada no desea cenar en un mismo restaurante más de una vez, ¿de cuántas
            maneras diferentes puede organizar sus salidas?
            \item[(b)] Fernanda reconsideró su postura, y estaría dispuesta a cenar en el mismo lugar más
            de una vez, bajo la condición de no hacerlo en días consecutivos. ¿De cuantas maneras diferentes
            puede hacer esto?
        \end{enumerate}
    \end{mdframed}

    \begin{solucion}
        \begin{enumerate}
            \item[(a)]Si Fernanada no desea cenar en un mismo restaurante más de una vez, ¿de cuántas
            maneras diferentes puede organizar sus salidas?\\
            Son 5 días distintos y hay 7 restaurantes, sin repetir. Se usa la formaula de seleccion
            ordenada sin reemplazo
            \begin{align*}
                V=\frac{n!}{(n-k)!}
            \end{align*}
            Aquí $n=7$,$k=5$
            \begin{align*}
                V=\frac{7!}{2!}=2520
            \end{align*}
            Por lo tanto hay 2520 maneras de organizar su salida.

            \item[(b)] Fernanda reconsideró su postura, y estaría dispuesta a cenar en el mismo lugar más
            de una vez, bajo la condición de no hacerlo en días consecutivos. ¿De cuantas maneras diferentes
            puede hacer esto?\\
            Se permite repetir, pero no en dias consecutivos, en el primer dia hay 7 opciones. Cada
            dia siguiente no puede repetir el anterior por lo tanto cada por cada dia hay 6 opciones.
            Así, por el principio multiplicativo
            \begin{align*}
                7\cdot6^{4}=7\cdot1296=9072.
            \end{align*}
            Por lo tanto hay 9072 maneras de hacer esto.
        \end{enumerate}
    \end{solucion}

    \begin{mdframed}[style=mdbluebox,frametitle={Ejercicio 2}]
        Una baraja de póquer contiene un total de 52 cartas, 12 cartas distintas por cada tipo entre corazones,
        tréboles, diamantes y picas. Para cierto juego, Sofía, Luisa, Rubén y Marco recibirán una mano de 13 cartas
        cada uno.
        \begin{enumerate}
            \item[(a)] ¿De cuántas maneras distintas puede Sofía recibir su mano?
            \item[(b)] ¿De cuántas maneras distintas pueden los cuatro jugadores recibir sus manos?
            \item[(c)] Explique el por qué la respuesta a la parte (b) no es la respuesta a la parte (a) elevada
            a la cuarta potencia.
        \end{enumerate}
    \end{mdframed}

    \begin{solucion}
        \begin{enumerate}
            \item[(a)]¿De cuántas maneras distintas puede Sofía recibir su mano?\\
            Sofia tiene 13 cartas de una baraja de 52, y el orden no importa. Es una selccion no rodenada sin
            reemplazo
            \begin{align*}
                \binom{52}{13}
            \end{align*}
            Por lo tanto hay $\binom{52}{13}$ maneras distintas de recibir su mano.
            \item[(b)] ¿De cuántas maneras distintas pueden los cuatro jugadores recibir sus manos?\\
            Las manos se reparten simultaneamente, sin importar el orden de las cartas en cada jugador, viendolo
            por etapas. Mano de sofia: $\binom{52}{13}$, mano de luisa quedando 39 cartas $\binom{39}{13}$,
            mano de ruben quedando 26 cartas $\binom{26}{13}$ y por ultimo la mano de marco quedando 13 cartas
            $\binom{13}{13}=1$. Así, por el principio multiplicativo
            \begin{align*}
                \binom{52}{13}\binom{39}{13}\binom{26}{13}\binom{13}{13}
            \end{align*}
            Por lo tanto hay $\binom{52}{13}\binom{39}{13}\binom{26}{13}\binom{13}{13}$ maneras distintas
            de recibir sus manos.
            \item[(c)] Explique el por qué la respuesta a la parte (b) no es la respuesta a la parte (a) elevada
            a la cuarta potencia.\\
            Porque las elecciones no son idependientes. Despues de repartir la mano de sofia, ya no quedan
            52 cartas, sino 39, despues 26 y luego 13. Elevar $\binom{52}{13}$ a la cuerta implicaria suponer
            que cada jugador elige sus 13 cartas de una baraja completa, lo cual no ocurre.
        \end{enumerate}
    \end{solucion}

    \begin{mdframed}[style=mdbluebox,frametitle={Ejercicio 3}]
        Supongamos que queremos formar una fila de ocho personas identificadas por letras $A,B,C,D,E,\\
        F,G,H$.
        ¿De cuántas maneras distintas podemos hacer esto en cada un de los siguientes casos?
        \begin{enumerate}
            \item[(a)] Si no hay ninguna restriccion sobre el orden de la fila.
            \item[(b)] Si las personas $D$ y $H$ deben estar sentadas de manera consecutiva.
            \item[(c)] Si $A,B,C,D$ son hombres, mientras que $E,F,G,H$ son mujeres, y ninguna cantidad de mujeres
            puede estar formada de maneras consecutivas.
            \item[(d)] Si hay cuatro parejas (digamos $A$ y $H$, $B$ y $G$, $C$ y $F$, $D$ y $E$), y cada persona
            debe estar sentada junto con su pareja. 
        \end{enumerate}
    \end{mdframed}

    \begin{solucion}
        \begin{enumerate}
            \item[(a)] Si no hay ninguna restriccion sobre el orden de la fila.\\
            Se trata de ordenar 8 personas distinas en una fila
            \begin{align*}
                8!
            \end{align*}
            Asi, hay $8!$ maneras distintas.
            \item[(b)] Si las personas $D$ y $H$ deben estar sentadas de manera consecutiva.\\
            Considerando el bloque $(D,H)$ como una sola unidad. El bloque puede estar como
            $DH$ o $HD$ osea 2 formas. Juntos con las otras 6 personas, hay 7 objetos a ordenar.
            Entonces
            \begin{align*}
                7!\cdot2
            \end{align*}
            Por lo tanto hay $7!\cdot2$ maneras distintas.
            \item[(c)] Si $A,B,C,D$ son hombres, mientras que $E,F,G,H$ son mujeres, y ninguna cantidad de mujeres
            puede estar formada de maneras consecutivas.\\
            Primero acomodamos a los 5 hombres $A,B,C,D$
            \begin{align*}
                4!
            \end{align*}
            Esto genera 5 espacios posibles donde colocar mujeres
            \begin{align*}
                \_H\_H\_H\_H\_
            \end{align*}
            Elegimos 4 de esos 5 espacios para colocar mujeres
            \begin{align*}
                \binom{5}{4}
            \end{align*}
            Luego ordenamos a las 4 mujeres distintas
            \begin{align*}
                4!
            \end{align*}
            Por el principio multiplicativo
            \begin{align*}
                4!\cdot\binom{5}{4}\cdot4!
            \end{align*}
            Así, hay $4!\binom{5}{4}4!$ maneras distintas.
            \item[(d)] Si hay cuatro parejas (digamos $A$ y $H$, $B$ y $G$, $C$ y $F$, $D$ y $E$), y cada persona
            debe estar sentada junto con su pareja.\\
            Cada pareja se considera como un bloque. Hay 4 bloques y se pueden ordenar de $4!$ formas.
            Dentro de cada pareja hay 2 ordenes posibles. Entonces
            \begin{align*}
                4!\cdot2^{4}
            \end{align*}
            Por lo tanto hay $4!\cdot2^{4}$ maneras distintas.
        \end{enumerate}
    \end{solucion}

    \begin{mdframed}[style=mdbluebox,frametitle={Ejercicio 4}]
        Un grupo de cuatro estudiantes quiere ordenar dos pizzas de Peter´s Pizza. Cada pizza
        puede ser pequeña, mediana o grande. Además, cada pizza puede llevar desde cero hasta cinco
        ingredientes (adicionales a la salsa de tomate y queso de la base). ¿De cuántas maneras posibles
        pueden ordenar sus dos pizzas?
    \end{mdframed}

    \begin{solucion}
        Para cada pizza hay dos decisiones independientes. Primero se elige el tamaño,
        que puede ser pequeña, mediana o grande, lo cual da 3 posibilidades. Despues se eligen
        los ingredientes adicionales, como hay 5 ingredientes posibles y se puede escoger cualquiera
        de ellos o ninguno, el numero de elecciones posibles de ingredientes es $2^{5}$. Por el
        principio multiplicativo, una sola pizza puede pedirse de $3\cdot2^{5}=96$ maneras distintas.
        Como se desean ordenar dos pizzas y cada una se elige de manera independiente de la otra, el
        número total de formas de hacer el pedido es el cuadrado de esa cantidad, es decir, 
        \begin{align*}
            96^{2}=9216
        \end{align*}
        Por lo tanto, pueden ordenar sus dos pizzas de 9216 maneras distintas.
    \end{solucion}

    Para los siguientes tres ejercicios, basta dar argumentos combinatorios (narrativos).

    \begin{mdframed}[style=mdbluebox,frametitle={Ejercicio 5}]
        Demuestre la siguiente igualdad
        \begin{align*}
            \binom{2n}{n}=\binom{n}{0}^{2}+\binom{n}{1}^{2}+\cdots+\binom{n}{n}^{2}
        \end{align*}
    \end{mdframed}

    \begin{proof}
        Consideremos un conjunto de $2n$ elementos, particionado en dos subconjuntos disjuntos
        de tamaño $n$ cada uno. Por ejemplo, un conjunto $A$ con $n$ elementos y un conjunto $B$
        con $n$ elementos, con $A\cap B=\emptyset$. Elegimos directamente $n$ elementos de
        un conjunto de $2n$ elementos. El número de formas de hacerlo es
        \begin{align*}
            \binom{2n}{n}.
        \end{align*}
        Elegimos primero cuantos elementos tomar del conjunto $A$. Sea $k$ el número de elementos elegidos
        de $A$, con $0\leq k\leq n$. Entonces necesariamente se deben elegir $n-k$ elementos de $B$.
        El numero de formas de elegir $k$ elementos de $A$ es
        \begin{align*}
            \binom{n}{k}.
        \end{align*}
        El número de formas de elegir $n-k$ elementos de $B$ es
        \begin{align*}
            \binom{n}{n-k}.
        \end{align*}
        Por el principio multiplicativo
        \begin{align*}
            \binom{n}{k}\binom{n}{n-k}
        \end{align*}
        formas de hacer la eleccion. Al permitir que $k$ recorra todos los valores de 0 a $n$, el número
        total de formas es
        \begin{align*}
            \sum_{k=0}^{n}\binom{n}{k}\binom{n}{n-k}
        \end{align*}
        Ambos procedimientos cuentan el mismo conjunto, por lo que los resultados son iguales. Así,
        \begin{align*}
            \binom{2n}{n}=\sum_{k=0}^{n}\binom{n}{k}\binom{n}{n-k}=\sum_{k=0}^{n}\binom{n}{k}^{2}.
        \end{align*}
    \end{proof}

    \begin{mdframed}[style=mdbluebox,frametitle={Ejercicio 6}]
        Demuestre la igualdad del bastón de Hockey
        \begin{align*}
            \binom{n+1}{k+1}=\binom{k}{k}+\binom{k+1}{k}+\cdots+\binom{n}{k}
        \end{align*}
    \end{mdframed}

    \begin{proof}
        Fijemos $n$ y $k$ con $0\leq k\leq n$. Consideremos un conjunto con $n+1$ elementos
        y fijemos a uno de ellos, al que llamaremos $x$. Elegimos directamente $k+1$ elementos de un conjunto
        con $n+1$ elementos. El numero de formas de hacerlo es
        \begin{align*}
            \binom{n+1}{k+1}.
        \end{align*}
        Contamos los subconjuntos de tamaño $k+1$ segun la posicion de $x$. Supongamos que el subconjunto elegido,
        el elemento $x$ ocupa la posición mas grande al ordenar los elementos de manera fija. Entonces
        los otros $k$ elementos deben elegirse entre los elementos que son menores que $x$. Si $x$ es el
        elemento numero $1+j$, entonces hay exactamente $j$ elementos menores que el, y el numero de formas de
        elegir los otros $k$ elementos es
        \begin{align*}
            \binom{j}{k}.
        \end{align*}
        Al permitir que $j$ recorra todos los valores desde $k$ hasta $n$, se obtiene todas las posibilidades,
        y el numero total de subconjuntos es
        \begin{align*}
            \binom{k}{k}+\binom{k+1}{k}+\cdots+\binom{n}{k}
        \end{align*}
        Ambos procedimientos cuentan el mismo conjunto de subconjuntos de tamaño $k+1$. Por lo que
        los resultados son iguales
        \begin{align*}
            \binom{n+1}{k+1}=\binom{k}{k}+\binom{k+1}{k}+\cdots+\binom{n}{k}.
        \end{align*}
    \end{proof}

    \begin{mdframed}[style=mdbluebox,frametitle={Ejercicio 7}]
        Consideremos una carrera atlética con $m$ corredores en la que lso empates simultáneos son
        posibles. Por ejemplo, si hubiera 7 corredores, es posible tener 3 en primer lugar, 2 en segundo
        y 1 en tercero. Denotemos por $N(m)$ el número total de posibles resultados. Notemos que $N(1)=1$
        mientras que $N(2)=3$.
        \begin{enumerate}
            \item[(a)] Enliste todos los posibles resultados para $m=3$.
            \item[(b)] Tomemos $N(0):=1$. Demuestre que se cumplen las relaciones recursivas
            \begin{align*}
                N(m)=\sum_{k=1}^{m}\binom{m}{k}N(m-k)=\sum_{k=0}^{m-1}\binom{m}{m-k}N(k)
            \end{align*} 
        \end{enumerate}
    \end{mdframed}
    \begin{solucion}
        \begin{enumerate}
            \item[(a)] Enliste todos los posibles resultados para $m=3$.\\
            Para $m=3$, hay tres posibles resultados. No hay empates: $1^{\circ},2^{\circ},3^{\circ}$.
            Empate de dos corredores: 2 en primer lugar y 1 en segundo y 1 en primer lugar y 2 en segundo.
            Empate total: 3 corredores en primer lugar. Por lo tanto, los tipos posibles son
            \begin{align*}
                (1,1,1), \quad (2,1), \quad (1,2), \quad (3).
            \end{align*}
            $(1,1,1)$: $3!=6$ resultados. $(2,1)$: elegir los 2 del empate $\binom{3}{2}=3$. $(1,2)$:
            elegir el que queda solo $\binom{3}{1}=3$. (3): 1 resultado. Entonces
            \begin{align*}
                N(3)=6+3+3+1=13.
            \end{align*}
            \item[(b)] Tomemos $N(0):=1$. Demuestre que se cumplen las relaciones recursivas\\
            Sea $k$ el número de corredores que ocupan el primer lugar, con
            \begin{align*}
                1\leq k\leq m.
            \end{align*}
            Elegimos cuales $k$ corredores quedan en primer lugar, esto puede hacerse de $\binom{m}{k}$
            maneras. Los $m-k$ corredores restantes pueden ordenarse en cualquier resultado valido, con
            empates permitidos. Por definicion, eso puede hacerse de $N(m-k)$ maneras. Por el principio
            multiplicativo, para un valor fijo de $k$ hay
            \begin{align*}
                \binom{m}{k}N(m-k)
            \end{align*}
            resultados posibles. Al permitir que $k$ recorra todos los valores de 1 a $m$, se obtienen
            todos los resultados posibles, sin omisiones ni repeticiones. Por lo tanto,
            \begin{align*}
                N(m)=\sum_{k=1}^{m}\binom{m}{k}N(m-k).
            \end{align*}
            Haciendo el cambio de indicie $j=m-k$, cuando $k=1$ se tiene $j=m-1$ y cuando $k=m$ se tiene
            $j=0$. Entonces,
            \begin{align*}
                 N(m)=\sum_{j=0}^{m-1}\binom{m}{m-j}N(j).
            \end{align*}
        \end{enumerate}
    \end{solucion}
\end{document}