\documentclass[12pt,a4paper]{book}
\raggedbottom

\usepackage{ugmath}
\usepackage{float}
\usepackage{placeins}
\usepackage[labelfont=bf,labelsep=space]{caption}

\begin{document}
    \begin{titlepage}
        \centering
        
        \rule{1\textwidth}{0.4pt}
        \vspace{-2.2em}  % Ajuste negativo para acercar

        {\Large\bfseries Cálculo Integral de Una Variable\par}

        \vspace{-0.9em}  % Ajuste negativo para acercar
        \rule{1\textwidth}{0.4pt}

        \vspace{2.5cm}

        \includegraphics[width=0.60\linewidth]{DEMAT_LOGO.png}

        \vspace{2.5cm}

        {\normalsize
        Notas del curso impartido por\par
        Fernando Nuñez Medina
        \vspace{0.2cm}
        \textbf{}\par
        \textit{Departamento de Matemáticas}\par
        \textit{Universidad de Guanajuato}\par
        }

        \vspace{1.3cm}

        {\normalsize
        Escritas por\par
        \vspace{0.2cm}
        \textbf{Ricardo León Martínez}\par
        \textit{Departamento de Matemáticas}\par
        \textit{Universidad de Guanajuato}\par
        }

    \end{titlepage}

    \tableofcontents
    \clearpage

    \chapter{La Integral}

    La integral de riemann, o simplemente la integral, es otro de los conceptos fundamentales
    del Cálculo. En este capítulo veremos su definición y sus propiedades básicas. Enbrte otras cosas,
    con la integral definiremos el área bajo la curva de una función (más adelante veremos con
    presición que significa esto) y posteriormente veremos que, sorprendentemente, está relacionada con
    la derivada.

    \section{Introducción}
    En este texto trabajaremos con la definición de integral dada por Gastón Darboux (1842-1917).
    En el apéndice A damos la definición original de integral dad por Bernard Riemann (1822-1866)
    y mostramos que ambas son equivalentes. En la practica, la definición de integral dada por Darboux
    es mas sencilla de manejar que la dada por Riemann, por lo cual en este texto trabajaremos con
    la definición de Darboux. De aquí en adelante, a menos que se especifique lo contrario, cuando
    consideremos un intervalo cerrado $[a,b]$, supondremos que $a\lt b$.

    \begin{mdframed}[style=mdbluebox,frametitle={Definición 1: Partición de un intervalo}]
        Una partición de un intervalo $[a,b]$ es una colección de puntos $P=\{t_{0},t_{1},\dots,t_{n}\}$
        tal que
        \begin{align*}
            a=t_{0}\lt t_{1}\lt\cdots\lt t_{n}=b.
        \end{align*}
    \end{mdframed}

    Denotaremos por $\mathcal{P}[a,b]$, o simplemente por $\mathcal{P}$ si no hay confusión,
    al conjunto de todas las particiones de $[a,b]$. Al número
    \begin{align*}
        \norm{P}=\max\{t_{i}-t_{i-1}:i=1,\dots,n\},
    \end{align*}
    Le llamaremos la norma de la partición $P$, es decir, $\norm{P}$ es el máximo de las
    longitudes de los intervalos $[t_{i-1},t_{i}],i=1,\dots n$.

    \begin{mdframed}[style=mdbluebox,frametitle={Definición 2: Refinamiento de una partición}]
        Dadas dos particiones $P$ y $Q$ de un intervalo $[a,b]$, se dice que $Q$ es un refinamiento
        de $P$ (o que $Q$ es mas fina que $P$) si $P\subset Q$. Notemos que $P\cup Q$ sigue siendo
        partición de $[a,b]$; a $P\cup Q$ se le llama el refinamiento común de $P$ y $Q$.
    \end{mdframed}

    \begin{mdframed}[style=mdbluebox,frametitle={Definición 3: Sumas inferiores y superiores}]
        Sea $f$ una función acotada en un intervalo $[a,b]$ y consideremos una partición $P=\{t_{0},t_{1},\dots,t_{n}\}$
        de $[a,b]$. Para cada $i=1,\dots,n$ definimos
        \begin{align*}
            m_{i}=\inf\{f(x):x\in[t_{i-1},t_{i}]\}
        \end{align*}
        y
        \begin{align*}
            M_{i}=\sup\{f(x):x\in[t_{i-1},t_{i}]\}.
        \end{align*}
        Definimos la suma inferior de $f$ respecto de $P$ como
        \begin{align*}
            L(f,P)=\sum_{i=1}^{n}m_{i}(t_{i}-t_{i-1})
        \end{align*}
        y la suma superior de $f$ respecto de $P$ como
        \begin{align*}
            U(f,P)=\sum_{i=1}^{n}M_{i}(t_{i}-t_{i-1})
        \end{align*}
    \end{mdframed}
    Es claro que $m_{i}\leq M_{i}$ para $i=1,\dots,n$; en consecuencia,
    \begin{align}
        L(f,P)\leq U(f,P).
    \end{align}
    La proposición siguiente establece otras propiedades de la sumas inferiores y de las 
    sumas superiores que usaremos posteriormente.

    \begin{mdframed}[style=mdbluebox,frametitle={Proposición 1}]
        Sea $f$ una función acotada en un intervalo $[a,b]$ y supongamos que $P$ y $Q$ son
        particiones de $[a,b]$. Se cumple lo siguiente:
        \begin{enumerate}
            \item[(a)] Si $Q$ es un refinamiento de $P$, es decir, si $P\subset Q$, entonces
            \begin{align*}
                L(f,P)\leq L(f,Q)\text{ y } U(f,Q)\leq U(f,P).
            \end{align*}
            \item[(b)]
            \begin{align}
                L(f,P)\leq U(f,Q).
            \end{align}
            \item[(c)]
            \begin{align}
                \sup_{P\in\mathcal{P}}L(f,P)\leq\inf_{P\in\mathcal{P}}U(f,P).
            \end{align}
        \end{enumerate}
        En base al inciso (a) podemos decir que las sumas inferiores (superiores) son
        monótonas crecientes (decrecientes) respecto a los refinamientos. El inciso (b)
        afirma que cualquier suma inferior es menor o igual que cualquier suma superior,
        sin importar que particiones se consideren
    \end{mdframed}
    \begin{proof}
        \begin{enumerate}
            \item[(a)] Supongamos que $P\subset Q$.Probaremos que $L(f,P)\leq L(f,Q)$, la
            prueba de que $U(f,Q)\leq U(f,P)$ es similar. Si $P=Q$ es claro que $L(f,P)\leq L(f,Q)$.
            Supongamos ahora que $Q$ tiene más puntos que $P$. Consideremos primero el caso en que $Q$
            contiene un punto más que $P$, digamos que $P\{t_{0},t_{1},\dots,t_{n}\}$ y que
            $Q=\{t_{0},t_{1},\dots,t_{j-1},c,t_{j},\dots,t_{n}\}$, para algún $1\leq j\leq n$.
            Sea $m_{i}$ como en la definición 3, es decir,
            \begin{align*}
                m_{i}=\inf\{f(x):x\in[t_{i-1},t_{i}\},\text{ }1\leq i\leq n.
            \end{align*}
            Definamos
            \begin{align*}
                s_{1}=\inf\{f(x):x\in[t_{j-1},c]\}
            \end{align*}
            y
            \begin{align*}
                s_{2}=\inf\{f(x):x\in[c,t_{j}]\}.
            \end{align*}
            Así, $m_{j}\leq s_{1}$ y $m_{j}\leq s_{2}$. En consecuencia,
            \begin{align*}
                L(f,P)&=\sum_{i=1}^{n}m_{i}(t_{i}-t_{t-1})\\
                &=\sum_{i=1}^{j-1}m_{i}(t_{j}-t_{j-1})+\sum_{i=j+1}^{n}m_{i}(t_{j}-t_{j-1})
            \end{align*}
        \end{enumerate}
    \end{proof}



\end{document}
