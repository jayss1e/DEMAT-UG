\documentclass[12pt,a4paper]{article}
\usepackage{ugmath}
\usepackage{float}
\usepackage{placeins}
\usepackage [spanish]{babel}
\usepackage[labelfont=bf,labelsep=space]{caption}
\newcommand{\paren}[1]{\left( #1 \right)}
\newcommand{\alumno}{Ricardo León Martínez}
\newcommand{\materia}{Calculo Diferencia e Integral II}
\newcommand{\profesor}{Fernando Nuñez Medina}
\newcommand{\tarea}{Tarea 4}
\newcommand{\fecha}{20/2/2026}

\begin{document}
    \begin{center}
    {\large\textbf{UNIVERSIDAD DE GUANAJUATO}}\\[0.3cm]
    {\normalsize\textbf{DIVISIÓN DE CIENCIAS NATURALES Y EXACTAS}}\\
    {\normalsize\textbf{CAMPUS GUANAJUATO}}\\[1cm]

    {\Large\textbf{\tarea\ (\materia)}}\\[1cm]
    \end{center}

    \textbf{Nombre:} \alumno \hfill 
    \textbf{Fecha:} \fecha \hfill 
    \textbf{Calificación:} \rule{3cm}{0.4pt} \\[0.3cm]

    \begin{mdframed}[style=mdbluebox,frametitle={Ejercicio 1}]
        \textbf{(Teorema del valor medio generalizado para integrales)} Sean $f$ y $g$ funciones integrables en un intervalo $[a,b]$. Prueba que si $f$ es continua y $g$ no cambia
        de signo, entonces existe $c\in[a,b]$ tal que
        \begin{align*}
            \int_{a}^{b}fg=f(c)\int_{a}^{b}g.
        \end{align*}
        \textbf{Sugerencia:} Supón primero que $g\geq0$ (el caso en que $g\leq0$ se sigue facilmente del caso en que $g\geq0$). Ten en cuenta que
        \begin{align*}
            m\leq f(x)\leq M,
        \end{align*}
        donde
        \begin{align*}
            m=\min\{f(x):x\in[a,b]\}
        \end{align*}
        y
        \begin{align*}
            M=\max\{f(x):x\in[a,b]\}.
        \end{align*}
        En consecuencia, por el ejercicio 2 de la tarea 3,
        \begin{align*}
            m\int_{a}^{b}g\leq\int_{a}^{b}fg\leq M\int_{a}^{b}g.
        \end{align*}
        Si $\int_{a}^{b}g=0$, entonces $\int_{a}^{b}fg=0$ y el resultado es claro. Si $\int_{a}^{b}g\neq0$, entonces
        \begin{align*}
            m\leq\frac{1}{\int_{a}^{b}g}\int_{a}^{b}fg\leq M.
        \end{align*}
        Intenta aplicar el teorema del valor intermedio.
    \end{mdframed}

    \begin{proof}
        Supongamos que $g\geq0$, el caso en que $g\leq0$ se sigue
        facilmente del caso en que $g\geq0$. Tengamos en cuenta que
        \begin{align*}
            m=\min\{f(x):x\in[a,b]\}
        \end{align*}
        y
        \begin{align*}
            M=\max\{f(x):x\in[a,b]\}.
        \end{align*}
        En consecuencia por el ejercicio 2 de la tarea 3,
        \begin{align*}
            m\int_{a}^{b}g\leq\int_{a}^{b}fg\leq M\int_{a}^{b}g.
        \end{align*}
        Si $\int_{a}^{b}g=0$, entonces $\int_{a}^{b}fg=0$ y el resultado es claro. Si $\int_{a}^{b}g\neq0$, entonces
        \begin{align*}
            m\leq\frac{1}{\int_{a}^{b}g}\int_{a}^{b}fg\leq M.
        \end{align*}
        Definamos
        \begin{align*}
            A:=\frac{1}{\int_{a}^{b}g}\int_{a}^{b}fg.
        \end{align*}
        Por el paso anterior,
        \begin{align*}
            A\in[m,M].
        \end{align*}
        Dado que $f$ es continua en $[a,b]$, su imagen es el intervalo cerrado $[m,M]$. Por el teorema del
        valor intermedio, existe un punto $c\in[a,b]$ tal que
        \begin{align*}
            f(c)=A
        \end{align*}
        Mutliplicando la igualdad anterior por $\int_{a}^{b}g$, obtenemos
        \begin{align*}
            \int_{a}^{b}fg=f(c)\int_{a}^{b}g.
        \end{align*}
    \end{proof}

    \begin{mdframed}[style=mdbluebox,frametitle={Ejercicio 2}]
        \textbf{(Integral de una función impar)} Prueba que si $f$ es una función impar e integrable en un intervalo $[-a,a]$, entonces
        \begin{align*}
            \int_{-a}^{a}f=0.
        \end{align*}
        \textbf{Sugerencia 1:} Sean $P=\{t_{0},\dots,t_{n}\}$ una partición del intervalo $[0,a]$ y $Q=\{-t_{0},\dots,-t_{n}\}$; así, $Q$ es una partición del intervalo $[-a,0]$.
        Prueba que
        \begin{align*}
            L(f,Q)=-U(f,P)
        \end{align*}
        y
        \begin{align*}
            U(f,Q)=-L(f,P)
        \end{align*}
        y usalo para probar lo afirmado. \textbf{Sugerencia 2:} Trata de usar el ejercicio 4 de la tarea 3.
    \end{mdframed}

    \begin{proof}
        Sea $f$ una función impar e integrable en $[-a,a]$. Entonces
        \[
        f(-x)=-f(x)
        \quad \text{para todo } x\in[-a,a].
        \]

        Sea $P=\{t_0,\dots,t_n\}$ una partición de $[0,a]$, con
        \[
        0=t_0<t_1<\cdots<t_n=a,
        \]
        y sea
        \[
        Q=\{-t_0,\dots,-t_n\}
        \]
        la partición correspondiente de $[-a,0]$, ordenada crecientemente:
        \[
        -a=-t_n<\cdots<-t_1<-t_0=0.
        \]

        Para cada $i=1,\dots,n$, consideremos los subintervalos
        \[
        I_i=[t_{i-1},t_i]
        \quad \text{y} \quad
        J_i=[-t_i,-t_{i-1}].
        \]

        Sea
        \[
        m_i=\inf\{f(x):x\in I_i\},
        \qquad
        M_i=\sup\{f(x):x\in I_i\}.
        \]

        Como $f$ es impar, para $x\in J_i$ se tiene $x=-y$ con $y\in I_i$, y por tanto
        \[
        f(x)=f(-y)=-f(y).
        \]
        De aquí se deduce que
        \[
        \inf_{x\in J_i} f(x)
        =
        -\sup_{y\in I_i} f(y)
        =
        - M_i,
        \]
        y
        \[
        \sup_{x\in J_i} f(x)
        =
        -\inf_{y\in I_i} f(y)
        =
        - m_i.
        \]

        Observando que
        \[
        |J_i|=t_i-t_{i-1}=|I_i|,
        \]
        se obtiene
        \[
        L(f,Q)
        =
        \sum_{i=1}^n (-M_i)(t_i-t_{i-1})
        =
        - U(f,P),
        \]
        y
        \[
        U(f,Q)
        =
        \sum_{i=1}^n (-m_i)(t_i-t_{i-1})
        =
        - L(f,P).
        \]

        Ahora, como
        \[
        \int_{-a}^a f
        =
        \int_{-a}^0 f+\int_0^a f,
        \]
        y por definición de integral como supremo de sumas inferiores e ínfimo de sumas superiores, se tiene

        \[
        \int_{-a}^0 f
        =
        \sup_{Q\in\mathcal{P}} L(f,Q)
        =
        \sup_{P\in\mathcal{P}} (-U(f,P))
        =
        - \inf_{P\in\mathcal{P}} U(f,P)
        =
        -\int_0^a f.
        \]

        Por lo tanto,
        \[
        \int_{-a}^a f
        =
        \int_{-a}^0 f+\int_0^a f
        =
        -\int_0^a f+\int_0^a f
        =
        0.
        \]
    \end{proof}

    \begin{mdframed}[style=mdbluebox,frametitle={Ejercicio 3}]
        Calcula las integrales siguientes:
        \begin{enumerate}
            \item[(a)] $\displaystyle\int\sin(x)\cos(x)dx$.
            \item[(b)] $\displaystyle\int\frac{\ln(x)}{x}dx$.  
        \end{enumerate}
    \end{mdframed}

    \begin{solucion}
        (a) $\displaystyle\int\sin(x)\cos(x)dx$.\\
        Hagamos $u=\sin(x)$ y $du=\cos(x)dx$. Por lo tanto
        \begin{align*}
            \int\sin(x)\cos(x)dx&=\int udu.\\
            &=\frac{1}{2}u^{2}\\
            &=\frac{1}{2}\sin^{2}(x).
        \end{align*}
        (b) $\displaystyle\int\frac{\ln(x)}{x}dx$.\\
        Sea $u=\ln(x)$ y $du=\frac{1}{x}dx$. En consecuencia
        \begin{align*}
            \int\frac{\ln(x)}{x}dx&=\int udu\\
            &=\frac{1}{2}u^{2}\\
            &=\frac{1}{2}\ln^{2}(x)
        \end{align*}
    \end{solucion}

    \begin{mdframed}[style=mdbluebox,frametitle={Ejercicio 4}]
        \begin{enumerate}
            \item[(a)] $\displaystyle\int\frac{1}{\sqrt{4-x^{2}}}dx$.
            \item[(b)] $\displaystyle\int\frac{1}{\sqrt{9+x^{2}}}dx$.
            \item[(c)] $\displaystyle\int\frac{1}{\sqrt{x^{2}-2}}dx$.
        \end{enumerate}
    \end{mdframed}

    \begin{solucion}
        (a) $\displaystyle\int\frac{1}{\sqrt{4-x^{2}}}dx$.\\
        Hagamos $x=2\sin\paren{\theta}$ y $dx=2\cos\paren{\theta}d\theta$.
        Por lo tanto
        \begin{align*}
            \int\frac{1}{\sqrt{4-x^{2}}}dx&=\int\frac{2\cos\paren{\theta}}{\sqrt{4-4\sin^{2}\paren{\theta}}}\\
            &=\int\frac{2\cos\paren{\theta}}{\sqrt{4\cos^{2}\paren{\theta}}}\\
            &=\theta.
        \end{align*}
        Notemos que $\theta=\arcsin\paren{\frac{x}{2}}$. Así,
        \begin{equation*}
            \int\frac{1}{\sqrt{4-x^{2}}}dx=\arcsin\paren{\frac{x}{2}}.
        \end{equation*}
        (b) $\displaystyle\int\frac{1}{\sqrt{9+x^{2}}}dx$.\\
        Sea $x=3\tan\paren{\theta}$ y $dx=3\sec^{2}\paren{\theta}d\theta$. Entonces
        $\tan\paren{\theta}=\frac{x}{3}$ y $\sec\paren{\theta}=\frac{\sqrt{x^{2}+9}}{3}$.
        En consecuencia,
        \begin{align*}
            \int\frac{1}{\sqrt{9+x^{2}}}dx&=\int\frac{3\sec^{2}\paren{\theta}}{\sqrt{9+9\tan^{2}\paren{\theta}}}d\theta\\
            &=\int\frac{3\sec^{2}\paren{\theta}}{\sqrt{9\sec^{2}(\theta)}}d\theta\\
            &=\int\sec\paren{\theta}d\theta\\
            &=\ln\paren{\abs{\sec\paren{\theta}+\tan\paren{\theta}}}.
        \end{align*}
        Así,
        \begin{equation*}
            \int\frac{1}{\sqrt{9+x^{2}}}dx=\ln\paren{\abs{\frac{\sqrt{x^{2}+9}}{3}+\frac{x}{3}}}.
        \end{equation*}
        (c) $\displaystyle\int\frac{1}{\sqrt{x^{2}-2}}dx$.\\
        Sea $x=\sqrt{2}\sec\paren{\theta}$ y $dx=\sqrt{2}\sec\paren{\theta}\tan\paren{\theta}d\theta$. Entonces
        $\sec\paren{\theta}=\frac{x\sqrt{2}}{2}$ y $\tan\paren{\theta}=\frac{\sqrt{2x^{2}-4}}{2}$. En consecuencia,
        \begin{align*}
            \int\frac{1}{\sqrt{x^{2}-2}}dx&=\int\frac{\sqrt{2}\sec\paren{\theta}\tan\paren{\theta}}{\sqrt{2\sec^{2}(\theta)-2}}d\theta\\
            &=\int\frac{\sqrt{2}\sec\paren{\theta}\tan\paren{\theta}}{\sqrt{2\tan^{2}(\theta)}}d\theta\\
            &=\int\sec(\theta)d\theta\\
            &=\ln\paren{\abs{\sec(\theta)+\tan(\theta)}}.
        \end{align*}
        Así,
        \begin{equation*}
            \int\frac{1}{\sqrt{x^{2}-2}}dx=\ln\paren{\abs{\frac{x\sqrt{2}}{2}+\frac{\sqrt{2x^{2}-4}}{2}}}.
        \end{equation*}
    \end{solucion}

    \begin{mdframed}[style=mdbluebox,frametitle={Ejercicio 5}]
        Calcula la integral
        \begin{align*}
            \int\frac{x+11}{x^{2}-5x-14}dx.
        \end{align*}
    \end{mdframed}

    \begin{solucion}
        Notemos que el integrando es una funcion racional propia, así que seguiremos el método
        citado para descomponerlo en fracciones simples. Descomponemos el denominador del integrando
        en factores lineales y cuadraticos irreducibles
        \begin{equation*}
            x^{2}-5x-14=(x-7)(x+2).
        \end{equation*}
        Por el factor $(x-7)$ la descomposición del integrando en fracciones simples incluye
        una fraccion de la forma
        \begin{equation*}
            \frac{A}{x-2}
        \end{equation*}
        y por el factor $(x+2)$ la descomposición del integrando en fracciones simples incluye
        una fracción de la forma
        \begin{equation*}
            \frac{B}{x+2}.
        \end{equation*}
        Así, la descomposición del integrando en fracciones simples es
        \begin{equation}
            \frac{x+11}{x^{2}-5x-14}=\frac{A}{x-7}+\frac{B}{x+2}.
        \end{equation}
        Notemos que $A$ y $B$ cumplen (1) si y solo si $A$ y $B$ cumple que
        \begin{equation}
            x+11=A(x+2)+B(x-7).
        \end{equation}
        Reexpresemos (2) como
        \begin{equation}
            x+11=(A+B)x+2A-7B.
        \end{equation}
        Puesto que dos polinomios son iguales si sus coeficientes son iguales, de (3)
        obtenemos que $A$ y $B$ satisfacen el sistema de ecuaciones
        \begin{align*}
            A+B&=1\\
            2A-7B&=11
        \end{align*}
        cuya solución es $A=2$ y $B=-1$. Por lo cual
        \begin{equation*}
            \frac{x+11}{x^{2}-5x-14}=\frac{2}{x-7}-\frac{1}{x+2}
        \end{equation*}
        y, en consecuencia
        \begin{align*}
            \frac{x+11}{x^{2}-5x-14}&=2\int\frac{1}{x-7}dx-\int\frac{1}{x+2}\\
            &=2\ln\paren{x-7}-\ln\paren{x+2}.
        \end{align*}
    \end{solucion}

    \begin{mdframed}[style=mdbluebox,frametitle={Ejercicio 5}]
        Prueba que las integrales impropias en los incisos (c),(d) y (e) de la definición 10 no dependen del punto $p$ en cuestión.
    \end{mdframed}
    
    \begin{proof}
        Para (c), supongamos que existe un punto $p$ tal que la integral impropia en $(a,b)$ está bien definida. Tomemos ahora un punto $p'$ y supongamos que $p<p'$ (el otro caso es análogo). Notemos que para $a<x<p$ se tiene
        \[
        \int_x^{p'} f
        =
        \int_x^{p} f
        +
        \int_{p}^{p'} f,
        \]
        porque $f$ es integrable en todo subintervalo cerrado de $(a,b)$, en particular en $[x,p']$.

        Luego, como $\lim_{x\to a^+}\int_x^{p} f$ existe (aunque podría no ser finito) y además
        \[
        \left|\int_p^{p'} f\right|<\infty
        \]
        por ser $f$ integrable en $[p,p']$, se tiene que el límite
        \[
        \lim_{x\to a^+}\int_x^{p'} f
        \]
        existe (podría no ser finito), pues
        \[
        \int_x^{p'} f
        =
        \int_x^{p} f+\int_p^{p'} f
        \]
        y el segundo término es constante respecto de $x$. Por lo tanto,
        \[
        \int_a^{p'} f
        =
        \lim_{x\to a^+}\int_x^{p'} f
        =
        \lim_{x\to a^+}\int_x^{p} f
        +
        \int_p^{p'} f
        =
        \int_a^{p} f+\int_p^{p'} f.
        \]

        De igual manera, se tiene que
        \[
        \int_{p'}^{b} f
        =
        \int_{p}^{b} f-\int_{p}^{p'} f.
        \]
        En efecto, para $p'<y<b$,
        \[
        \int_{p'}^{y} f
        =
        \int_{p}^{y} f-\int_{p}^{p'} f,
        \]
        y pasando al límite cuando $y\to b^{-}$ se obtiene la igualdad.

        Estas igualdades garantizan que
        \[
        \int_a^{p'} f+\int_{p'}^{b} f
        \]
        no presenta la forma $\infty-\infty$ o $-\infty+\infty$, ya que
        \[
        \int_a^{p'} f+\int_{p'}^{b} f
        =
        \left(\int_a^{p} f+\int_p^{p'} f\right)
        +
        \left(\int_{p}^{b} f-\int_p^{p'} f\right),
        \]
        y el término $\int_p^{p'} f$ es un número real finito que se cancela, por lo que la suma coincide exactamente con
        \[
        \int_a^{p} f+\int_{p}^{b} f,
        \]
        la cual está bien definida por hipótesis y no es de tipo indeterminado.

        Finalmente,
        \[
        \int_a^{p} f+\int_{p}^{b} f
        =
        \int_a^{p'} f+\int_{p'}^{b} f,
        \]
        con lo que la integral impropia en $(a,b)$ no depende del punto intermedio elegido.
        Para (d), supongamos que existe un punto $p\in(a,\infty)$ tal que la integral impropia en $(a,\infty)$ está bien definida, es decir,
        \[
        \int_a^\infty f
        =
        \int_a^p f+\int_p^\infty f,
        \]
        sin que aparezca una indeterminación del tipo $\infty-\infty$.

        Tomemos ahora un punto $p'\in(a,\infty)$ y supongamos $p<p'$ (el otro caso es análogo).

        Notemos que para $a<x<p$ se tiene
        \[
        \int_x^{p'} f
        =
        \int_x^{p} f
        +
        \int_{p}^{p'} f,
        \]
        porque $f$ es integrable en todo intervalo cerrado y acotado, en particular en $[x,p']$.

        Luego, como $\lim_{x\to a^+}\int_x^{p} f$ existe (aunque podría no ser finito) y además
        \[
        \left|\int_p^{p'} f\right|<\infty
        \]
        por ser $f$ integrable en $[p,p']$, se tiene que el límite
        \[
        \lim_{x\to a^+}\int_x^{p'} f
        \]
        existe (podría no ser finito), pues
        \[
        \int_x^{p'} f
        =
        \int_x^{p} f+\int_p^{p'} f
        \]
        y el segundo término es constante respecto de $x$. Por lo tanto,
        \[
        \int_a^{p'} f
        =
        \lim_{x\to a^+}\int_x^{p'} f
        =
        \lim_{x\to a^+}\int_x^{p} f
        +
        \int_p^{p'} f
        =
        \int_a^{p} f+\int_p^{p'} f.
        \]

        De igual manera, para $p'<y$ se tiene
        \[
        \int_{p'}^{y} f
        =
        \int_{p}^{y} f-\int_{p}^{p'} f,
        \]
        Pasando al límite cuando $y\to\infty$, se obtiene
        \[
        \int_{p'}^{\infty} f
        =
        \int_{p}^{\infty} f-\int_{p}^{p'} f.
        \]

        Estas igualdades garantizan que
        \[
        \int_a^{p'} f+\int_{p'}^{\infty} f
        \]
        no presenta la forma $\infty-\infty$ o $-\infty+\infty$, ya que
        \[
        \int_a^{p'} f+\int_{p'}^{\infty} f
        =
        \left(\int_a^{p} f+\int_p^{p'} f\right)
        +
        \left(\int_{p}^{\infty} f-\int_p^{p'} f\right),
        \]
        y el término $\int_p^{p'} f$ es un número real finito que se cancela, por lo que la suma coincide exactamente con
        \[
        \int_a^{p} f+\int_{p}^{\infty} f,
        \]
        la cual está bien definida por hipótesis y no es de tipo indeterminado.

        Finalmente,
        \[
        \int_a^{p} f+\int_{p}^{\infty} f
        =
        \int_a^{p'} f+\int_{p'}^{\infty} f,
        \]
        con lo que la integral impropia en $(a,\infty)$ no depende del punto intermedio elegido.
        Para (e), supongamos que existe un punto $p\in\mathbb{R}$ tal que la integral impropia en $(-\infty,\infty)$ está bien definida, es decir,
        \[
        \int_{-\infty}^{\infty} f
        =
        \int_{-\infty}^{p} f+\int_p^{\infty} f,
        \]
        sin que aparezca una indeterminación del tipo $\infty-\infty$.

        Tomemos ahora un punto $p'\in\mathbb{R}$ y supongamos $p<p'$ (el otro caso es análogo).

        Notemos que para $x<p$ se tiene
        \[
        \int_x^{p'} f
        =
        \int_x^{p} f
        +
        \int_{p}^{p'} f,
        \]
        porque $f$ es integrable en todo intervalo cerrado y acotado, en particular en $[x,p']$.

        Luego, como $\lim_{x\to -\infty}\int_x^{p} f$ existe (aunque podría no ser finito) y además
        \[
        \left|\int_p^{p'} f\right|<\infty
        \]
        por ser $f$ integrable en $[p,p']$, se tiene que el límite
        \[
        \lim_{x\to -\infty}\int_x^{p'} f
        \]
        existe (podría no ser finito), pues
        \[
        \int_x^{p'} f
        =
        \int_x^{p} f+\int_p^{p'} f
        \]
        y el segundo término es constante respecto de $x$. Por lo tanto,
        \[
        \int_{-\infty}^{p'} f
        =
        \lim_{x\to -\infty}\int_x^{p'} f
        =
        \lim_{x\to -\infty}\int_x^{p} f
        +
        \int_p^{p'} f
        =
        \int_{-\infty}^{p} f+\int_p^{p'} f.
        \]

        De igual manera, para $p'<y$ se tiene
        \[
        \int_{p'}^{y} f
        =
        \int_{p}^{y} f-\int_{p}^{p'} f,
        \]
        por aditividad en el intervalo compacto $[p,y]$. Pasando al límite cuando $y\to\infty$, se obtiene
        \[
        \int_{p'}^{\infty} f
        =
        \int_{p}^{\infty} f-\int_{p}^{p'} f.
        \]

        Estas igualdades garantizan que
        \[
        \int_{-\infty}^{p'} f+\int_{p'}^{\infty} f
        \]
        no presenta la forma $\infty-\infty$ o $-\infty+\infty$, ya que
        \[
        \int_{-\infty}^{p'} f+\int_{p'}^{\infty} f
        =
        \left(\int_{-\infty}^{p} f+\int_p^{p'} f\right)
        +
        \left(\int_{p}^{\infty} f-\int_p^{p'} f\right),
        \]
        y el término $\int_p^{p'} f$ es un número real finito que se cancela, por lo que la suma coincide exactamente con
        \[
        \int_{-\infty}^{p} f+\int_p^{\infty} f,
        \]
        la cual está bien definida por hipótesis y no es de tipo indeterminado.

        Finalmente,
        \[
        \int_{-\infty}^{p} f+\int_p^{\infty} f
        =
        \int_{-\infty}^{p'} f+\int_{p'}^{\infty} f,
        \]
        con lo que la integral impropia en $(-\infty,\infty)$ no depende del punto intermedio elegido.
    \end{proof}
    
\end{document}


