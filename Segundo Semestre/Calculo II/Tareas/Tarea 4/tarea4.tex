\documentclass[12pt,a4paper]{article}
\usepackage{ugmath}
\usepackage{float}
\usepackage{placeins}
\usepackage[labelfont=bf,labelsep=space]{caption}
\newcommand{\paren}[1]{\left( #1 \right)}
\newcommand{\alumno}{Ricardo León Martínez}
\newcommand{\materia}{Calculo Diferencia e Integral II}
\newcommand{\profesor}{Fernando Nuñez Medina}
\newcommand{\tarea}{Tarea 3}
\newcommand{\fecha}{13/2/2026}

\begin{document}
    \begin{center}
    {\large\textbf{UNIVERSIDAD DE GUANAJUATO}}\\[0.3cm]
    {\normalsize\textbf{DIVISIÓN DE CIENCIAS NATURALES Y EXACTAS}}\\
    {\normalsize\textbf{CAMPUS GUANAJUATO}}\\[1cm]

    {\Large\textbf{\tarea\ (\materia)}}\\[1cm]
    \end{center}

    \textbf{Nombre:} \alumno \hfill 
    \textbf{Fecha:} \fecha \hfill 
    \textbf{Calificación:} \rule{3cm}{0.4pt} \\[0.3cm]

    \begin{mdframed}[style=mdbluebox,frametitle={Ejercicio 1}]
        \textbf{(Teorema del valor medio generalizado para integrales)} Sean $f$ y $g$ funciones integrables en un intervalo $[a,b]$. Prueba que si $f$ es continua y $g$ no cambia
        de signo, entonces existe $c\in[a,b]$ tal que
        \begin{align*}
            \int_{a}^{b}fg=f(c)\int_{a}^{b}g.
        \end{align*}
        \textbf{Sugerencia:} Supón primero que $g\geq0$ (el caso en que $g\leq0$ se sigue facilmente del caso en que $g\geq0$). Ten en cuenta que
        \begin{align*}
            m\leq f(x)\leq M,
        \end{align*}
        donde
        \begin{align*}
            m=\min\{f(x):x\in[a,b]\}
        \end{align*}
        y
        \begin{align*}
            M=\max\{f(x):x\in[a,b]\}.
        \end{align*}
        En consecuencia, por el ejercicio 2 de la tarea 3,
        \begin{align*}
            m\int_{a}^{b}g\leq\int_{a}^{b}fg\leq M\int_{a}^{b}g.
        \end{align*}
        Si $\int_{a}^{b}g=0$, entonces $\int_{a}^{b}fg=0$ y el resultado es claro. Si $\int_{a}^{b}g\neq0$, entonces
        \begin{align*}
            m\leq\frac{1}{\int_{a}^{b}g}\int_{a}^{b}fg\leq M.
        \end{align*}
        Intenta aplicar el teorema del valor intermedio.
    \end{mdframed}

    \begin{mdframed}[style=mdbluebox,frametitle={Ejercicio 2}]
        \textbf{(Integral de una función impar)} Prueba que si $f$ es una función impar e integrable en un intervalo $[-a,a]$, entonces
        \begin{align*}
            \int_{-a}^{a}f=0.
        \end{align*}
        \textbf{Sugerencia 1:} Sean $P=\{t_{0},\dots,t_{n}\}$ una partición del intervalo $[0,a]$ y $Q=\{-t_{0},\dots,-t_{n}\}$; así, $Q$ es una partición del intervalo $[-a,0]$.
        Prueba que
        \begin{align*}
            L(f,Q)=-U(f,P)
        \end{align*}
        y
        \begin{align*}
            U(f,Q)=-L(f,P)
        \end{align*}
        y usalo para probar lo afirmado. \textbf{Sugerencia 2:} Trata de usar el ejercicio 4 de la tarea 3.
    \end{mdframed}

    \begin{mdframed}[style=mdbluebox,frametitle={Ejercicio 3}]
        Calcula las integrales siguientes:
        \begin{enumerate}
            \item[(a)] $\displaystyle\int\sin(x)\cos(x)dx$.
            \item[(b)] $\displaystyle\int\frac{\ln(x)}{x}dx$.  
        \end{enumerate}
    \end{mdframed}

    \begin{mdframed}[style=mdbluebox,frametitle={Ejercicio 4}]
        \begin{enumerate}
            \item[(a)] $\displaystyle\int\frac{1}{\sqrt{4-x^{2}}}dx$.
            \item[(b)] $\displaystyle\int\frac{1}{\sqrt{9+x^{2}}}dx$.
            \item[(c)] $\displaystyle\int\frac{1}{\sqrt{x^{2}-2}}dx$.
        \end{enumerate}
    \end{mdframed}

    \begin{mdframed}[style=mdbluebox,frametitle={Ejercicio 4}]
        Calcula la integral
        \begin{align*}
            \int\frac{x+11}{x^{2}-5x-14}dx.
        \end{align*}
    \end{mdframed}

    \begin{mdframed}[style=mdbluebox,frametitle={Ejercicio 4}]
        Prueba que las integrales impropias en los incisos (c),(d) y (e) de la definición 10 no dependen del punto $p$ en cuestión.
    \end{mdframed}
\end{document}