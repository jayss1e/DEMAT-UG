\message{ !name(tarea1.tex)}\documentclass[12pt,a4paper]{article}
\usepackage{ugmath}
\usepackage{float}
\usepackage{placeins}
\usepackage[labelfont=bf,labelsep=space]{caption}
\newcommand{\alumno}{Ricardo León Martínez}
\newcommand{\materia}{Calculo Diferencia e Integral II}
\newcommand{\profesor}{Fernando Nuñez Medina}
\newcommand{\tarea}{Tarea 1}
\newcommand{\fecha}{30/1/2026}

\begin{document}

\message{ !name(tarea1.tex) !offset(-3) }

    \begin{center}
    {\large\textbf{UNIVERSIDAD DE GUANAJUATO}}\\[0.3cm]
    {\normalsize\textbf{DIVISIÓN DE CIENCIAS NATURALES Y EXACTAS}}\\
    {\normalsize\textbf{CAMPUS GUANAJUATO}}\\[1cm]

    {\Large\textbf{\tarea\ (\materia)}}\\[1cm]
    \end{center}

    \textbf{Nombre:} \alumno \hfill 
    \textbf{Fecha:} \fecha \hfill 
    \textbf{Calificación:} \rule{3cm}{0.4pt} \\[0.3cm]

        \begin{mdframed}[style=mdbluebox,frametitle={Ejercicio 1}]
            Prueba la proposición 2.
        \end{mdframed}

        \begin{proof}
            Por definición, $f$ es integrable si
            \begin{align*}
                \sup_{P\in\mathcal{P}}L(f,P)=\inf_{P\in\mathcal{P}}U(f,P).
            \end{align*}
            A este numero se le denota por $\int_{a}^{b}f$. Ahora es inmediato de la definición
            de supremo que para cualquier partición $P$ se cumple
            \begin{align*}
                L(f,P)\leq\sup_{Q\in\mathcal{P}}L(f,Q)=\int_{a}^{b}f.
            \end{align*}
            Analogamente,
            \begin{align*}
                \int_{a}^{b}f=\inf_{Q\in P}U(f,Q)\leq U(f,P).
            \end{align*}
            Combinando ambas desigualdades obtenemos
            \begin{align*}
                L(f,P)\leq\int_{a}^{b}\leq U(f,P)\text{ para toda }P\in\mathcal{P}.
            \end{align*}
            Ahora veamos que es unico, supongamos que existe un número $A$ tal que
            \begin{align*}
                L(f,P)\leq A\leq U(f,P)\text{ para toda }P\in\mathcal{P}.
            \end{align*}
            Entonces $A$ es una cota superior par todas las sumas inferiores, luego
            \begin{align*}
                A\geq\sup_{p\in\mathcal{P}}L(f,P)=\int_{a}^{b}f.
            \end{align*}
            Tambien $A$ es una cota inferior para todas las sumas superiores, luego
            \begin{align*}
                A\leq\inf_{P\in\mathcal{P}}U(f,P)=\int_{a}^{b}f.
            \end{align*}
            Por tanto $A=\int_{a}^{b}f$.
        \end{proof}

        \begin{mdframed}[style=mdbluebox,frametitle={Ejercicio 2}]
            Definamos
            \begin{align*}
                f(x)=
                \begin{cases}
                    0,\text{ si }x\in[0,1),\\
                    1,\text{ si } x=1.
                \end{cases}
            \end{align*}
            \begin{enumerate}
                \item[(a)] Dibuja la gráfica de $f$.
                \item[(b)] Determina si $f$ es integrable en [0,1]. Argumenta tu respuesta.
                \item[(c)] Si $f$ es integrable, calcula $\int_{0}^{1}f$.   
            \end{enumerate}
        \end{mdframed}
        \begin{solucion}
            \begin{enumerate}
                \item[(a)] Dibuja la gráfica de $f$.
                \begin{center}
                    \includegraphics[width=0.40\linewidth]{Figuras.pdf}
                \end{center}

                \item[(b)] Determina si $f$ es integrable en [0,1]. Argumenta tu respuesta.\\
                Es claro que $f$ es monótona creciente en $[0,1]$, pues es constante en $[0,1)$
                y solo presenta un salto hacia arriba en el punto $x=1.$ Por la proposición 5,
                se concluye que $f$ es integrable en $[0,1]$.

                \item[(c)] Para todo subintervalo $[t_{i-1},t_{i}]\subset[0,1]$ se tiene
                \begin{align*}
                    m_{i},
                \end{align*}
                pues en cualquier intervalo hay puntos $x\lt1$ donde $f(x)=0$. Por lo tanto
                \begin{align*}
                    L(f,P)=0,
                \end{align*}
                para toda partición $P$. En consecuencia
                \begin{align*}
                    \int_{a}^{b}f=0.
                \end{align*}
                
            \end{enumerate}
        \end{solucion}

        \begin{mdframed}[style=mdbluebox,frametitle={Ejercicio 3}]
            Dada una función $f$ acotada en un intervalo $[a,b]$ y $P=\{t_{0},t_{1},\dots,t_{n}\}$
            una partición de $[a,b]$, para cada $i=1.\dots,n$, definimos
            \begin{align*}
                m_{i}^{f}=\inf\{f(x):x\in[t_{i-1},t_{i}]\}
            \end{align*}
            y
            \begin{align*}
                M_{i}^{f}=\sup\{f(x):x\in[t_{i-1},t_{i}]\}.
            \end{align*}
            Supongamos que $f$ y $g$ son funciones acotadas en un intervalo $[a,b]$, que $c$ es una constante
            y que $P$ es una partición de $[a,b]$. Prueba lo siguiente:
            \begin{enumerate}
                \item[(a)] $m_{i}^{f}+m_{i}^{g}\leq m_{i}^{f+g}\leq M_{i}^{f+g}\leq M_{i}^{f}+M_{i}^{g}$.
                \item[(b)] Si $c\geq0$, entonces $m_{i}^{cf}=cm_{i}^{f}$ y $M_{i}^{cf}=cM_{i}^{f}$.
                \item[(c)] Si $c\lt0$, entonces $m_{i}^{cf}=cM_{i}^{f}$ y $M_{i}^{cf}=cm_{i}^{f}$. 
            \end{enumerate}
        \end{mdframed}

        \begin{proof}
            \begin{enumerate}
                \item[(a)]
                \(m_i^f + m_i^g \le m_i^{f+g} \le M_i^{f+g} \le M_i^f + M_i^g.\)

                Sea \(I_i=[t_{i-1},t_i]\). Definamos
                \begin{align*}
                    A:=\{f(x):x\in I_i\}, \qquad
                    B:=\{g(x):x\in I_i\},
                \end{align*}
                y
                \begin{align*}
                    C:=\{f(x)+g(x):x\in I_i\}.
                \end{align*}

                Por definición,
                \begin{align*}
                    m_i^f=\inf A, \quad m_i^g=\inf B, \quad m_i^{f+g}=\inf C,
                \end{align*}
                \begin{align*}
                    M_i^f=\sup A, \quad M_i^g=\sup B, \quad M_i^{f+g}=\sup C.
                \end{align*}

                Observemos que, para todo \(x\in I_i\),
                \begin{align*}
                    f(x)\in A \quad\text{y}\quad g(x)\in B,
                \end{align*}
                por lo que
                \begin{align*}
                    f(x)+g(x)\in A+B.
                \end{align*}
                De aquí se sigue que
                \begin{align*}
                    C \subseteq A+B.
                \end{align*}

                Por un resultado conocido se tiene
                \begin{align*}
                    \sup(A+B)=\sup A+\sup B.
                \end{align*}
                Como \(C\subseteq A+B\), se sigue que
                \begin{align*}
                    \sup C \le \sup(A+B),
                \end{align*}
                y por lo tanto
                \begin{align*}
                    M_i^{f+g} \le M_i^f + M_i^g.
                \end{align*}

                Análogamente,
                \begin{align*}
                    \inf(A+B)=\inf A+\inf B.
                \end{align*}
                Como \(C\subseteq A+B\), se cumple
                \begin{align*}
                    \inf(A+B) \le \inf C,
                \end{align*}
                y de aquí se obtiene
                \begin{align*}
                    m_i^f + m_i^g \le m_i^{f+g}.
                \end{align*}

                Finalmente, como por definición siempre \(\inf C \le \sup C\), se concluye que
                \begin{align*}
                    m_i^f + m_i^g
                    \le
                    m_i^{f+g}
                    \le
                    M_i^{f+g}
                    \le
                    M_i^f + M_i^g.
                \end{align*}

                \item[(b)] Si $c\geq0$, entonces $m_{i}^{cf}=cm_{i}^{f}$ y $M_{i}^{cf}=cM_{i}^{f}$\\
                Sea $I_{i}=[t_{i-1},t_{i}]$. Definamos
                \begin{align*}
                    A:=\{f(x):x\in I_{i}\} \qquad cA:=\{cf(x):x\in I_{i}\}.
                \end{align*}
                Por definición,
                \begin{align*}
                    m_{i}^{cf}=\inf(cA), \quad cm_{i}^{f}=c\inf(A)\\
                    M_{i}^{cf}=\sup(cA), \quad cM_{i}^{f}=c\sup(A).
                \end{align*}
                De un resultado conocido sabemos que $\inf(cA)=c\inf(A)$ de esto se tiene directamente
                \begin{align*}
                    m_{i}^{cf}=cm_{i}^{f}.
                \end{align*}
                Finalmente, sabiendo que $\sup(cA)=c\sup(A)$, obetenemos
                \begin{align*}
                    M_{i}^{cf}=cM_{i}^{f}
                \end{align*}
                \item[(c)] Si $x\lt0$, entonces $m_{i}^{cf}=cM_{i}^{f}$ y $M_{i}^{cf}=cm_{i}^{f}$.\\
                Sea $I_{i}$. Definanamos $A=\{f(x):x\in I_{i}\}$ usando las definiciones del inciso
                anterior y de un resultado conocido que dice que si $c\lt0$ entonces
                \begin{align*}
                    \inf(-cA)=-c\sup(A) \qquad \sup(-cA)=-c\inf(A)
                \end{align*}
                se sigue inmdiatamente que
                \begin{align*}
                    m_{i}^{cf}=cM_{i}^{f} \qquad M_{i}^{cf}=cm_{i}^{f}.
                \end{align*}
            \end{enumerate}
        \end{proof}

        \begin{mdframed}[style=mdbluebox,frametitle={Ejercicio 4}]
            Sean $f$ y $g$ funciones acotadas en un intervalo $[a,b]$ y $P$ una partición de
            $[a,b]$. Prueba que
            \begin{align*}
                L(f, P)+L(g, P)\leq L(f+g, P)\leq U(f+g, P)\leq U(f, P)+U(g, P).
            \end{align*}
            \textbf{Sugerencia:} Considera el inciso (a) del ejercicio 8 de las notas
            (ejercicio 3 de esta tarea).
        \end{mdframed}

        \begin{proof}
            Del inciso (a) del ejercicio 3 de esta tarea tenemos que 
            \begin{align*}
                m_i^f + m_i^g \leq m_i^{f+g} \leq M_i^{f+g} \leq M_i^f + M_i^g.
            \end{align*}
            Multiplicando por $(t_{i-1}-t_{i})$ obtenemos
            \begin{align*}
                m_i^f(t_{i-1}-t_{i}) + m_i^g(t_{i-1}-t_{i}) \leq m_i^{f+g}(t_{i-1}-t_{i}) \leq M_i^{f+g}(t_{i-1}-t_{i}) \leq M_i^f(t_{i-1}-t_{i}) + M_i^g(t_{i-1}-t_{i}).
            \end{align*}
            Se sigue que
            \begin{align*}
                \sum_{i=1}^{n} m_i^f(t_{i-1}-t_{i}) + \sum_{i=1}^{n} m_i^g(t_{i-1}-t_{i}) \leq \sum_{i=1}^{n} m_i^{f+g}(t_{i-1}-t_{i}) \leq \sum_{i=1}^{n} M_i^{f+g}(t_{i-1}-t_{i}) &\leq \\
                \sum_{i=1}^{n} M_i^f(t_{i-1}-t_{i}) + \sum_{i=1}^{n} M_i^g(t_{i-1}-t_{i}).
            \end{align*}
            esto por definición de suma superior e inferior es
            \begin{align*}
                L(f, P)+L(g, P)\leq L(f+g, P)\leq U(f+g, P)\leq U(f, P)+U(g, P).
            \end{align*}
        \end{proof}

        \begin{mdframed}[style=mdbluebox,frametitle={Ejercicio 5}]
            Sean $f$ una función acotada en un intervalo $[a,b]$, $c$ un número real y $P$ una
            partición de $[a.b]$. Prueba lo siguiente:
            \begin{enumerate}
                \item[(a)] Si $c\geq0$, entonces
                \begin{align*}
                    cL(f, P)=L(cf, P)\leq U(cf, P)=cU(f, P).
                \end{align*}
                \item[(b)] Si $c\lt0$, entonces
                \begin{align*}
                    cU(f, P)=L(cf, P)\leq U(cf, P)=cL(f, P).
                \end{align*} 
            \end{enumerate}
            \textbf{Sugerencia:} Considera los incisos (b) y (c) del ejercicio 8 de las notas
            (ejercicio 3 de esta tarea).
        \end{mdframed}

        \begin{proof}
            \begin{enumerate}
                \item[(a)]
                Del inciso (b) del ejercicio 3 de esta tarea tenemos que si $c\geq0$, entonces
                \begin{align*}
                    m_{i}^{cf}=cm_{i}^{f}, \quad \text{ y } \quad M_{i}^{cf}=cM_{i}^{f}.
                \end{align*}
                tomando $m_{i}^{cf}=cm_{i}^{f}$ y multiplicandolo por $(t_{i-1}-t_{i})$ obtenemos
                \begin{align*}
                    m_{i}^{cf}(t_{i-1}-t_{i})=cm_{i}^{f}(t_{i-1}-t_{i}).
                \end{align*}
                Se sigue que
                \begin{align*}
                    \sum_{i=1}^{n}m_{i}^{cf}(t_{i-1}-t_{i})=c\sum_{i=1}^{n}m_{i}^{f}(t_{i-1}-t_{i}).
                \end{align*}
                esto por definición es
                \begin{align*}
                    L(f,p)=cL(f,P).
                \end{align*}
                Ahora tomando $M_{i}^{cf}=cM_{i}^{f}$ y multiplicandolo por $(t_{i-1}-t_{i})$ tenemos
                \begin{align*}
                    M_{i}^{cf}(t_{i-1}-t_{i})=cM_{i}^{f}(t_{i-1}-t_{i}).
                \end{align*}
                Se sigue que
                \begin{align*}
                    \sum_{i=1}^{n}M_{i}^{cf}(t_{i-1}-t_{i})=c\sum_{i=1}^{n}M_{i}^{f}(t_{i-1}-t_{i}).
                \end{align*}
                Nuevamente por definición se tiene
                \begin{align*}
                    U(f,p)=cU(f,p).
                \end{align*}
                Ahora del inciso (b) proposición 1 y juntando las igualdades tenemos
                \begin{align*}
                    cL(f, P)=L(cf, P)\leq U(cf, P)=cU(f, P).
                \end{align*}
                \item[(b)]
                 Del inciso (c) del ejercicio 3 de esta tarea tenemos que si $c\lt0$, entonces
                 \begin{align*}
                    m_{i}^{cf}=cM_{i}^{f} \quad \text{ y } \quad M_{i}^{cf}=cm_{i}^{f}.
                 \end{align*}
                 Tomando $m_{i}^{cf}=cM_{i}^{f}$ y multiplicandolo por $(t_{i-1}-t_{i})$ obtenemos
                 \begin{align*}
                    m_{i}^{cf}(t_{i-1}-t_{i})=cM_{i}^{f}(t_{i-1}-t_{i}),
                 \end{align*}
                 se sigue que
                 \begin{align*}
                    \sum_{i=1}^{n}m_{i}^{cf}(t_{i-1}-t_{i})=c\sum_{i=1}^{n}M_{i}^{f}(t_{i-1}-t_{i})
                 \end{align*}
                 esto por definición es
                 \begin{align*}
                    L(cf,P)=cU(f,P).
                 \end{align*}
                 La prueba para $M_{i}^{cf}=cM_{i}^{f}$ es analoga. De estas dos igualdades y por la proposición 1 tenemos
                 \begin{align*}
                    cU(f, P)=L(cf, P)\leq U(cf, P)=cL(f, P).
                 \end{align*}
            \end{enumerate}
        \end{proof}

        \begin{mdframed}[style=mdbluebox,frametitle={Ejercicio 6}]
            Sea $f$ una función integrable en un intervalo $[a,b]$ y $c$ un número real. Prueba
            que $cf$ es integrable $[a,b]$ y que
            \begin{align*}
                \int_{a}^{b}cf=c\int_{a}^{b}f.
            \end{align*}
            \textbf{Sugerencia:} Trata de imitar la prueba del inciso (a) de la proposición 6 y
            considera el ejercicio 11 de las notas (ejercicio 5 de esta tarea).
        \end{mdframed}
        \begin{proof}
            Para $c\geq0$. Dado que f es integrable tenemos
            \begin{align*}
                \sup_{P\in\mathcal{P}}L(f,P)=\inf_{P\in\mathcal{P}}U(f,P)
            \end{align*}
            al multiplicarlo por $c\geq0$ a ambos lados obtenemos
            \begin{align*}
                c\sup_{P\in\mathcal{P}}L(f,P)=c\inf_{P\in\mathcal{P}}U(f,P)
            \end{align*}
            se sigue que
            \begin{align*}
                \sup_{P\in\mathcal{P}}cL(f,P)=\inf_{P\in\mathcal{P}}cU(f,P)
            \end{align*}
            finalmente por la el ejercicio 5 inciso (a) tenemos que
            \begin{align*}
                \sup_{P\in\mathcal{P}}L(cf,P)=\inf_{P\in\mathcal{P}}U(cf,P)
            \end{align*}
            Por lo tanto
            \begin{align*}
                \int_{a}^{b}cf=c\int_{a}^{b}f.
            \end{align*}
            Para el caso donde $c\lt0$. Como $f$ es integrable tenemos
            \begin{align*}
                \sup_{P\in\mathcal{P}}L(f,P)=\inf_{P\in\mathcal{P}}U(f,P)
            \end{align*}
            multiplicando por $c\lt0$ obtenemos
            \begin{align*}
                -c\sup_{P\in\mathcal{P}}L(f,P)=-c\inf_{P\in\mathcal{P}}U(f,P)
            \end{align*}
            Por propiedades del infimo y supremo se sigue que
            \begin{align*}
                \inf_{P\in\mathcal{P}}-cL(f,P)=\sup_{P\in\mathcal{P}}-cU(f,P)
            \end{align*}
            Por el inciso (b) del ejercicio 5 de esta tarea tenemos
            \begin{align*}
                \inf_{P\in\mathcal{P}}U(-cf,P)=\sup_{P\in\mathcal{P}}L(-cf,P)
            \end{align*}
            Asi,
            \begin{align*}
                \int_{a}^{b}-cf=-c\int_{a}^{b}.
            \end{align*}
            Quedando asi demostrado para todos los casos.
        \end{proof}

\end{document}
\message{ !name(tarea1.tex) !offset(-391) }
