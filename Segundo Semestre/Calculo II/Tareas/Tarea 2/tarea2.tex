\documentclass[12pt,a4paper]{article}
\usepackage{ugmath}
\usepackage{float}
\usepackage{placeins}
\usepackage[labelfont=bf,labelsep=space]{caption}
\newcommand{\alumno}{Ricardo León Martínez}
\newcommand{\materia}{Calculo Diferencia e Integral II}
\newcommand{\profesor}{Fernando Nuñez Medina}
\newcommand{\tarea}{Tarea 2}
\newcommand{\fecha}{6/2/2026}

\begin{document}
    \begin{center}
    {\large\textbf{UNIVERSIDAD DE GUANAJUATO}}\\[0.3cm]
    {\normalsize\textbf{DIVISIÓN DE CIENCIAS NATURALES Y EXACTAS}}\\
    {\normalsize\textbf{CAMPUS GUANAJUATO}}\\[1cm]

    {\Large\textbf{\tarea\ (\materia)}}\\[1cm]
    \end{center}

    \textbf{Nombre:} \alumno \hfill 
    \textbf{Fecha:} \fecha \hfill 
    \textbf{Calificación:} \rule{3cm}{0.4pt} \\[0.3cm]

    \begin{mdframed}[style=mdbluebox,frametitle={Ejercicio 1}]
        Definamos
        \begin{align*}
            f(x)=
            \begin{cases}
                1,\text{ si }x=\frac{1}{2},\\
                0,\text{ si }x\in[0,1]\setminus\{\frac{1}{2}\},
            \end{cases}
        \end{align*}
        \begin{enumerate}
            \item[(a)] Dibuja la gráfica de $f$.
            \item[(b)] Determina si $f$ es integrable en $[0,1]$. Argumenta tu
            respuesta.
            \item[(c)] Si $f$ es integrable, calcula $\int_{0}^{1}f$.  
        \end{enumerate}
    \end{mdframed}
    \begin{solucion}
        \begin{enumerate}
            \item[(a)] Dibuja la gráfica de $f$.
            \begin{center}
                \includegraphics[width=0.40\linewidth]{Figura.pdf}
            \end{center}
            \item[(b)]  Determina si $f$ es integrable en $[0,1]$. Argumenta tu
            respuesta.\\
            Sea $P$ una partición arbitraria de $[0,1]$. Para todo $i=1,\dots,n$ se tiene $m_{i}=0$,
            ya que en cada subintervalo $[x_{i-1},x_{i}]$ existen puntos distintos de $\frac{1}{2}$
            donde $f(x)=0$. En consecuencia,
            \begin{align*}
                L(f,P)=0,
            \end{align*}
            y como esto vale para toda partición $P$,
            \begin{align*}
                \sup_{P\in\mathcal{P}}L(f,P)=0.
            \end{align*}
            Para la misma particion $P$, existe un único índice $j$ tal que $\frac{1}{2}\in[x_{j-1},x_{j}]$.
            En este caso se tiene $M_{j}=1$, mientras que $M_{i}=0$ para $i\neq j$. Por lo tanto,
            \begin{align*}
                U(f,P)=x_{j}-x_{j-1}.
            \end{align*}
            Sea $\varepsilon\gt0$. Podemos elegir una partición $P$ tal que
            \begin{align*}
                x_{j}-x_{j-1}\lt\varepsilon,
            \end{align*}
            y entonces
            \begin{align*}
                U(f,P)\lt\varepsilon.
            \end{align*}
            Dado que siempre $L(f,P)=0$, se obtiene
            \begin{align*}
                0\leq U(f,P)-L(f,P)\lt\varepsilon.
            \end{align*}
            Se sigue que
            \begin{align*}
                \inf_{P\in\mathcal{P}}U(f,P)=0=\sup_{P\in\mathcal{P}}L(f,P),
            \end{align*}
            y por lo tanto $f$ es integrable en $[0,1]$.
            \item[(c)] Si $f$ es integrable, calcula $\int_{a}^{b}f$.\\
            Del inciso anteior es inmediato que
            \begin{align*}
                \int_{0}^{1}f=0.
            \end{align*}
        \end{enumerate}
    \end{solucion}
    
    \begin{mdframed}[style=mdbluebox,frametitle={Ejercicio 2}]
        Sea $f$ una función acotada en un intervalo $[a,b]$. Prueba que si $f$ es continua en $[a,b]$,
        excepto en un punto $p\in[a,b]$, entonces $f$ es integrable en $[a,b]$.\\
        \textbf{Sugerencia:}
        Supón que $p\in(a,b)$, los casos en que $p=a$ y $p=b$ se prueban de manera similar. Como $f$
        es acotada, existe $M\gt0$ tal que $\abs{f(x)}\leq M$ para toda $x\in[a,b]$. Dado $\epsilon\gt0$
        dos puntos $c$ y $d$ tales que $a\lt c\lt p\lt d\lt b$ y $(d-c)2M\lt \epsilon/3$, es decir,
        el area del rectangulo $[c,d]\times[-M,M]$ es menor que $\epsilon/3$. Como $f$ es integrable
        en $[a,c]$ y en $[d,b]$, existen particiones $P$ y $Q$ de $[a,c]$ y $[d,b]$, respectivamente,
        tales que
        \begin{align*}
            U(f,P)-L(f,P)\lt\frac{\epsilon}{3}
        \end{align*}
        y
        \begin{align*}
            U(f,Q)-L(f,Q)\lt\frac{\epsilon}{3}.
        \end{align*}
        Aplica el criterio de integrabilidad con la particion $P\cup Q$ de $[a,b]$.
    \end{mdframed}

    \begin{proof}
        Supongamos que $p\in(a,b)$, los casos $p=a$ y $p=b$ se prueban de manera análoga. Como $f$ es acotada en $[a,b]$, existe $M\gt0$ tal que
        \begin{align*}
            \abs{f(x)}\leq M\text{ para toda }x\in[a,b].
        \end{align*}
        Sea $\varepsilon\gt0$. Elegimos $c,d$ tales que
        \begin{align*}
            a\lt c\lt p\lt d\lt b \text{ y }(d-c)2M\lt\frac{\varepsilon}{3}.
        \end{align*}
        La funcion $f$ es continua en los intervalos cerrados $[a,c]$ y $[d,b]$. Por la proposición 4,
        $f$ es integrable en $[a,c]$ y en $[d,b]$. Por la proposición 3, existen particiones $P$ de
        $[a,c]$ y $Q$ de $[d,b]$ tales que
        \begin{align*}
            U(f,P)-L(f,P)\lt\frac{\varepsilon}{3},\\
            U(f,Q)-L(f,Q)\lt\frac{\varepsilon}{3}.
        \end{align*}
        En el intervalo $[c,d]$, por la acotación de $f$,
        \begin{align*}
            \sup_{x\in[c,d]}f(x)\leq M, \qquad \inf_{x\in[c,d]}f(x)\geq-M.
        \end{align*}
        Luego,
        \begin{align*}
            U(f,[c,d])-L(f,[c,d])\leq(M-(-M))(d-c)=2M(d-c)\lt\frac{\varepsilon}{3}.
        \end{align*}
        Sea
        \begin{align*}
            R:=P\cup Q\cup\{c,d\},
        \end{align*}
        que es una partición de $[a,b]$. Por la definición de suma superior e inferior,
        \begin{align*}
            U(f,R)-L(f,R)&=(U(f,P)-L(f,P))+(U(f,[c,d])-L(f,[c,d]))\\
            &+ (U(f,Q)-L(f,Q))\\
            &\lt\frac{\varepsilon}{3}+\frac{\varepsilon}{3}+\frac{\varepsilon}{3}=\varepsilon.
        \end{align*}
        Como para todo $\varepsilon\gt0$ existe una partición $R$ de $[a,b]$ tal que
        \begin{align*}
            U(f,R)-L(f,R)\lt\varepsilon,
        \end{align*}
        por la proposición 3 concluimos que $f$ es integrable en $[a,b]$.
    \end{proof}

    \begin{mdframed}[style=mdbluebox,frametitle={Ejercicio 3}]
        Muestra funciones $f$ y $g$ integrables en $[a,b]$ tales que
        \begin{align*}
            \int_{a}^{b}fg\neq\left(\int_{a}^{b}f\right)\left(\int_{a}^{b}g\right).
        \end{align*}
    \end{mdframed}
    
    \begin{solucion}
        Sean $m,n\in\mathbb{R}$ y sea $f(x)=m$ y $g(x)=n$. Asi,
        \begin{align*}
            \int_{a}^{b}fg&=\int_{a}^{b}mn\\
            &=mn(b-a),
        \end{align*}
        y tambien
        \begin{align*}
            \left(\int_{a}^{b}f\right)\left(\int_{a}^{b}g\right)&=\left(\int_{a}^{b}m\right)\left(\int_{a}^{b}n\right)\\
            &=m(b-a)n(b-a)\\
            &=mn(b-a)^{2}.
        \end{align*}
        De esta forma es claro que
        \begin{align*}
            \int_{a}^{b}fg\neq\left(\int_{a}^{b}f\right)\left(\int_{a}^{b}g\right).
        \end{align*}
    \end{solucion}

    \begin{mdframed}[style=mdbluebox,frametitle={Ejercicio 4}]
        Sea $f$ una función integrable en un intervalo $[a,b]$ tal que para toda $x\in[a,b]$,
        \begin{align*}
            m\leq f(x)\leq M.
        \end{align*}
        Prueba que
        \begin{align*}
            m(b-a)\leq\int_{a}^{b}f\leq M(b-a).
        \end{align*}
    \end{mdframed}

    \begin{proof}
        Por hipotesis sabemos que
        \begin{align*}
            m\leq f(x)\leq M.
        \end{align*}
        Luego, por la monotonia de la integral se sigue que
        \begin{align*}
            \int_{a}^{b}m\leq\int_{a}^{b}f(x)\leq\int_{a}^{b}M.
        \end{align*}
        Asi, por la integral de una funcion constante obtenemos
        \begin{align*}
            m(b-a)\leq\int_{a}^{b}f\leq M(b-a).
         \end{align*}
    \end{proof}

    \begin{mdframed}[style=mdbluebox,frametitle={Ejercicio 5}]
        \textbf{(Una función que es integrable en un intervalo $[a,b]$ y que no tiene antiderivada en $[a,b]$)}
        Considera la función
        \begin{align*}
            f(x)=
            \begin{cases}
                0,\text{ si }0\leq x\leq1,\\
                1,\text{ si }1\leq x\leq2.
            \end{cases}
        \end{align*}
        Muestra que $f$ es integrable en $[0,2]$ y que no tiene una antiderivada en $[0,2]$.\\
        \textbf{Sugerencia:} Para mostrar que $f$ no tiene una antiderivada ten presente el teorema
        del valor intermedio para derivadas. Una vez que hayas realizado correctamente este ejercicio
        podrás probar sin dificultad que toda función con un doscontinuidad de salto no posee una
        antiderivada.
    \end{mdframed}

    \begin{proof}
        La función $f$ es acotada y es continua en $[0,2]$ excepto en el punto $x=1$, donde presenta
        una discontinuidad de salto. Por la proposicion 4 $f$ es integrable en $[0,2]$.
        Supongamos, por contradicción, que existe una función
        \begin{align*}
            F:[0,2]\to\mathbb{R}
        \end{align*}
        tal que
        \begin{align*}
            F^{\prime}(x)=f(x)\text{ para todo }x\in[0,2].
        \end{align*}
        Entonces para todo $x\in(0,1)$, se tiene $F^{\prime}(x)=0$ y para todo $x\in(1,2)$, se tiene
        $F^{\prime}(X)=1$. Observamos que $F^{\prime}$ toma los valores de 0 y 1 en el intervalo (0,2).
        Por el teorema del valor intermedio para derivadas, si una derivada toma dos valores distintos
        en un intervalo, entonces debe tomar todos los valores intermedios. En particular, debería
        existir algún $c\in(0,2)$ tal que
        \begin{align*}
            F^{\prime}(c)=\frac{1}{2}.
        \end{align*}
        Sin embargo, esto es imposible, pues por definición
        \begin{align*}
            F^{\prime}(x)=f(x)\in\{0,1\}\text{ para todo }x\in(0,2).
        \end{align*}
        Asi, $f$ es integrable en $[0,2]$, pero no tiene antiderivada en $[0,2]$.
    \end{proof}


    \begin{mdframed}[style=mdbluebox,frametitle={Ejercicio 6}]
        Calcula la derivada de la función
        \begin{align*}
            g(x)=\int_{0}^{x^{2}}\sqrt{1+t^{2}+t^{4}}dt
        \end{align*}
    \end{mdframed}

    \begin{proof}
        Sea
        \begin{align*}
            g(x)=\int_{0}^{x^{2}}\sqrt{1+t^{2}+t^{4}}dt.
        \end{align*}
        Definimos
        \begin{align*}
            f(t)=\sqrt{1+t^{2}+t^{4}}.
        \end{align*}
        La funcion $1+t^{2}+t^{4}$ es un polinomio, por lo tanto es continua en $\mathbb{R}$.
        La función raiz cuadrada es continua en $(0,\infty)$. En consecuencia, $f$ es continua
        en $\mathbb{R}$. Por la proposicion 4, $f$ es integrable en todo intervalo cerrado.
        Definimos la función
        \begin{align*}
            F(u)=\int_{0}^{u}f(t)dt=\int_{0}^{u}\sqrt{1+t^{2}+t^{4}}dt.
        \end{align*}
        Como $f$ es integrable en cualquier intervalo y continua en todo punto $u$, por el
        Teorema Fundamental del Cálculo I, se cumple que $F$ es derivable y
        \begin{align*}
            F^{\prime}(u)=f(u)=\sqrt{1+u^{2}+u^{4}}\text{ para todo }u\in\mathbb{R}.
        \end{align*}
        Observamos que
        \begin{align*}
            g(x)=F(x^{2}).
        \end{align*}
        Por la regla de la cadena,
        \begin{align*}
            g^{\prime}(x)=F^{\prime}(x^{2})\cdot(x^{2})^{\prime}.
        \end{align*}
        Sustituyendo la expresion de $F^{\prime}$,
        \begin{align*}
            g^{\prime}(x)=\sqrt{1+(x^{2})^{2}+(x^{2})^{4}}\cdot2x=2x\sqrt{1+x^{4}+x^{8}}.
        \end{align*}
        Concluyendo que
        \begin{align*}
            g^{\prime}(x)=2x\sqrt{1+x^{4}+x^{8}}.
        \end{align*}
    \end{proof}

    \begin{mdframed}[style=mdbluebox,frametitle={Ejercicio 7}]
        Calcula las integrales siguientes:
        \begin{multicols}{2}
            \begin{enumerate}
                \item[(a)] $\int\left(3x^{2}-2x+6\right)dx.$
                \item[(b)]  $\int\left(5x+4\cos\left(x\right)\right)dx.$
                \item[(c)]  $\int xe^{x}dx.$
                \item[(d)]  $\int x^{2}e^{x}dx.$
                \item[(e)]  $\int x\left(x^{2}-7\right)dx.$
                \item[(f)]  $\int5x\left(5x^{2}-4\right)dx.$
            \end{enumerate}
        \end{multicols}
    \end{mdframed}
    \begin{solucion}
        \begin{enumerate}
            \item[(a)] $\int(3x^{2}-2x+6)dx$.\\
            Puesto que
            \begin{align*}
                \int x^{2}dx=\frac{x^{3}}{3},\int xdx=\frac{x^{2}}{2}\text{ y }\int dx
            \end{align*}
            entonces, por la linealidad de la integral indefinida obtenemos que
            \begin{align*}
                \int(3x^{2}-2x+6)dx&=3\int x^{2}dx-2\int xdx+6\int dx\\
                &=x^{3}-x^{2}+6x.
            \end{align*}
            \item[(b)] $\int(5x+\cos(x))dx$.\\
            Puesto que
            \begin{align*}
                \int x dx=\frac{x^{2}}{2}\text{ y }\int\cos(x)dx=\sin(x)
            \end{align*}
            entonces por la linealidad de la integral indefinida obtenemos que
            \begin{align*}
                \int(5x+4\cos(x))dx&=5\int xdx+4\int\cos(x)dx\\
                &=\frac{5}{2}x^{2}+4\sin(x).
            \end{align*}
            \item[(c)] $\int xe^{x}dx$.\\
            Sea $u=x$ y $dv=e^{x}dx$. Entonces $du=2xdx$ y $v=e^{x}$. En consecuencia
            \begin{align*}
                \int xe^{x}dx&=\int udv\\
                &=uv-\int vdu\\
                &=xe^{x}-\int e^{x}dx\\
                &=xe^{x}-e^{x}.
            \end{align*}
            \item[(d)] $\int x^{2}e^{x}dx$.\\
            Sean $u=x^{2}$ y $dv=e^{x}dx$. Entonces $du=2xdx$ y $v=e^{x}$. En consecuencia
            \begin{align*}
                \int x^{2}e^{x}&=\int udv\\
                &=uv-\int vdu\\
                &=x^{2}e^{x}-\int e^{x}2xdx.
            \end{align*}
            Integrando nuevamente por partes y del inciso anterior, obtenemos que
            \begin{align*}
                \int e^{x}2xdx&=2\int xe^{x}dx\\
                &=2xe^{x}-2e^{x}.
            \end{align*}
            Así,
            \begin{align*}
                \int x^{2}e^{x}dx=x^{2}e^{x}-2xe^{x}+2e^{x}.
            \end{align*}
            \item[(e)] $\int x(x^{2}-7)dx$.\\
            Hagamos $u=x^{2}-7$ y $du=2xdx$. Así
            \begin{align*}
                \int x(x^{2}-7)dx&=\frac{1}{2}\int udu\\
                &=\frac{1}{2}\frac{u^{2}}{2}\\
                &=\frac{(x^{2}-7)^{2}}{4}.
            \end{align*}
            \item[(f)] $\int 5x(5x^{2}-4)$.\\
            Hagamos $u=5x^{2}-4$ y $du=10xdx$. Así,
            \begin{align*}
                \int 5x(5x^{2}-4)&=\frac{1}{2}\int udu\\
                &=\frac{1}{2}\frac{u^{2}}{2}\\
                &=\frac{(5x^{2}-4)^{2}}{4}.
            \end{align*}
        \end{enumerate}
    \end{solucion}
\end{document}