\documentclass[12pt,a4paper]{article}
\usepackage{ugmath}
\usepackage{float}
\usepackage{placeins}
\usepackage{array}
\usepackage[labelfont=bf,labelsep=space]{caption}
\newcommand{\paren}[1]{\left( #1 \right)}
\newcommand{\alumno}{Ricardo León Martínez}
\newcommand{\materia}{Algebra Lineal I}
\newcommand{\profesor}{Claudia Reynoso Alcántara}
\newcommand{\tarea}{Tarea 3}
\newcommand{\fecha}{13/2/2026}

\begin{document}

    \begin{center}
        {\large\textbf{UNIVERSIDAD DE GUANAJUATO}}\\[0.3cm]
        {\normalsize\textbf{DIVISIÓN DE CIENCIAS NATURALES Y EXACTAS}}\\
        {\normalsize\textbf{CAMPUS GUANAJUATO}}\\[1cm]

        {\Large\textbf{\tarea\ (\materia)}}\\[1cm]
    \end{center}

    \noindent
    \textbf{Nombre:} \alumno 
    \hfill 
    \textbf{Fecha:} \fecha 
    \hfill 
    \textbf{Calificación:} \rule{3cm}{0.4pt}

    \vspace{0.3cm}


    En todos los problemas $F$ es un campo que contiene a $\mathbb{Q}$.

    \begin{mdframed}[style=mdbluebox,frametitle={Ejercicio 1}]
        Enumera todas las posibles matrices reducidas por filas y escalonadas de tamaño $3\times 4$.
        Escribe 0 y 1 donde corresponda; y usa * para lugares donde pueda ir cualquier elemento de $F$.
    \end{mdframed}

    \begin{mdframed}[style=mdbluebox,frametitle={Ejercicio 2}]
        Responde si es falso (F) o verdadero (V) y justifica tu respuesta:
        \begin{enumerate}
            \item[(a)] Si un sistema de ecuaciones lineales $A\textbf{x}=\textbf{B}$ tiene más variables que
            ecuaciones, entonces el sistema tiene infinitas soluciones.
            \item[(b)] Un sistema consistente tiene una solución única si y solo si no existen variables libres.
            \item[(c)] Si el sistema homogéneo $A\textbf{x}=\textbf{0}$ tiene al menos una variable libre, entonces
            el sistema $A\textbf{x}=\textbf{B}$ tiene infinitas soluciones para cualquier matriz $\textbf{B}$.
            \item[(d)] Si el sistema de ecuaciones lineales $A\textbf{x}=\textbf{B}$ tiene al menos una variable libre, entonces
            el sistema es consistente. 
        \end{enumerate}
    \end{mdframed}

    \begin{mdframed}[style=mdbluebox,frametitle={Ejercicio 3}]
        En este problema se trabajará sobre $\mathbb{R}$. Encuentra la matriz reducida por filas y escalonada
        equivalente a:
        \begin{align*}
            \left(
                \begin{array}{c}
                    1\\
                    2\\
                    7\\
                    0
                \end{array}
            \right),
            \qquad
            \left(
                \begin{array}{cc}
                    1 & -2\\
                    0 & 0\\
                    11 & 1+\sqrt{2}
                \end{array}
            \right),
            \qquad
            \left(
                \begin{array}{ccc}
                    1 & -1 & 2\\
                    2 & 0 & 3
                \end{array}
            \right),
            \qquad
            \left(
                \begin{array}{ccc}
                    1 & -1 & 2\\
                    2 & 0 & 3\\
                    0 & -3 & -1\\
                    1 & 0 & 0
                \end{array}
            \right),
        \end{align*}
        encuentra las soluciones del sistema homogéneo correspondiente. Finalmente, determina cuáles son las variables independientes
        y dependientes en cada caso.
    \end{mdframed}

    \begin{solucion}
        Sea
        \begin{align*}
            A_{1}=
            \begin{pmatrix}
                1\\
                2\\
                7\\
                0
            \end{pmatrix}
        \end{align*}
        La matriz reducida por filas y escalonada es
        \begin{align*}
            R_{1}=
            \begin{pmatrix}
                1\\
                0\\
                0\\
                0
            \end{pmatrix}
        \end{align*}
        Por tanto, el sistema homogéneo asociado es $R_{1}\textbf{x}_{1}=0$. Es decir, explicitamente
        \begin{align*}
            \begin{cases}
                x_{1}=0.
            \end{cases}
        \end{align*}
        La primera columna contiene un pivote. Por tanto, $x_{1}$ es variable dependiente. No hay variables libres.
        El conjunto solución es
        \begin{align*}
            S_{1}=\left\{0\right\}.
        \end{align*}
        Sea
        \begin{align*}
            A_{2}=
            \begin{pmatrix}
                1 & -2\\
                0 & 0\\
                11 & 1+\sqrt{2}
            \end{pmatrix}
        \end{align*}
        La matriz reducida por filas y escalonada es
        \begin{align*}
            R_{2}=
            \begin{pmatrix}
                1 & -2\\
                0 & 1\\
                0 & 0
            \end{pmatrix}
        \end{align*}
        Por tanto, el sistema homogéneo asociado es $R_{2}\textbf{x}_{2}=0$. Es decir, explicitamente
        \begin{align*}
            \begin{cases}
                x_{1}+x_{2}=0\\
                x_{2}=0
            \end{cases}
        \end{align*}
        En la primera y en la segunda columna contienen un pivote. Por lo tanto, $x_{1}$ y $x_{2}$ son
        variables dependientes. No hay variables libres. El conjunto solucion es
        \begin{align*}
            S_{2}=\left\{(0,0)\right\}
        \end{align*}
        Sea
        \begin{align*}
            A_{3}=
            \begin{pmatrix}
                1 & -1 & 2\\
                2 & 0 & 3
            \end{pmatrix}
        \end{align*}
        La matriz reducida por fila y escalonada es
        \begin{align*}
            R_{3}=
            \begin{pmatrix}
                1 & 0 & \frac{3}{2}\\
                0 & 1 & -\frac{1}{2}
            \end{pmatrix}
        \end{align*}
        Por tanto, el sistema homogéneo asociado es $R_{3}\textbf{x}_{3}=0$. Es decir explicitamente
        \begin{align*}
            \begin{cases}
                x_{1}+\frac{3}{2}x_{3}=0\\
                x_{2}-\frac{1}{2}x_{3}=0
            \end{cases}
        \end{align*}
        En la primera y en la segunda columna contiene un pivote. Por lo tanto $x_{1}$ y $x_{2}$ son variables
        dependientes. Como en la tercera columna no contiene pivote $x_{3}$ es una variable libre. Así, sea
        $x_{3}=t$ con $t\in\mathbb{R}$. El conjunto solución es
        \begin{align*}
            S_{3}=\left\{(-\frac{3}{2}t,\frac{1}{2}t,t)\right\}.
        \end{align*}
        Sea
        \begin{align*}
            A_{4}=
            \begin{pmatrix}
                1 & -1 & 2\\
                2 & 0 & 3\\
                0 & -3 & -1\\
                1 & 0 & 0
            \end{pmatrix}
        \end{align*}
        La matriz reducida por filas y escalonada es
        \begin{align*}
            R_{4}=
            \begin{pmatrix}
                1 & 0 & 0\\
                0 & 1 & 0\\
                0 & 0 & 1\\
                0 & 0 & 0
            \end{pmatrix}
        \end{align*}
        Por tanto, el sistema homogéneo asociado a $R_{4}\textbf{x}_{4}=0$. Es decir, explicitamente
        \begin{align*}
            \begin{cases}
                x_{1}=0\\
                x_{2}=0\\
                x_{3}=0
            \end{cases}
        \end{align*}
        Las tres columnas contienen pivote. Por lo tanto $x_{1},x_{2}$ y $x_{3}$ son variables dependientes.
        No hay variables libres. El conjunto solución es
        \begin{align*}
            S_{4}=\left\{(0,0,0)\right\}
        \end{align*}
    \end{solucion}

    \begin{mdframed}[style=mdbluebox,frametitle={Ejercicio 4}]
        En este problema se trabajará sobre $\mathbb{R}$. Considera el sistema
        \begin{align*}
            3x_{1}-x_{2}+2x_{3}&=b_{1}\\
            2x_{1}+x_{2}+x_{3}&=b_{2}\\
            x_{1}-3x_{2}&=b_{3},
        \end{align*}
        encuentra todas las ternas $(b_{1},b_{2},b_{3})$ para las que el sistema tiene alguna solución (justifica tu respuesta).
    \end{mdframed}

    \begin{mdframed}[style=mdbluebox,frametitle={Ejercicio 5}]
        Sea $A\in M_{n\times n}(F)$. Demuestra que $A$ es equivalente por filas a la matriz identidad si y solo si, el sistema homogéneo
        asociado a $A$ tiene solo la solución trivial $(0,\dots,0)\in F^{n}$.
    \end{mdframed}
\end{document}