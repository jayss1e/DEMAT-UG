\documentclass[12pt,a4paper]{article}
\usepackage{ugmath}
\usepackage{float}
\usepackage{placeins}
\usepackage{array}
\usepackage[labelfont=bf,labelsep=space]{caption}
\newcommand{\paren}[1]{\left( #1 \right)}
\newcommand{\alumno}{Ricardo León Martínez}
\newcommand{\materia}{Algebra Lineal I}
\newcommand{\profesor}{Claudia Reynoso Alcántara}
\newcommand{\tarea}{Tarea 3}
\newcommand{\fecha}{13/2/2026}

\begin{document}

    \begin{center}
        {\large\textbf{UNIVERSIDAD DE GUANAJUATO}}\\[0.3cm]
        {\normalsize\textbf{DIVISIÓN DE CIENCIAS NATURALES Y EXACTAS}}\\
        {\normalsize\textbf{CAMPUS GUANAJUATO}}\\[1cm]

        {\Large\textbf{\tarea\ (\materia)}}\\[1cm]
    \end{center}

    \noindent
    \textbf{Nombre:} \alumno 
    \hfill 
    \textbf{Fecha:} \fecha 
    \hfill 
    \textbf{Calificación:} \rule{3cm}{0.4pt}

    \vspace{0.3cm}


    En todos los problemas $F$ es un campo que contiene a $\mathbb{Q}$.

    \begin{mdframed}[style=mdbluebox,frametitle={Ejercicio 1}]
        Enumera todas las posibles matrices reducidas por filas y escalonadas de tamaño $3\times 4$.
        Escribe 0 y 1 donde corresponda; y usa * para lugares donde pueda ir cualquier elemento de $F$.
    \end{mdframed}

    \begin{solucion}
        Para $r=0$
        \begin{equation*}
            \begin{pmatrix}
                0 & 0 & 0 & 0\\
                0 & 0 & 0 & 0\\
                0 & 0 & 0 & 0
            \end{pmatrix}
        \end{equation*}
        Para $r=1$. El pivote puede estar en cualquiera de las 4 columnas.
        \begin{align*}
            \begin{pmatrix}
                1 & * & * & *\\
                0 & 0 & 0 & 0\\
                0 & 0 & 0 & 0
            \end{pmatrix}
            \quad
            \begin{pmatrix}
                0 & 1 & * & *\\
                0 & 0 & 0 & 0\\
                0 & 0 & 0 & 0
            \end{pmatrix}
            \quad
            \begin{pmatrix}
                0 & 0 & 1 & *\\
                0 & 0 & 0 & 0\\
                0 & 0 & 0 & 0
            \end{pmatrix}
            \quad
            \begin{pmatrix}
                0 & 0 & 0 & 1\\
                0 & 0 & 0 & 0\\
                0 & 0 & 0 & 0
            \end{pmatrix}
        \end{align*}
        Para $r=2$. Tomamos dos columnas $i\lt j$ para los pivotes.
        \begin{align*}
            \begin{pmatrix}
                1 & 0 & * & *\\
                0 & 1 & * & *\\
                0 & 0 & 0 & 0
            \end{pmatrix}
            \qquad
            \begin{pmatrix}
                1 & * & 0 & *\\
                0 & 0 & 1 & *\\
                0 & 0 & 0 & 0
            \end{pmatrix}
            \qquad
            \begin{pmatrix}
                1 & * & * & 0\\
                0 & 0 & 0 & 1\\
                0 & 0 & 0 & 0
            \end{pmatrix}\\
            \begin{pmatrix}
                0 & 1 & 0 & *\\
                0 & 0 & 1 & *\\
                0 & 0 & 0 & 0
            \end{pmatrix}
            \qquad
            \begin{pmatrix}
                0 & 1 & * & 0\\
                0 & 0 & 0 & 1\\
                0 & 0 & 0 & 0
            \end{pmatrix}
            \qquad
            \begin{pmatrix}
                0 & 0 & 1 & 0\\
                0 & 0 & 0 & 1\\
                0 & 0 & 0 & 0
            \end{pmatrix}
        \end{align*}
        Para $r=3$. Elegimos tres columnas $i\lt j\lt k$.
        \begin{align*}
            \begin{pmatrix}
                1 & 0 & 0 & *\\
                0 & 1 & 0 & *\\
                0 & 0 & 1 & *
            \end{pmatrix}
            \qquad
            \begin{pmatrix}
                1 & 0 & * & 0\\
                0 & 1 & * & 0\\
                0 & 0 & 0 & 1
            \end{pmatrix}
            \qquad
            \begin{pmatrix}
                1 & * & 0 & 0\\
                0 & 0 & 1 & 0\\
                0 & 0 & 0 & 1
            \end{pmatrix}
            \qquad
            \begin{pmatrix}
                0 & 1 & 0 & 0\\
                0 & 0 & 1 & 0\\
                0 & 0 & 0 & 1
            \end{pmatrix}
        \end{align*}
        Por lo tanto existen 15 posibles matrices.
    \end{solucion}

    \begin{mdframed}[style=mdbluebox,frametitle={Ejercicio 2}]
        Responde si es falso (F) o verdadero (V) y justifica tu respuesta:
        \begin{enumerate}
            \item[(a)] Si un sistema de ecuaciones lineales $A\textbf{x}=\textbf{B}$ tiene más variables que
            ecuaciones, entonces el sistema tiene infinitas soluciones.
            \item[(b)] Un sistema consistente tiene una solución única si y solo si no existen variables libres.
            \item[(c)] Si el sistema homogéneo $A\textbf{x}=\textbf{0}$ tiene al menos una variable libre, entonces
            el sistema $A\textbf{x}=\textbf{B}$ tiene infinitas soluciones para cualquier matriz $\textbf{B}$.
            \item[(d)] Si el sistema de ecuaciones lineales $A\textbf{x}=\textbf{B}$ tiene al menos una variable libre, entonces
            el sistema es consistente. 
        \end{enumerate}
    \end{mdframed}

    \begin{solucion}
        \begin{enumerate}
            \item[(a)] \textbf{Falso}\\
            Tener más variables que ecuaciones no garantiza que el sistema sea consistente. Por ejemplo,
            \begin{equation*}
                \begin{cases}
                    x+y+z=1\\
                    x+y+z=2
                \end{cases}
            \end{equation*}
            Aquí hay mas variables que ecuaciones pero el sistema es incompatible, pues al restar ambas
            ecuaciones se obtiene $0=1$. Por lo tanto, no necesariamente hay soluciones.
            \item[(b)] \textbf{Verdadero}\\
            Una variable libre aparece cuando su columna no contiene pivote en la forma escalonada. Si el sistema
            es consistente y no hay variables libres, entonces cada variables corresponde a un pivote y todas las variables
            quedan determinadas de manera unica. Por lo tanto existe solucion unica. De igual manera, si existe almenos una
            variable libre, esta puede tomar infinitos valores, generando infinitas soluciones.
            \item[(c)] \textbf{Falso}\\
            Que el sistema homogéneo tenga variable libre implica que tiene infinitas soluciones. Pero no garantiza
            que el sistema no homogéneo sea consistente. Puede ocurrir que para cierto $B$ el sistema sea incompatible.
            Por ejemplo,
            \begin{equation*}
                \begin{cases}
                    x+y=1\\
                    x+y=2
                \end{cases}
            \end{equation*}
            El sistema homogéneo asociado es
            \begin{equation*}
                x+y=0
            \end{equation*}
            tiene variable libre, pero el sistema original es incompatible.
            \item[(d)] \textbf{Verdadero}\\
            Si al reducir la matriz aumentada a forma escalonada aparece una variable libre, significa que
            no todas las columnas contienen pivotes. Pero no aparece una fila de ceros, pues en tal caso el
            sistema sería incompatible y no se hablaría de variables libres en la solución. La presencia
            de variables libres solo sucede cuando el sistema es consisitente y tiene infinitas soluciones.
        \end{enumerate}
    \end{solucion}

    \begin{mdframed}[style=mdbluebox,frametitle={Ejercicio 3}]
        En este problema se trabajará sobre $\mathbb{R}$. Encuentra la matriz reducida por filas y escalonada
        equivalente a:
        \begin{align*}
            \left(
                \begin{array}{c}
                    1\\
                    2\\
                    7\\
                    0
                \end{array}
            \right),
            \qquad
            \left(
                \begin{array}{cc}
                    1 & -2\\
                    0 & 0\\
                    11 & 1+\sqrt{2}
                \end{array}
            \right),
            \qquad
            \left(
                \begin{array}{ccc}
                    1 & -1 & 2\\
                    2 & 0 & 3
                \end{array}
            \right),
            \qquad
            \left(
                \begin{array}{ccc}
                    1 & -1 & 2\\
                    2 & 0 & 3\\
                    0 & -3 & -1\\
                    1 & 0 & 0
                \end{array}
            \right),
        \end{align*}
        encuentra las soluciones del sistema homogéneo correspondiente. Finalmente, determina cuáles son las variables independientes
        y dependientes en cada caso.
    \end{mdframed}

    \begin{solucion}
        Sea
        \begin{align*}
            A_{1}=
            \begin{pmatrix}
                1\\
                2\\
                7\\
                0
            \end{pmatrix}
        \end{align*}
        La matriz reducida por filas y escalonada es
        \begin{align*}
            R_{1}=
            \begin{pmatrix}
                1\\
                0\\
                0\\
                0
            \end{pmatrix}
        \end{align*}
        Por tanto, el sistema homogéneo asociado es $R_{1}\textbf{x}_{1}=0$. Es decir, explicitamente
        \begin{align*}
            \begin{cases}
                x_{1}=0.
            \end{cases}
        \end{align*}
        La primera columna contiene un pivote. Por tanto, $x_{1}$ es variable dependiente. No hay variables libres.
        El conjunto solución es
        \begin{align*}
            S_{1}=\left\{0\right\}.
        \end{align*}
        Sea
        \begin{align*}
            A_{2}=
            \begin{pmatrix}
                1 & -2\\
                0 & 0\\
                11 & 1+\sqrt{2}
            \end{pmatrix}
        \end{align*}
        La matriz reducida por filas y escalonada es
        \begin{align*}
            R_{2}=
            \begin{pmatrix}
                1 & -2\\
                0 & 1\\
                0 & 0
            \end{pmatrix}
        \end{align*}
        Por tanto, el sistema homogéneo asociado es $R_{2}\textbf{x}_{2}=0$. Es decir, explicitamente
        \begin{align*}
            \begin{cases}
                x_{1}+x_{2}=0\\
                x_{2}=0
            \end{cases}
        \end{align*}
        En la primera y en la segunda columna contienen un pivote. Por lo tanto, $x_{1}$ y $x_{2}$ son
        variables dependientes. No hay variables libres. El conjunto solucion es
        \begin{align*}
            S_{2}=\left\{(0,0)\right\}
        \end{align*}
        Sea
        \begin{align*}
            A_{3}=
            \begin{pmatrix}
                1 & -1 & 2\\
                2 & 0 & 3
            \end{pmatrix}
        \end{align*}
        La matriz reducida por fila y escalonada es
        \begin{align*}
            R_{3}=
            \begin{pmatrix}
                1 & 0 & \frac{3}{2}\\
                0 & 1 & -\frac{1}{2}
            \end{pmatrix}
        \end{align*}
        Por tanto, el sistema homogéneo asociado es $R_{3}\textbf{x}_{3}=0$. Es decir explicitamente
        \begin{align*}
            \begin{cases}
                x_{1}+\frac{3}{2}x_{3}=0\\
                x_{2}-\frac{1}{2}x_{3}=0
            \end{cases}
        \end{align*}
        En la primera y en la segunda columna contiene un pivote. Por lo tanto $x_{1}$ y $x_{2}$ son variables
        dependientes. Como en la tercera columna no contiene pivote $x_{3}$ es una variable libre. Así, sea
        $x_{3}=t$ con $t\in\mathbb{R}$. El conjunto solución es
        \begin{align*}
            S_{3}=\left\{(-\frac{3}{2}t,\frac{1}{2}t,t)\right\}.
        \end{align*}
        Sea
        \begin{align*}
            A_{4}=
            \begin{pmatrix}
                1 & -1 & 2\\
                2 & 0 & 3\\
                0 & -3 & -1\\
                1 & 0 & 0
            \end{pmatrix}
        \end{align*}
        La matriz reducida por filas y escalonada es
        \begin{align*}
            R_{4}=
            \begin{pmatrix}
                1 & 0 & 0\\
                0 & 1 & 0\\
                0 & 0 & 1\\
                0 & 0 & 0
            \end{pmatrix}
        \end{align*}
        Por tanto, el sistema homogéneo asociado a $R_{4}\textbf{x}_{4}=0$. Es decir, explicitamente
        \begin{align*}
            \begin{cases}
                x_{1}=0\\
                x_{2}=0\\
                x_{3}=0
            \end{cases}
        \end{align*}
        Las tres columnas contienen pivote. Por lo tanto $x_{1},x_{2}$ y $x_{3}$ son variables dependientes.
        No hay variables libres. El conjunto solución es
        \begin{align*}
            S_{4}=\left\{(0,0,0)\right\}
        \end{align*}
    \end{solucion}

    \begin{mdframed}[style=mdbluebox,frametitle={Ejercicio 4}]
        En este problema se trabajará sobre $\mathbb{R}$. Considera el sistema
        \begin{align*}
            3x_{1}-x_{2}+2x_{3}&=b_{1}\\
            2x_{1}+x_{2}+x_{3}&=b_{2}\\
            x_{1}-3x_{2}&=b_{3},
        \end{align*}
        encuentra todas las ternas $(b_{1},b_{2},b_{3})$ para las que el sistema tiene alguna solución (justifica tu respuesta).
    \end{mdframed}

    \begin{solucion}
        Sea el sistema de ecuaciones lineales
        \begin{equation*}
            \begin{cases}
                3x_{1}-x_{2}+2x_{3}=b_{1}\\
                2x_{1}+x_{2}+x_{3}=b_{2}\\
                x_{1}-3x_{2}=b_{3}
            \end{cases}
        \end{equation*}
        Su matriz extendida asociada es
        \begin{equation*}
            \left(
                \begin{array}{ccc|c}
                    3 & -1 & 2 & b_{1}\\
                    2 & 1 & 1 & b_{2}\\
                    1 & -3 & 0 & b_{3}
                \end{array}
            \right)
        \end{equation*}
        El sistema reducido por filas escalonado es
        \begin{align*}
            \left(
                \begin{array}{ccc|c}
                    1 & 0 & 0 & \frac{-b_{1}+2b_{2}+b_{3}}{2}\\
                    0 & 1 & 0 & \frac{-b_{1}+2b_{2}-b_{3}}{6}\\
                    0 & 0 & 1 & \frac{7b_{1}-8b_{2}-5b_{3}}{6}
                \end{array}
            \right)
        \end{align*}
        Por lo tanto, el sistema tiene solución para todo $(b_{1},b_{2},b_{3})\in\mathbb{R}$.
    \end{solucion}

    \begin{mdframed}[style=mdbluebox,frametitle={Ejercicio 5}]
        Sea $A\in M_{n\times n}(F)$. Demuestra que $A$ es equivalente por filas a la matriz identidad si y solo si, el sistema homogéneo
        asociado a $A$ tiene solo la solución trivial $(0,\dots,0)\in F^{n}$.
    \end{mdframed}

    \begin{proof}
        Supongamos que $A$ es equivalente por filas a $I_{n}$. Entonces existe una suceción finita
        de operaciones elementales por filas que transforma $A$ en $I_{n}$. Aplicando esas mismas operaciones
        al sistema homogéneo $A\textbf{x}=0$, obtenemos un sistema equialente
        \begin{equation*}
            I_{n}\textbf{x}=0.
        \end{equation*}
        Pero
        \begin{equation*}
            I_{n}\textbf{x}=0 \Longleftrightarrow x_{1}=0,\dots,x_{n}=0.
        \end{equation*}
        Por tanto, $A$ solo tiene la solución trivial. Supongamos ahora que el sistema
        homogéneo
        \begin{equation*}
            A\textbf{x}=0
        \end{equation*}
        tiene únicamente la solución trivial. Reducimos $A$ por filas hasta obtener su forma
        reducida por filas y escalonada $R$. El sistema
        \begin{equation*}
            R\textbf{x}=0
        \end{equation*}
        es equivalente al sistema original, luego también tiene solo la solución trivial.
        Si $R$ tuviera una fila nula, entonces habría menos de $n$ pivotes, y por lo tanto existiria
        al menos una variable libre. Una variable libre permitiría soluciones no triviales
        del sistema homogéneo, contradicción. Por lo tanto no hay filas nulas, cada columna contiene
        un pivote y hay exactamente $n$ pivotes. Dado que la matriz es $n\times n$, esto implica que
        los pivotes están necesariamente en todas las columnas. Ademas, al estar en forma reducida por filas
        cada pivote es 1 y cada columna tiene ceros en las demas entradas. Luego,
        \begin{equation*}
            R=I_{n}
        \end{equation*}
        Como $R$ se obtuvo mediante operaciones elementales por filas, concluimos que $A$ es equivalente por filas
        a $I_{n}$.
    \end{proof}
\end{document}