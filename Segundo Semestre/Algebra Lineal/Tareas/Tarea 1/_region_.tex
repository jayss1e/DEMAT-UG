\message{ !name(tarea1.tex)}\documentclass[12pt,a4paper]{article}
\usepackage{ugmath}
\usepackage{float}
\usepackage{placeins}
\usepackage[labelfont=bf,labelsep=space]{caption}
\newcommand{\alumno}{Ricardo León Martínez}
\newcommand{\materia}{Algebra Lineal I}
\newcommand{\profesor}{Claudia Reynoso Alcántara}
\newcommand{\tarea}{Tarea 1}
\newcommand{\fecha}{2/2/2026}

\begin{document}

\message{ !name(tarea1.tex) !offset(-3) }

    \begin{center}
    {\large\textbf{UNIVERSIDAD DE GUANAJUATO}}\\[0.3cm]
    {\normalsize\textbf{DIVISIÓN DE CIENCIAS NATURALES Y EXACTAS}}\\
    {\normalsize\textbf{CAMPUS GUANAJUATO}}\\[1cm]

    {\Large\textbf{\tarea\ (\materia)}}\\[1cm]
    \end{center}

    \textbf{Nombre:} \alumno \hfill 
    \textbf{Fecha:} \fecha \hfill 
    \textbf{Calificación:} \rule{3cm}{0.4pt} \\[0.3cm]

    En todos los problemas, $F$ es el campo $\mathbb{Q}$, $\mathbb{R}$ o $\mathbb{C}$.
    \begin{mdframed}[style=mdbluebox,frametitle={Ejercicio 1}]
        Demuestra que el cero de $F$ es único y que para todo $a\in F\setminus\{0\}$,
        su inverso multiplicativo, $a^{-1}$ es único.
    \end{mdframed}
    \begin{proof}
        Supongamos que $a,0\in F$ son elementos neutros para la suma. Entonces para toda
        $a\in F$ se cumple
        \begin{align*}
            a+0=a \quad \text{y} \quad a+b=a.
        \end{align*}
        Sumando $-a$ a ambos lados de la segunda igualdad se obtiene $b=0$, por lo que
        el 0 de $F$ es unico. Ahora sea $a\in F\setminus\{0\}$ y supongamos que $b,c\in F$
        son tales que
        \begin{align*}
            ab=1 \quad \text{y} \quad ac=1.
        \end{align*}
        Entonces,
        \begin{align*}
            b=b\cdot 1=b(ac)=(ba)c=1\cdot c.
        \end{align*}
        Por lo tanto el inverso multiplicativo de $a$ es unico.
    \end{proof}

    \begin{mdframed}[style=mdbluebox,frametitle={Ejercicio 2}]
        Considera el siguiente sistema de 3 ecuaciones lineales en 2 variables con
        coeficientes en F:
        \begin{align*}
            \begin{cases}
                x_{1}+6x_{2}=2\\
                x_{1}-x_{2}=2\\
                x_{1}+x_{2}=1.
            \end{cases}
        \end{align*}
        Encuentra el conjunto solución. ¿Existe alguna solución del sistema homogéneo? Justifica
        tus respuestas.
    \end{mdframed}
    \begin{solucion}
        Consideremos el siguiente sistema de ecuaciones
        \begin{align*}
            S=
            \begin{cases}
                E_{1}:x_{1}+6x_{2}=2\\
                E_{2}:x_{1}-x_{2}=2\\
                E_{3}:x_{1}+x_{2}=1.
            \end{cases}
        \end{align*}
        Resolviendo $E_{2}$ a $E_{3}$ se obtiene al par
        \begin{align*}
            (x_{1},x_{2})=(\frac{3}{2},-\frac{1}{2}).
        \end{align*}
        Pero este par no satisface a $E_{1}$. Ahora si resolvemos $E_{1}$ y $E_{2}$ obtenemos
        \begin{align*}
            (x_{1},x_{2})=(2,0).
        \end{align*}
        Pero esto no satisface $E_{3}$. Dado que no existe ningun par que satisfaga simultaneamente
        las tres ecuaciones, se concluye que el sistema es incompatible. Por lo tanto, el conjunto
        solución es
        \begin{align*}
            \text{Sol}(S)=\emptyset
        \end{align*}
        Ahora consideremos el siguiente sistema de ecuaciones
        \begin{align*}
            A=
            \begin{cases}
                E_{1}:x_{1}+6x_{2}=0\\
                E_{2}:x_{1}-x_{2}=0\\
                E_{3}:x_{1}+x_{2}=0.
            \end{cases}
        \end{align*}
        De $E_{2}$ se tiene que $x_{1}=x_{2}$ sustituyendo $x_{2}$ en $E_{3}$ se sigue que
        $x_{1}=0$ y $x_{2}=0$ por lo tanto existe solucion para el sistema homogeneo y su
        conjunto solucion es,
        \begin{align*}
            \text{Sol}(A)=\{(0,0)\}.
        \end{align*}
    \end{solucion}

    \begin{mdframed}[style=mdbluebox,frametitle={Ejercicio 3}]
        Sean $m,n\in\mathbb{N}$. Demuestra que la relación definida en el conjutno de sistemas
        de $m$ ecuaciones lineales en $n$ variables con coeficientes en $F$ de ser \textbf{ser equivalentes},
        es una relación transitiva.
    \end{mdframed}
    \begin{proof}
        Sea $S$ el conjunto de todos los sistmas de ecuaciones lineales, decimos que dos
        sistemas estan relacionados o que $S\sim S^{\prime}$ si su conjunto solucion es el mismo.
        Ahora sean $S_{1},S_{2},S_{3}\in S$ tales que
        \begin{align*}
            S_{1}\sim S_{2} \quad \text{y} \quad S_{2}\sim S_{3}
        \end{align*}
        Por definicion de la relacion $\sim$, se tiene
        \begin{align*}
            \text{Sol}(S_{1})=\text{Sol}(S_{2}) \quad \text{y} \quad \text{Sol}(S_{2})=\text{Sol}(S_{3}).
        \end{align*}
        Por transitividad de la igualdad entre conjuntos, se sigue que
        \begin{align*}
            \text{Sol}(S_{1})=\text{Sol}(S_{3})
        \end{align*}
        Nuevamente por la definicion de relacion entre sistemas de ecuaciones se concluye que
        \begin{align*}
        S_{1}\sim S_{3}
        \end{align*}
        Y por lo tanto es transititva.
    \end{proof}

    \vspace{1cm}

    \begin{mdframed}[style=mdbluebox,frametitle={Ejercicio 4}]
        Sean $a_{ij},b_{j}\in F$, $1\leq i\leq 2$, $1\leq j\leq 2$, con $a_{11}\neq0$. Da condiciones
        necesarias y suficientes para el siguiente sistema lineal de 2 ecuaciones y 2 variables tenga
        conjunto solución no vacío:
        \begin{align*}
            \begin{cases}
                a_{11}x_{1}+a_{12}x_{2}=b_{1}\\
                a_{21}x_{1}+a_{22}x_{2}=b_{2}.
            \end{cases}
        \end{align*}
    \end{mdframed}
    \begin{solucion}
        Para $a_{11}\neq0$, podemos aplicar operaciones elementales y despejar $x_{1}$ de la primera ecuación obteniendo
        \begin{align*}
            x_{1}=\frac{b_{1}}{a_{11}}-\frac{a_{12}}{a_{11}}x_{2}.
        \end{align*}
        Sustituyendo en la segunda ecuacion
        \begin{align*}
            a_{21}\left(\frac{b_{1}}{a_{11}}-\frac{a_{12}}{a_{11}}x_{2}\right)+a_{22}x_{2}=b_{2}.
        \end{align*}
        Reordenando
        \begin{align*}
            \left(a_{22}-\frac{a_{21}a_{12}}{a_{11}}\right)x_{2}=b_{2}-\frac{a_{21}}{a_{11}}b_{1}.
        \end{align*}
        Ahora consideremos dos casos. Primero supongamos que
        \begin{align*}
            a_{22}-\frac{a_{21}a_{12}}{a_{11}}\neq0.
        \end{align*}
        Entonces la ecuacion tiene solución unica para $x_{2}$, y por tanto eol sistema tiene solución.
        Ahora supongamos que
        \begin{align*}
            a_{22}-\frac{a_{21}a_{12}}{a_{11}}=0.
        \end{align*}
        Entonces la ecuación se reduce a
        \begin{align*}
            0\cdot x_{2}=b_{2}.
        \end{align*}
        Si
        \begin{align*}
            b_{2}-\frac{a_{21}}{a_{11}}b_{1}=0,
        \end{align*}
        el sistema tiene infinitas soluciones.Si
        \begin{align*}
            b_{2}-\frac{a_{21}}{a_{11}}b_{1}\neq0,
        \end{align*}
        el sistema no tiene solución. Asi el sistema tiene conjunto solución no vacio si y solo si
        cumple
        \begin{align*}
            \begin{aligned}
                a_{22}-\frac{a_{21}a_{12}}{a_{11}}\neq0
            \end{aligned}
            \qquad
            \text{o bien}
            \qquad
            \begin{aligned}
                \begin{cases}
                    a_{22}-\frac{a_{21}a_{12}}{a_{11}}=0,\\
                    b_{2}-\frac{a_{21}}{a_{11}}b_{1}=0.
                \end{cases}
            \end{aligned}
        \end{align*}
    \end{solucion}

    \begin{mdframed}[style=mdbluebox,frametitle={Ejercicio 5}]
        Determina si los siguientes sistemas de ecuaciones son equivalentes, y si lo son, expresa
        cada ecuación de uno como combinación lineal de las ecuaciones de otro:
        \begin{enumerate}
            \item[(a)] 3 ecuaciones y 3 variables:
            \[
                \begin{aligned}
                    \begin{cases}
                        -x_{1}+x_{2}+4x_{3}=0\\
                        x_{1}+3x_{2}+8x_{3}=0\\
                        \frac{1}{2}x_{1}+x_{2}+\frac{5}{2}x_{3}=0
                    \end{cases}
                \end{aligned}
                \qquad
                \begin{aligned}
                    \begin{cases}
                        x_{1}-x_{3}=0\\
                        x_{2}+3x_{3}=0
                    \end{cases}
                \end{aligned}
            \]
            \item[(b)] 3 ecuaciones y 3 variables:
            \begin{align*}
                \begin{aligned}
                    \begin{cases}
                        -x_{1}+x_{2}+4x_{3}=0\\
                        x_{1}+3x_{2}+8x_{3}=0\\
                        \frac{1}{2}x_{1}+x_{2}+\frac{5}{2}x_{3}=0
                    \end{cases}
                \end{aligned}
                \qquad
                \begin{aligned}
                    \begin{cases}
                        x_{1}-x_{3}=0\\
                        x_{2}+x_{3}=0
                    \end{cases}
                \end{aligned}
            \end{align*}
        \end{enumerate}
    \end{mdframed}
    \begin{solucion}
        Consideremos los dos siguientes sistemas de ecuaciones
        \begin{align*}
            \begin{aligned}
                S_{1}=
                \begin{cases}
                    -x_{1}+x_{2}+4x_{3}=0\\
                    x_{1}+3x_{2}+8x_{3}=0\\
                    \frac{1}{2}x_{1}+x_{2}+\frac{5}{2}x_{3}=0.
                \end{cases}
            \end{aligned}
            \qquad
            \begin{aligned}
                S_{2}=
                \begin{cases}
                    x_{1}-x_{3}=0\\
                    x_{2}+3x_{3}=0.
                \end{cases}
            \end{aligned}
        \end{align*}
        Veamos si son equivalentes. La matriz aumentanda de $S_{1}$ es
        \begin{align*}
            \left(
                \begin{array}{ccc|c}
                    -1 & 1 & 4 & 0\\
                    1 & 3 & 8 & 0\\
                    \frac{1}{2} & 1 & \frac{5}{2} & 0
                \end{array}
            \right)
        \end{align*}
        Ahora multiplicamos la fila 1 por $-1$ y la restamos a la fila 2 obteniendo
        \begin{align*}
            \left(
                \begin{array}{ccc|c}
                    -1 & 1 & 4 & 0\\
                    0 & 4 & 12 & 0\\
                    \frac{1}{2} & 1 & \frac{5}{2} & 0
                \end{array}
            \right)
        \end{align*}
        despues multiplicamos la fila 1 por $-\frac{1}{2}$ y la restamos a la fila 3 teniendo
        \begin{align*}
            \left(
                \begin{array}{ccc|c}
                    -1 & 1 & 4 & 0\\
                    0 & 4 & 12 & 0\\
                    0 & \frac{3}{2} & \frac{9}{2} & 0
                \end{array}
            \right)
        \end{align*}
        finalmente multiplicamos la fila 2 por $\frac{3}{8}$ y la restamos a la fila 3 obteniendo
        la siguiente matriz
        \begin{align*}
                        \left(
                \begin{array}{ccc|c}
                    -1 & 1 & 4 & 0\\
                    0 & 4 & 12 & 0\\
                    0 & 0 & 0 & 0
                \end{array}
            \right)
        \end{align*}
        o bien el sistema de ecuaciones lineales
        \begin{align}
            \begin{cases}
                -x_{1}+x_{2}+4x_{3}=0\\
                \hfill 4x_{2}+12x_{3}=0.
            \end{cases}
        \end{align}
        De la ecuaciónn 2 de (1) obtenemos
        \begin{align*}
            4x_{2}=12_{x_{3}} \implies x_{2}=-3x_{3}
        \end{align*}
        y de la ecuación 1 de (1) se obtiene
        \begin{align*}
            -x_{1}=-x_{2}-4x_{3}=-(-3x_{3})-4x_{3}=-x_{3} \implies x_{1}=x_{3}.
        \end{align*}
        Sea $x_{3}=t$, con $t\in\mathbb{R}$. Entonces
        \begin{align*}
            (x_{1},x_{2},x_{3})=(t,-3t,t).
        \end{align*}
        y asi su conjunto solución es
        \begin{align*}
            S=\{(t,-3t,t):t\in\mathbb{R}\}.
        \end{align*}
        Tomemos ahora $S_{2}$
        \begin{align*}
            \begin{cases}
                x_{1}-x_{3}=0\\
                x_{2}+3_{3}=0
            \end{cases}
        \end{align*}
        de aqui se deduce inmediatamente
        \begin{align*}
            x_{1}=x_{3}, \qquad x_{2}=-3x_{3}.
        \end{align*}
        Sea $x_{3}=t$, con $t\in\mathbb{R}$. Entonces
        \begin{align*}
            (x_{1},x_{2},x_{3})=(t,-3t,t).
        \end{align*}
        y asi su conjunto solucion es
        \begin{align*}
            S=\{(t,-3t,t):t\in\mathbb{R}\}.
        \end{align*}
        Dado que los conjuntos solucion de $S_{1}$ y $S_{2}$ son iguales entonces los sitemas son
        equivalentes. Se verifica que cada ecuación de $S_{1}$ puede expresarse como combinación lineal de las
        ecuaciones de $S_{2}$. En efecto
            \begin{align*}
                -x_{1}+x_{2}+4x_{3}=-(x_{1}-x_{3})+(x_{2}+3x_{3}),\\
                x_{1}+3x_{2}+8x_{3}=(x_{1}-x_{3})+3(x_{2}+3x_{3}),\\
                \frac{1}{2}x_{1}+x_{2}+\frac{5}{2}x_{3}=\frac{1}{2}(x_{1}-x_{3})+(x_{2}+3x_{3}).
            \end{align*}
        Ahora consideremos el sistema $S_{2}^{\prime}$ dado por
        \begin{align*}
            \begin{cases}
                x_{1}-x_{3}=0\\
                x_{2}+x_{3}=0
            \end{cases}
        \end{align*}
        Es claro que
        \begin{align*}
            x_{1}=x_{3}, \qquad x_{2}=-x_{3}.
        \end{align*}
        y por lo tanto su conjunto solución es
        \begin{align*}
            S(S_{2}^{\prime})=\{(t,-t,t):t\in\mathbb{R}\}.
        \end{align*}
        Asi, ya que el conjunto solución de $S_{2}^{\prime}$ no es igual al de $S_{1}$
        no son sistemas equivalentes.
    \end{solucion}

    \begin{mdframed}[style=mdbluebox,frametitle={Ejercicio 6}]
        ¿Para qué valores de $a\in\mathbb{Q}$ el siguiente sistema tiene una infinidad de soluciones?
        \begin{align*}
            \begin{cases}
                ax_{1}+2x_{2}=0\\
                2x_{1}+ax_{2}=0.
            \end{cases}
        \end{align*}
    \end{mdframed}
    \begin{solucion}
        Queremos que exista $\lambda\in\mathbb{Q}$ tal que
        \begin{align*}
            (2x_{1}+ax_{2})=\lambda(ax_{1}+2x_{2}).
        \end{align*}
        Igualamos coeficientes
        \begin{align*}
            \begin{cases}
                2=\lambda a\\
                a=2\lambda
            \end{cases}
        \end{align*}
        De la segunda ecuación obtenemos
        \begin{align*}
            \lambda=\frac{a}{2}.
        \end{align*}
        Sustituyendo en la primera
        \begin{align*}
            2=a\left(\frac{a}{2}\right) \implies a^{2}=4.
        \end{align*}
        Por lo tanto
        \begin{align*}
            a=\pm2.
        \end{align*}
      \end{solucion}

      \begin{align*}
        \int_{a}^{b}f(x).
      \end{align*}
      

\end{document}

\message{ !name(tarea1.tex) !offset(-410) }
