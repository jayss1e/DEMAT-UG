\documentclass[12pt,a4paper]{article}
\usepackage{ugmath}
\usepackage{float}
\usepackage{placeins}
\usepackage{array}
\usepackage[labelfont=bf,labelsep=space]{caption}
\newcommand{\paren}[1]{\left( #1 \right)}
\newcommand{\alumno}{Ricardo León Martínez}
\newcommand{\materia}{Algebra Lineal I}
\newcommand{\profesor}{Claudia Reynoso Alcántara}
\newcommand{\tarea}{Tarea 4}
\newcommand{\fecha}{23/2/2026}

\begin{document}

    \begin{center}
        {\large\textbf{UNIVERSIDAD DE GUANAJUATO}}\\[0.3cm]
        {\normalsize\textbf{DIVISIÓN DE CIENCIAS NATURALES Y EXACTAS}}\\
        {\normalsize\textbf{CAMPUS GUANAJUATO}}\\[1cm]

        {\Large\textbf{\tarea\ (\materia)}}\\[1cm]
    \end{center}

    \noindent
    \textbf{Nombre:} \alumno 
    \hfill 
    \textbf{Fecha:} \fecha 
    \hfill 
    \textbf{Calificación:} \rule{3cm}{0.4pt}

    \vspace{0.3cm}

    Sea $F$ el campo $\mathbb{Q}$, $\R$ o $\C$

    \begin{mdframed}[style=mdbluebox,frametitle={Ejercicio 1}]
        Demuestra por inducción en el número de filas que el producto de matrices
        con coeficientes en un capo $F$ es asociativo.
    \end{mdframed}

    \begin{mdframed}[style=mdbluebox,frametitle={Ejercicio 2}]
        Sean $A$,$B$ matrices con coeficientes en un campo $F$, de tamaño $2\times 1$, $1\times2$,
        respectivamente. Demuestra que $AB$ no es invertible.
    \end{mdframed}

    \begin{mdframed}[style=mdbluebox,frametitle={Ejercicio 3}]
        Determina si la matriz
        \begin{equation*}
            A=
            \left(
                \begin{array}{cccc}
                    1 & 2 & 3 & 4\\
                    0 & 1 & 3 & 4\\
                    0 & 0 & 1 & 4\\
                    0 & 0 & 0 & 1
                \end{array}
            \right),
        \end{equation*}
        es invertible.
    \end{mdframed}

    \begin{mdframed}[style=mdbluebox,frametitle={Ejercicio 4}]
        Sea $A$ una matriz cuadrada $n\times n$ con coeficientes en $F$. Demuestra que si $A$
        no es invertible, entonces el sistema lineal homogéneo, $AX=0$ tiene una infinidad de
        soluciones.
    \end{mdframed}

    \begin{mdframed}[style=mdbluebox,frametitle={Ejercicio 5}]
        Sea $A\in M_{n\times n}(F)$. Demuestra que si $A$ tiene inversa derecho o inversa
        izquierda, entonces $A$ es invertible.
    \end{mdframed}

    \begin{mdframed}[style=mdbluebox,frametitle={Ejercicio 6}]
        Sea $A$ una matriz $n\times n$. Demuestra que si $A$ es invertible y $AB=0_{n\times n}$
        para alguna matriz $B$ de tamaño $n\times n$, entonces $B=0_{n\times n}$.
    \end{mdframed}

    \begin{mdframed}[style=mdbluebox,frametitle={Ejercicio 7}]
        Sean
        \begin{equation*}
            A_{1}=
            \left(
                \begin{array}{ccc}
                    2 & 0 & 0\\
                    1 & -3 & 1\\
                    0 & 1 & 3
                \end{array}
            \right),
            \qquad
            \left(
                \begin{array}{cccc}
                    1 & 2 & 1 & 0\\
                    1 & 0 & 3 & 5\\
                    1 & 0 & 1 & 1
                \end{array}
            \right)
        \end{equation*}
        matrices con coeficientes en $\R$. Encuentra las matrices reducidas por filas y escalonadas
        $R_{1}$, $R_{2}$ equivalentes a $A_{1}$ y $A_{2}$, respectivamente. Encuentra las matrices
        invertibles $P_{1}$, $P_{2}$ tales que $R_{1}=P_{1}A_{1}$ y $R_{2}=P_{2}A_{2}$.
    \end{mdframed}
\end{document}