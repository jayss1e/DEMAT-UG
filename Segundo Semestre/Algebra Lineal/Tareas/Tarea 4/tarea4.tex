\documentclass[12pt,a4paper]{article}
\usepackage{ugmath}
\usepackage{float}
\usepackage{placeins}
\usepackage{array}
\usepackage[labelfont=bf,labelsep=space]{caption}
\newcommand{\paren}[1]{\left( #1 \right)}
\newcommand{\alumno}{Ricardo León Martínez}
\newcommand{\materia}{Algebra Lineal I}
\newcommand{\profesor}{Claudia Reynoso Alcántara}
\newcommand{\tarea}{Tarea 4}
\newcommand{\fecha}{23/2/2026}

\begin{document}

    \begin{center}
        {\large\textbf{UNIVERSIDAD DE GUANAJUATO}}\\[0.3cm]
        {\normalsize\textbf{DIVISIÓN DE CIENCIAS NATURALES Y EXACTAS}}\\
        {\normalsize\textbf{CAMPUS GUANAJUATO}}\\[1cm]

        {\Large\textbf{\tarea\ (\materia)}}\\[1cm]
    \end{center}

    \noindent
    \textbf{Nombre:} \alumno 
    \hfill 
    \textbf{Fecha:} \fecha 
    \hfill 
    \textbf{Calificación:} \rule{3cm}{0.4pt}

    \vspace{0.3cm}

    Sea $F$ el campo $\mathbb{Q}$, $\R$ o $\C$

    \begin{mdframed}[style=mdbluebox,frametitle={Ejercicio 1}]
        Demuestra por inducción en el número de filas que el producto de matrices
        con coeficientes en un capo $F$ es asociativo.
    \end{mdframed}

    \begin{proof}
        Sea $A=(a_{1r})$ una matriz $1\times n$. Caso base $m=1$. Para cada $j$, la entrada $(1,j)$ de $(AB)C$ es
        \begin{equation*}
            ((AB)C)_{1j}=\sum_{k=1}^{n}(AB)_{1k}c_{kj}=\sum_{k=1}^{n}\left(\sum_{r=1}^{p}a_{1r}b_{rk}\right)c_{kj}.
        \end{equation*}
        Por distributiidad en $F$
        \begin{equation*}
            ((AB)C)_{1j}=\sum_{k=1}^{n}\sum_{r=1}^{p}a_{1r}b_{rk}c_{kj}.
        \end{equation*}
        Como las sumsa son finitas, podemos cambiar el orden
        \begin{equation*}
            ((AB)C)_{1j}=\sum_{r=1}^{p}\sum_{k=1}^{n}a_{1r}b_{rk}c_{kj}.
        \end{equation*}
        Factorizando $a_{1r}$,
        \begin{equation*}
            ((AB)C)_{1j}=\sum_{r=1}^{p}a_{1r}\left(\sum_{k=1}^{n}b_{rk}c_{kj}\right).
        \end{equation*}
        De esta forma, se obtiene
        \begin{equation*}
            ((AB)C)_{1j}=(A(BC))_{1j},
        \end{equation*}
        cuando $A$ tiene una fila. Supongamos que el resultado es valido para
        matrices con $m$ filas. Definimos
        \begin{equation*}
            A=
            \begin{pmatrix}
                A^{\prime}\\
                r
            \end{pmatrix},
        \end{equation*}
        Donde $A^{\prime}$ tiene $m$ filas y $r$ es la ultima fila. Por hipotesis inductiva,
        \begin{equation*}
            (A^{\prime}B)C=A^{\prime}(BC).
        \end{equation*}
        Por el caso base,
        \begin{equation*}
            (rB)C=r(BC).
        \end{equation*}
        Por lo tanto, las primeras $m$ filas coinciden y la ultima fila tambien coincide. Luego,
        \begin{equation*}
            (AB)C=A(BC).
        \end{equation*}
    \end{proof}

    \begin{mdframed}[style=mdbluebox,frametitle={Ejercicio 2}]
        Sean $A$,$B$ matrices con coeficientes en un campo $F$, de tamaño $2\times 1$, $1\times2$,
        respectivamente. Demuestra que $AB$ no es invertible.
    \end{mdframed}

    \begin{proof}
        Sea
        \begin{equation*}
            A=
            \begin{pmatrix}
                a\\
                b
            \end{pmatrix},
            \qquad
            B=
            \begin{pmatrix}
                c & d
            \end{pmatrix}
        \end{equation*}
        Entonces
        \begin{equation*}
            AB=
            \begin{pmatrix}
                ac & ad\\
                bc & bd
            \end{pmatrix}.
        \end{equation*}
        Si $AB$ es invertible, entonces existe $C$ tal que
        \begin{equation*}
            (AB)C=I_{2}.
        \end{equation*}
        Pero observa que las columnas de $AB$ son
        \begin{equation*}
            \begin{pmatrix}
                ac\\
                bc
            \end{pmatrix}
            =c
            \begin{pmatrix}
                a\\
                b
            \end{pmatrix},
            \qquad
            \begin{pmatrix}
                ad\\
                bd
            \end{pmatrix}
            =d
            \begin{pmatrix}                
                a\\
                b
            \end{pmatrix}.
        \end{equation*}
        Es decir, ambas columnas son multiplos de la misma matriz
        $\displaystyle\begin{pmatrix}
                a\\
                b  
        \end{pmatrix}$
        Por lo tanto, no pueden coincidir on las columnas de la identidad
        \begin{equation*}
            I_{2}=
            \begin{pmatrix}
                1 & 0\\
                0 & 1
            \end{pmatrix}.
        \end{equation*}
        cuyas columnas
        \begin{equation*}
            \begin{pmatrix}
                1\\
                0
            \end{pmatrix},
            \qquad
            \begin{pmatrix}
                0\\
                1
            \end{pmatrix}
        \end{equation*}
        no son multiplos entre si. Esto nos da una contradicción, por lo tanto $AB$ no es invertible.  
    \end{proof}

    \begin{mdframed}[style=mdbluebox,frametitle={Ejercicio 3}]
        Determina si la matriz
        \begin{equation*}
            A=
            \left(
                \begin{array}{cccc}
                    1 & 2 & 3 & 4\\
                    0 & 1 & 3 & 4\\
                    0 & 0 & 1 & 4\\
                    0 & 0 & 0 & 1
                \end{array}
            \right),
        \end{equation*}
        es invertible.
    \end{mdframed}

    \begin{proof}
        Sea
        \begin{equation*}
            A=
            \left(
                \begin{array}{cccc}
                    1 & 2 & 3 & 4\\
                    0 & 1 & 3 & 4\\
                    0 & 0 & 1 & 4\\
                    0 & 0 & 0 & 1
                \end{array}
            \right),
        \end{equation*}
        con $A\in M_{4\times4}(F)$. Al aplicar operaciones elementales por filas a $A$,
        se obtiene
        \begin{align*}
            I_{4}=
            \left(
                \begin{array}{cccc}
                    1 & 0 & 0 & 0\\
                    0 & 1 & 0 & 0\\
                    0 & 0 & 1 & 0\\
                    0 & 0 & 0 & 1
                \end{array}
            \right).
        \end{align*}
        Dado que si toda matriz es equivalente a la identidad entonces esta matriz es invertible, se sigue inmediantamente que
        $A$ es invertible.
    \end{proof}

    \begin{mdframed}[style=mdbluebox,frametitle={Ejercicio 4}]
        Sea $A$ una matriz cuadrada $n\times n$ con coeficientes en $F$. Demuestra que si $A$
        no es invertible, entonces el sistema lineal homogéneo, $AX=0$ tiene una infinidad de
        soluciones.
    \end{mdframed}

    \begin{proof}
        Procederemos por contrapositiva. Si $A$ no es invertible entonces su forma escalonada reducida
        no puede ser la identidad, se sigue que existe al menos una columna sin pivote, luego en el sistema
        asociado existe por lo menos una variable libre, por tanto el sistema tiene infinitas soluciones.
    \end{proof}

    \begin{mdframed}[style=mdbluebox,frametitle={Ejercicio 5}]
        Sea $A\in M_{n\times n}(F)$. Demuestra que si $A$ tiene inversa derecho o inversa
        izquierda, entonces $A$ es invertible.
    \end{mdframed}

    \begin{proof}
        Supongamos que existe $B$ tal que
        \begin{equation*}
            AB=I_{n}.
        \end{equation*}
        Tambien supongamos que existe $C$ tal que
        \begin{equation*}
            CA=I_{n}.
        \end{equation*}
        Multiplicando la primera igualdad por $C$ a la izquierda, se obtiene
        \begin{equation*}
            CAB=CI_{n}.
        \end{equation*}
        Por la asociatividad del producto de matrices y por definicion de la identidad,
        \begin{equation*}
            (CA)B=C.
        \end{equation*}
        Luego,
        \begin{equation*}
            I_{n}B=C.
        \end{equation*}
        Nuevamente por definición de la identidad,
        \begin{equation*}
            B=C.
        \end{equation*}
        Asi, como la inversa por la izquierda y por la derecha coinciden, se concluye que $A$ es invertible.
    \end{proof}

    \begin{mdframed}[style=mdbluebox,frametitle={Ejercicio 6}]
        Sea $A$ una matriz $n\times n$. Demuestra que si $A$ es invertible y $AB=0_{n\times n}$
        para alguna matriz $B$ de tamaño $n\times n$, entonces $B=0_{n\times n}$.
    \end{mdframed}

    \begin{proof}
        Como $A$ es invertible, existe $A^{-1}\in M_{n\times n}(F)$. Multiplicando la igualdad
        $AB=0_{n\times n}$ por $A^{-1}$ a la izquierda, se obtiene
        \begin{equation*}
            A^{-1}AB=A^{-1}0_{n\times n}.
        \end{equation*}
        Por la asociatividad del producto de matrices,
        \begin{equation*}
            (A^{-1}A)B=0_{n\times n}.
        \end{equation*}
        Luego,
        \begin{equation*}
            I_{n}B=0_{n\times n},
        \end{equation*}
        y por definición de la matriz identidad,
        \begin{equation*}
            B=0_{n\times n}.
        \end{equation*}
    \end{proof}

    \begin{mdframed}[style=mdbluebox,frametitle={Ejercicio 7}]
        Sean
        \begin{equation*}
            A_{1}=
            \left(
                \begin{array}{ccc}
                    2 & 0 & 0\\
                    1 & -3 & 1\\
                    0 & 1 & 3
                \end{array}
            \right),
            \qquad
            A_{2}=
            \left(
                \begin{array}{cccc}
                    1 & 2 & 1 & 0\\
                    1 & 0 & 3 & 5\\
                    1 & 0 & 1 & 1
                \end{array}
            \right)
        \end{equation*}
        matrices con coeficientes en $\R$. Encuentra las matrices reducidas por filas y escalonadas
        $R_{1}$, $R_{2}$ equivalentes a $A_{1}$ y $A_{2}$, respectivamente. Encuentra las matrices
        invertibles $P_{1}$, $P_{2}$ tales que $R_{1}=P_{1}A_{1}$ y $R_{2}=P_{2}A_{2}$
    \end{mdframed}

    \begin{solucion}
        Sea
        \begin{equation*}
            A_{1}=
            \left(
                \begin{array}{ccc}
                    2 & 0 & 0\\
                    1 & -3 & 1\\
                    0 & 1 & 3
                \end{array}
            \right),
        \end{equation*}
        Aplicamos las siguientes operaciones elementales
        \begin{align*}
            F_{1}/(2)\to F_{1},\quad F_{2}-1\cdot F_{1}\to F_{2}, \quad F_{2}/(-3)\to F_{2},\\
            F_{3}-1\cdot F_{2}\to F_{3}, \quad F_{3}/(\frac{10}{3})\to F_{3}, \quad F_{2}-(-\frac{1}{3})\cdot F_{3}\to F_{2}.
        \end{align*}
        Entonces obtenemos
        \begin{equation*}
            R_{1}=
            \left(
                \begin{array}{ccc}
                    1 & 0 & 0\\
                    0 & 1 & 0\\
                    0 & 0 & 1
                \end{array}
            \right),
        \end{equation*}
        Las operaciones elementales que efectuamos corresponden a las siguientes matrices elementales
        \begin{align*}
            E_1=
            \begin{pmatrix}
                \frac12 & 0 & 0\\
                0 & 1 & 0\\
                0 & 0 & 1
            \end{pmatrix}
            \quad
            E_2=
            \begin{pmatrix}
                1 & 0 & 0\\
                -1 & 1 & 0\\
                0 & 0 & 1
            \end{pmatrix}
            \quad
            E_3=
            \begin{pmatrix}
                1 & 0 & 0\\
                0 & -\frac13 & 0\\
                0 & 0 & 1
            \end{pmatrix}
            \\
            E_4=
            \begin{pmatrix}
                1 & 0 & 0\\
                0 & 1 & 0\\
                0 & -1 & 1
            \end{pmatrix}
            \quad
            E_5=
            \begin{pmatrix}
                1 & 0 & 0\\
                0 & 1 & 0\\
                0 & 0 & \frac{3}{10}
            \end{pmatrix}
            \quad
            E_6=
            \begin{pmatrix}
                1 & 0 & 0\\
                0 & 1 & \frac13\\
                0 & 0 & 1
            \end{pmatrix}
        \end{align*}
        Por definición de matriz elemental,
        \begin{align*}
            E_{6}E_5E_4E_3E_2E_1A_{1}=R_{1}.
        \end{align*}
        Definimos
        \begin{equation*}
            P_{1}=E_{6}E_5E_4E_3E_2E_1,
        \end{equation*}
        Como las matrices elementales son invertibles, $P_{1}$ es invertible. Calculando el producto, obtenemos
        \begin{equation*}
            P_{1}=
            \begin{pmatrix}
                \frac12 & 0 & 0\\
                \frac{3}{20} & -\frac{3}{10} & \frac{1}{10}\\
                -\frac{1}{20} & \frac{1}{10} & \frac{3}{10}
            \end{pmatrix}.
        \end{equation*}
        Sea
        \begin{equation*}
            A_{2}=
            \left(
                \begin{array}{cccc}
                    1 & 2 & 1 & 0\\
                    1 & 0 & 3 & 5\\
                    1 & 0 & 1 & 1
                \end{array}
            \right).
        \end{equation*}
        Aplicamos las siguientes operaciones elementales
        \begin{align*}
            F_{2}-1\cdot F_{1} \to F_{2}, \quad
            F_{3}-1\cdot F_{1} \to F_{3}, \quad
            F_{2}/(-2) \to F_{2}, \quad
            F_{3}-(-2)\cdot F_{2} \to F_{3}, \\
            F_{3}/(-2) \to F_{3}, \quad
            F_{2}-\left(-\frac{1}{2}\right)\cdot F_{3} \to F_{2}, \quad
            F_{1}-1\cdot F_{3} \to F_{1}, \quad
            F_{1}-2\cdot F_{2} \to F_{1}.
        \end{align*}
        Entonces obtenemos
        \begin{equation*}
            R_{2}=
            \left(
                \begin{array}{cccc}
                    1 & 0 & 0 & -1\\
                    0 & 1 & 0 & -\frac{1}{2}\\
                    0 & 0 & 1 & 2
                \end{array}
            \right).
        \end{equation*}
        Las operaciones elementales que efectuamos corresponden a las siguientes matrices elementales
        \begin{align*}
            E_{1}=
            \begin{pmatrix}
                1 & 0 & 0\\
                -1 & 1 & 0\\
                0 & 0 & 1
            \end{pmatrix}
            \quad
            E_{2}=
            \begin{pmatrix}
                1 & 0 & 0\\
                0 & 1 & 0\\
                -1 & 0 & 1
            \end{pmatrix}
            \quad
            E_{3}=
            \begin{pmatrix}
                1 & 0 & 0\\
                0 & -\frac12 & 0\\
                0 & 0 & 1
            \end{pmatrix}
            \quad
            E_{4}=
            \begin{pmatrix}
                1 & 0 & 0\\
                0 & 1 & 0\\
                0 & 2 & 1
            \end{pmatrix}
            \\
            E_{5}=
            \begin{pmatrix}
                1 & 0 & 0\\
                0 & 1 & 0\\
                0 & 0 & -\frac12
            \end{pmatrix}
            \quad
            E_{6}=
            \begin{pmatrix}
                1 & 0 & 0\\
                0 & 1 & \frac12\\
                0 & 0 & 1
            \end{pmatrix}
            \quad
            E_{7}=
            \begin{pmatrix}
                1 & 0 & -1\\
                0 & 1 & 0\\
                0 & 0 & 1
            \end{pmatrix}
            \quad
            E_{8}=
            \begin{pmatrix}
                1 & -2 & 0\\
                0 & 1 & 0\\
                0 & 0 & 1
            \end{pmatrix}
        \end{align*}
        Por definición de matriz elemental,
        \begin{align*}
            E_{8}E_{7}E_6E_5E_4E_3E_2E_1A_{2}=R_{2}.
        \end{align*}
        Definimos
        \begin{equation*}
            P_{2}=E_{8}E_{7}E_6E_5E_4E_3E_2E_1,
        \end{equation*}
        Como las matrices elementales son invertibles, $P_{2}$ es invertible. Calculando el producto, obtenemos
        \begin{equation*}
            P_{2}=
            \begin{pmatrix}
                1 & -2 & -1\\
                \frac12 & -1 & \frac{1}{2}\\
                -1 & 2 & -\frac12
            \end{pmatrix}.
        \end{equation*}
    \end{solucion}
\end{document}
