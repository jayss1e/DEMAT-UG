\documentclass[12pt,a4paper]{article}
\usepackage{ugmath}
\usepackage{float}
\usepackage{placeins}
\usepackage[labelfont=bf,labelsep=space]{caption}
\newcommand{\alumno}{Ricardo León Martínez}
\newcommand{\materia}{Algebra Lineal I}
\newcommand{\profesor}{Claudia Reynoso Alcántara}
\newcommand{\tarea}{Tarea 2}
\newcommand{\fecha}{9/2/2026}

\begin{document}
    \begin{center}
    {\large\textbf{UNIVERSIDAD DE GUANAJUATO}}\\[0.3cm]
    {\normalsize\textbf{DIVISIÓN DE CIENCIAS NATURALES Y EXACTAS}}\\
    {\normalsize\textbf{CAMPUS GUANAJUATO}}\\[1cm]

    {\Large\textbf{\tarea\ (\materia)}}\\[1cm]
    \end{center}

    \textbf{Nombre:} \alumno \hfill 
    \textbf{Fecha:} \fecha \hfill 
    \textbf{Calificación:} \rule{3cm}{0.4pt} \\[0.3cm]

    En todos los problemas $F$ es un campo ($\mathbb{Q},\mathbb{R},\mathbb{C}$).

    \begin{mdframed}[style=mdbluebox,frametitle={Ejercicio 1}]
        Demuestra que el intercambio de dos filas en una matris $m\times n$ con coeficientes en
        $F$, puede hacerse mediante un número finito de operaciones elementales fila de los
        otros dos tipos.
    \end{mdframed}

    \begin{proof}
        Consideremos dos filas distintas, digamos la fila $i$ y la fila $j$. Denotemos estas
        filas por
        \begin{align*}
            R_{i},\quad R_{j}.
        \end{align*}
        Reemplazamos la fila $R_{i}$ por
        \begin{align*}
            R_{i}\to R_{i}+R_{j}.
        \end{align*}
        Ahora las filas son
        \begin{align*}
            R_{i}=R_{i}+R_{j},\qquad R_{j}=R_{j}.
        \end{align*}
        Reemplazamos la fila $R_{j}$ por
        \begin{align*}
            R_{j}\to R_{i}-R_{j}.
        \end{align*}
        Sustituyendo el valor actual de $R_{i}$, obtenemos
        \begin{align*}
            R_{j}=(R_{i}+R_{j})-R_{j}=R_{i}.
        \end{align*}
        En este punto
        \begin{align*}
            R_{i}=R_{i}+R_{j},\qquad R_{j}=R_{i}.
        \end{align*}
        Reemplazando la fila $R_{i}$ por
        \begin{align*}
            R_{i}\to R_{i}-R_{j}.
        \end{align*}
        Sustituyendo,
        \begin{align*}
            R_{i}=(R_{i}+R_{j})-R_{i}=R_{j}.
        \end{align*}
        Después de estas tres operaciones elementales, las filas quedan
        \begin{align*}
            R_{i}=R_{j}, \qquad R_{j}=R_{i},
        \end{align*}
        es decir, las filas $i$ y $j$ han sido intercambiadas
    \end{proof}

    \begin{mdframed}[style=mdbluebox,frametitle={Ejercicio 2}]
        Sea $(A\text{\textbar}B)\in M_{m\times(n+1)}(F)$ una matriz aumentada correspondiente a un sistema
        de $m$ ecuaciones lineales en $n$ variables. Demuestra que si $(A_{1}\text{\textbar}B_{1})$
        es una matriz equivalente por filas a $(A\text{\textbar}B)$ entonces toda solución del sistema
        correspondiente a $(A\text{\textbar}B)$ es solución del sistema correspondiente a $(A_{1}\text{\textbar}B_{1})$.
    \end{mdframed}

    \begin{proof}
        Como $A_{1}\text{\textbar}B_{1}$ es equivalente por filas a $(A\text{\textbar}B)$, los sistemas
        de ecuaciones lineales asociados a estas matrices son equivalentes. Por lo visto en clase
        sabemos que si dos sistemas de ecuaciones lineales son equivalentes se sigue inmediatamente que
        ambos sistemas poseen exactamente el mismo conjunto solución. En particular toda solución
        del sistema asociado a $A_{1}\text{\textbar}B_{1}$ es tambien solución del sistema asociado a
        $(A\text{\textbar}B)$.
    \end{proof}

    \begin{mdframed}[style=mdbluebox,frametitle={Ejercicio 3}]
        Encuentra todas las soluciones que el sistema:
        \begin{align*}
            \begin{cases}
                2x_{1}-3x_{2}-7x_{3}+x_{4}+2x_{5}=-2\\
                x_{1}-2x_{2}-4x_{3}+3x_{4}+x_{5}=-1\\
                2x_{1}-4x_{3}+3x_{4}+x_{5}=3\\
                x_{3}+6x_{4}+2x_{5}=0.
            \end{cases}
        \end{align*}
        tiene en $\mathbb{Q}^{5}$.
    \end{mdframed}

    \begin{solucion}
        Consideremos el siguiente sistema de ecuaciones lineales
        \begin{align*}
            \begin{cases}
                2x_{1}-3x_{2}-7x_{3}+x_{4}+2x_{5}=-2\\
                x_{1}-2x_{2}-4x_{3}+3x_{4}+x_{5}=-1\\
                2x_{1}-4x_{3}+3x_{4}+x_{5}=3\\
                x_{3}+6x_{4}+2x_{5}=0.
            \end{cases}
        \end{align*}
        La matriz aumentada de $S$ es
        \begin{align*}
            \left(
                \begin{array}{ccccc|c}
                    2 & -3 & -7 & 1 & 2 & -2\\
                    1 & -2 & -4 & 3 & 1 & -1\\
                    2 & 0 & -4 & 3 & 1 & 3\\
                    0 & 0 & 1 & 6 & 2 & 0
                \end{array}
            \right).
        \end{align*}
        Multilicando la fila 1 por $\frac{1}{2}$ y restandola a la fila 2 tenemos
        \begin{align*}
            \left(
                \begin{array}{ccccc|c}
                    2 & -3 & -7 & 1 & 2 & -2\\
                    0 & -\frac{1}{2} & -\frac{1}{2} & \frac{5}{2} & 0 & 0\\
                    2 & 0 & -4 & 3 & 1 & 3\\
                    0 & 0 & 1 & 6 & 2 & 0
                \end{array}
            \right).
        \end{align*}
        Le restamos la fila 1 a la fila 3 obteniendo
        \begin{align*}
            \left(
                \begin{array}{ccccc|c}
                    2 & -3 & -7 & 1 & 2 & -2\\
                    0 & -\frac{1}{2} & -\frac{1}{2} & \frac{5}{2} & 0 & 0\\
                    0 & 3 & 3 & 2 & -1 & 5\\
                    0 & 0 & 1 & 6 & 2 & 0
                \end{array}
            \right).
        \end{align*}
        Multiplicamos la fila 2 por -6 y la restamos a la fila 6
        \begin{align*}
            \left(
                \begin{array}{ccccc|c}
                    2 & -3 & -7 & 1 & 2 & -2\\
                    0 & -\frac{1}{2} & -\frac{1}{2} & \frac{5}{2} & 0 & 0\\
                    0 & 0 & 0 & 17 & -1 & 5\\
                    0 & 0 & 1 & 6 & 2 & 0
                \end{array}
            \right).
        \end{align*}
        Por ultimo, intercambiamos las filas 4 y 3 obtenemos la matriz
        \begin{align*}
            \left(
                \begin{array}{ccccc|c}
                    2 & -3 & -7 & 1 & 2 & -2\\
                    0 & -\frac{1}{2} & -\frac{1}{2} & \frac{5}{2} & 0 & 0\\
                    0 & 0 & 1 & 6 & 2 & 0\\
                    0 & 0 & 0 & 17 & -1 & 5
                \end{array}
            \right)
        \end{align*}
        o bien el sistema de ecuaciones lineales
        \begin{align}
            \begin{cases}
                2x_{1}-3x_{2}-7x_{3}+x_{4}+2x_{5}=-2\\
                \frac{1}{2}x_{2}-\frac{1}{2}x_{3}+\frac{5}{2}x_{4}=0\\
                x_{3}+6x_{4}+2x_{5}=0\\
                17x_{4}-x_{5}=5
            \end{cases}
        \end{align}
        De la ecuación 4 del sistema (1) tenemos
        \begin{align}
            17x_{4}=5+x_{5} \implies x_{4}=\frac{5}{17}+\frac{1}{17}x_{5}.
        \end{align}
        De la ecuacion 3 del sistema (1) y de (2)
        \begin{align}
            x_{3}=-6x_{4}-2x_{5}=-6\left(\frac{5}{17}+\frac{1}{17}x_{5}\right)-2x_{5}=-\frac{30}{17}-\frac{40}{17}x_{5}.
        \end{align}
        De la ecuacion 2 del sistema (1) y de (3) y (2)
        \begin{align*}
            -\frac{1}{2}x_{2}=\frac{1}{2}x_{3}-\frac{5}{2}x_{4}=\frac{1}{2}\left(-\frac{30}{17}-\frac{40}{17}x_{5}\right)-\frac{5}{2}\left(\frac{5}{17}+\frac{1}{17}x_{5}\right)=-\frac{-55}{34}-\frac{45}{34}x_{5}\\
        \end{align*}
        luego,
        \begin{align}
            x_{2}=\frac{55}{17}-\frac{45}{17}x_{5}
        \end{align}
        De la ecuacion (1) y de (4),(3) y (2)
        \begin{align*}
            2x_{1}=-2+3x_{2}+7x_{3}-x_{4}-2x_{5}&=-2+3\left(\frac{55}{17}-\frac{45}{17}x_{5}\right)+7\left(-\frac{30}{17}-\frac{40}{17}x_{5}\right)-\left(\frac{5}{17}+\frac{1}{17}x_{5}\right)-2x_{5}\\
            &=-\frac{84}{17}-\frac{180}{17}x_{5}
        \end{align*}
        entonces,
        \begin{align*}
            x_{1}=-\frac{42}{17}-\frac{90}{17}x_{5}.
        \end{align*}
        Sea $x_{5}=t$, con $t\in\mathbb{Q}$. Entonces
        \begin{align*}
            (x_{1},x_{2},x_{3},x_{4},x_{5})=\left(-\frac{42}{17}-\frac{90}{17}t,\frac{55}{17}-\frac{45}{17}t,-\frac{30}{17}-\frac{40}{17}t,\frac{5}{17}+\frac{1}{17}t,t\right).
        \end{align*}
        Así, el conjunto solución es
        \begin{align*}
            S=\{\left(-\frac{42}{17}-\frac{90}{17}t,\frac{55}{17}-\frac{45}{17}t,-\frac{30}{17}-\frac{40}{17}t,\frac{5}{17}+\frac{1}{17}t,t\right):t\in\mathbb{Q}\}.
        \end{align*}
    \end{solucion}

    \begin{mdframed}[style=mdbluebox,frametitle={Ejercicio 4}]
        Sean $m,n\in\mathbb{N}$ y sea el $r$ el número de pivotes de una matriz reducida por filas de
        tamaño $m\times n$ con coeficientes $F$. Demuestra que si $r\lt n$, entonces el sistema lineal
        homogéneo asociado tiene una infinidad de soluciones.
    \end{mdframed}

    \begin{proof}
        Como $A$ está en forma reducida por filas y tiene $r$ pivotes, existen exactamente $r$
        columnas pivote. Dado que $r\lt n$, se sigue que no todas las columnas son columnas pivote.
        Por definición, toda columna que no es columna pivote corresponde a una variable libre del
        sistema homogéneo. En consecuencia, el sistema tiene al menos una variable libre.
        Sea $x_{j}$ una de estas variables libres. Como el sistema es homogéneo, podemos asignar a
        $x_{j}$ cualquier valor de $F$, y los valores de las demas variables quedan determinados por
        las ecuaciones del sistema. Como $F$ es inifinito, existen infinitos valores posibles para
        $x_{j}$, y por lo tanto el sistema posee infinitas soluciones distintas.
    \end{proof}


    \begin{mdframed}[style=mdbluebox,frametitle={Ejercicio 5}]
        Determina si las siguientes matrices definidas sobre el campo de los números reales:
        \begin{align*}
            \begin{aligned}
                \left(
                    \begin{array}{ccc}
                        2 & -1 & 0\\
                        1 & -2 & 1\\
                        1 & 3 & -1
                    \end{array}
                \right)
                \quad
                \left(
                    \begin{array}{ccc}
                        1 & 1 & 0\\
                        0 & 1 & 0\\
                        0 & 0 & 1
                    \end{array}
                \right),
            \end{aligned}
        \end{align*}
        son equivalentes por filas a la matriz
        \begin{align*}
            \left(
                \begin{array}{ccc}
                    1 & 0 & 0\\
                    0 & 1 & 0\\
                    0 & 0 & 1
                \end{array}
            \right).
        \end{align*}
        Si es así, describe las operaciones elementales de fila que aplicaste en cada caso.
    \end{mdframed}
    
    \begin{solucion}
        Consideremos la siguiente matriz
        \begin{align*}
            \left(
                \begin{array}{ccc}
                    2 & -1 & 0\\
                    1 & -2 & 1\\
                    1 & 3 & -1
                \end{array}
            \right).
        \end{align*}
        Dividimos la fila 1 por 2
        \begin{align*}
            \left(
                \begin{array}{ccc}
                    1 & -\frac{1}{2} & 0\\
                    1 & -2 & 1\\
                    1 & 3 & -1
                \end{array}
            \right).
        \end{align*}
        Restamos la fila 1 a la fila 2
        \begin{align*}
            \left(
                \begin{array}{ccc}
                    1 & -\frac{1}{2} & 0\\
                    0 & -\frac{3}{2} & 1\\
                    1 & 3 & -1
                \end{array}
            \right).
        \end{align*}
        Restamos la fila 1 a la fila 3
        \begin{align*}
            \left(
                \begin{array}{ccc}
                    1 & -\frac{1}{2} & 0\\
                    0 & -\frac{3}{2} & 1\\
                    0 & \frac{7}{2} & -1
                \end{array}
            \right).
        \end{align*}
        Dividimos la fila 2 por $-\frac{3}{2}$
        \begin{align*}
            \left(
                \begin{array}{ccc}
                    1 & -\frac{1}{2} & 0\\
                    0 & 1 & -\frac{2}{3}\\
                    0 & \frac{7}{2} & -1
                \end{array}
            \right).
        \end{align*}
        Multiplicando la fila 2 por $\frac{7}{2}$ y la restamos a la fila 3
        \begin{align*}
            \left(
                \begin{array}{ccc}
                    1 & -\frac{1}{2} & 0\\
                    0 & 1 & -\frac{2}{3}\\
                    0 & 0 & \frac{4}{3}
                \end{array}
            \right)
        \end{align*}
        Dividimos la fila 3 por $\frac{4}{3}$
        \begin{align*}
            \left(
                \begin{array}{ccc}
                    1 & -\frac{1}{2} & 0\\
                    0 & 1 & -\frac{2}{3}\\
                    0 & 0 & 1
                \end{array}
            \right).
        \end{align*}
        Multiplicamos la fila 3 por $-\frac{2}{3}$ y la restamos a la fila 2
        \begin{align*}
            \left(
                \begin{array}{ccc}
                    1 & -\frac{1}{2} & 0\\
                    0 & 1 & 0\\
                    0 & 0 & 1
                \end{array}
            \right).
        \end{align*}
        Finalmente, multiplicamos la fila 2 por $-\frac{1}{2}$ y la restamos a la fila 2
        \begin{align*}
            \left(
                \begin{array}{ccc}
                    1 & 0 & 0\\
                    0 & 1 & 0\\
                    0 & 0 & 1
                \end{array}
            \right).
        \end{align*}
        Ahora consideremos la matriz
        \begin{align*}
            \left(
                \begin{array}{ccc}
                    1 & 1 & 0\\
                    0 & 1 & 0\\
                    0 & 0 & 1
                \end{array}
            \right).
        \end{align*}
        Restamos la fila 2 a la fila 1
        \begin{align*}
            \left(
                \begin{array}{ccc}
                    1 & 0 & 0\\
                    0 & 1 & 0\\
                    0 & 0 & 1
                \end{array}
            \right).
        \end{align*}
        Así, las dos matrices son equivalentes por filas a la matriz identidad.
    \end{solucion}
\end{document}