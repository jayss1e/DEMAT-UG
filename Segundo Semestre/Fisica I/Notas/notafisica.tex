\documentclass[12pt,a4paper]{book}
\raggedbottom

\usepackage{ugmath}
\usepackage{float}
\usepackage{placeins}
\usepackage[labelfont=bf,labelsep=space]{caption}
\usepackage{pdfpages}


\begin{document}


    \thispagestyle{empty}
    \includepdf[pages=1]{Figura.pdf}
    

    \setcounter{page}{1}
    \tableofcontents
    \clearpage

    \chapter{Introducción}

    \section{¿Qué es la física?}

    La palabra física proviene del término griego $\phi\upsilon\sigma\iota\zeta$ que significa \textbf{naturaleza}, y por ello la física
    debía ser una ciencia dedicada al estudio de todos los fenómenos naturales. Especificamente, la física es una ciencia cuyo objetivo
    es estudiar los componentes de la materia y sus interacciones mutuas.

    \section{Partes clásicas de la física}

    Al principio los sentidos del ser humano fueron las fuentes de informacion de ellos clasificó los fenómenos observados de acuerdo a la manera
    en la que los percibía.\\
    La \textbf{luz} fue relacionada con la \textbf{visión} y la \textbf{óptica} se desarrolló como una ciencia más o menos independiente relacionada
    a ella. El \textbf{sonido} fue relacionado con la \textbf{audición} y la \textbf{acústica} se desarrolló como ciencia correlativa. El \textbf{calor}
    fue relacionado con otra sensación física, y por muchos años el estudio de calor (denominado \textbf{termodinámica}) fue otra parte autónoma
    de la física.

\end{document}