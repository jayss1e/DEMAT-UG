\documentclass[12pt,a4paper]{article}
\newcommand{\grupo}{}
% ------------------------------------------------------
% Idioma y codificación
% ------------------------------------------------------
\usepackage[spanish]{babel}
\usepackage[utf8]{inputenc}
\usepackage[T1]{fontenc}

% ------------------------------------------------------
% Tipografía y formato
% ------------------------------------------------------
\usepackage{lmodern}    % Fuente Latin Modern
\usepackage{microtype}  % Mejor espaciado
\usepackage{xcolor}     % Colores
\definecolor{UGblue}{RGB}{0,33,71}
\definecolor{UGgold}{RGB}{212,175,55}
\usepackage{graphicx}
\usepackage{setspace}
\usepackage{titlesec}
% Añade esto en el preámbulo (antes de \begin{document})
\usepackage{bookmark}
\bookmarksetup{open,numbered}
\usepackage{tikz}
\usetikzlibrary{calc,decorations.pathreplacing,angles,quotes}
\usepackage{pgfplots}
\pgfplotsset{compat=1.18}
\usepackage{caption}

% ------------------------------------------------------
% Matemáticas
% ------------------------------------------------------
\usepackage{amsmath,amssymb,amsthm,mathtools}
\usepackage{mathrsfs}
\usepackage{stmaryrd}
\usepackage{physics}

% ------------------------------------------------------
% Otros paquetes útiles
% ------------------------------------------------------
\usepackage{enumitem}
\usepackage{csquotes}
\usepackage{hyperref}
\hypersetup{
    colorlinks=true,
    linkcolor=UGblue,
    urlcolor=UGgold,
    citecolor=UGblue
}
\usepackage{geometry}
\geometry{margin=2.5cm}
\usepackage{multicol}

% ------------------------------------------------------
% Encabezados y pies de página
% ------------------------------------------------------
% ------------------------------------------------------
% Teoremas, definiciones, etc.
% ------------------------------------------------------
\theoremstyle{plain}
\newtheorem{teo}{Teorema}[section]
\newtheorem{prop}[teo]{Proposición}
\newtheorem{lema}[teo]{Lema}
\theoremstyle{definition}
\newtheorem{defi}[teo]{Definición}
\theoremstyle{remark}
\newtheorem{ejemplo}[teo]{Ejemplo}
\newtheorem{obs}[teo]{Observación}

% ------------------------------------------------------
% Comandos útiles
% ------------------------------------------------------
\newcommand{\N}{\mathbb{N}}
\newcommand{\Z}{\mathbb{Z}}
\newcommand{\Q}{\mathbb{Q}}
\newcommand{\R}{\mathbb{R}}
\newcommand{\C}{\mathbb{C}}
\newcommand{\pfrac}[2]{\left( \frac{#1}{#2} \right)} %parentesis
\newcommand{\bfrac}[2]{\left[ \frac{#1}{#2} \right]} %corchetes
\newcommand{\vecpar}[1]{\left( #1 \right)}
% En el preámbulo
\newcommand{\point}[2]{\left( #1, #2 \right)}
\newcommand{\pointfrac}[4]{\left( \frac{#1}{#2}, \frac{#3}{#4} \right)}
\newcommand{\pointmix}[4]{\left( #1, \frac{#3}{#4} \right)} % Primera coordenada normal, segunda fracción
\renewcommand{\qedsymbol}{$\blacksquare$}
\newcommand{\gt}{\ensuremath{>}}
\newcommand{\lt}{\ensuremath{<}}
\renewcommand{\div}{$÷$}

\renewcommand{\baselinestretch}{1.2}
\setlength{\parindent}{0pt}

\newenvironment{solucion}
{\par\noindent\textit{Solución.}\ }
{\hfill$\blacktriangleleft$\par}

% ------------------------------------------------------
% Datos personales (modificar según tarea)
% ------------------------------------------------------
\newcommand{\alumno}{Ricardo León Martínez}
\newcommand{\materia}{Elementos de Geometría}
\newcommand{\profesor}{Valentina Muñoz Porras}
\newcommand{\tarea}{Tarea 5}
\newcommand{\fecha}{22/09/2025}

\setlength{\parindent}{0pt}

\begin{document}
    \textbf{Ejercicio 1, Inciso a} 
    \begin{proof}
        Para $m\in\mathbb{N}_{0}$, consideremos el siguiente conjunto
        \begin{align*}
            S=\{n\in\mathbb{N}_{0}:m+n=n+m\}.
        \end{align*}
        Probemos que $S\subseteq\mathbb{N}_{0}$ es un conjunto de sucesores.
        Primero notemos que, por definicion de la suma se tiene
        \begin{align*}
            m+0=m=0+m,
        \end{align*}
        y por lo tanto $0\in S$. Supongamos ahora que $n\in S$, esto significa que
        \begin{align*}
            m+n=n+m.
        \end{align*}
        Por definicion de la suma se tiene que
        \begin{align*}
            S_{m}(n+)=s(m+n) \quad \text{ y } \quad S_{n+}(m)=s(n+m).
        \end{align*}
        Como $m+n=n+m$, se obtiene
        \begin{align*}
            S_{m}(n+)=s(m+n)=s(n+m)=S_{n+}(m),
        \end{align*}
        por lo que, $n+\in S$. Asi, $S$ es un conjunto de sucesores que contiene al
        0. Por el teorema 4.1, se sigue que $S=\mathbb{N}_{0}$. En consecuencia, para
        todo $n\in\mathbb{N}_{0}$ se cumple
        \begin{align*}
            m+n=n+m.
        \end{align*}
    \end{proof}
    \textbf{Ejercicio 1, Inciso b} 
    \begin{proof}
        Para $n,m\in\mathbb{N}_{0}$, definimos el conjunto
        \begin{align*}
            S=\{k\in\mathbb{N}_{0}:n+(m+k)=(n+m)+k\}.
        \end{align*}
        Probemos que $S\subseteq\mathbb{N}_{0}$ es un conjunto de sucesores. Por
        definición de la suma y lo probado en el inciso anterior,
        \begin{align*}
            m+0=m, \qquad n+m=n+m,
        \end{align*}
        y ademas
        \begin{align*}
            n+(m+0)=n+m=(n+m)+0.
        \end{align*}
        Por lo tanto, $0\in S$. Sea ahora $k\in S$ esto es,
        \begin{align}
            n+(m+k)=(n+m)+k.
        \end{align}
        Consideremos $k+$. Usando la definición de la suma, se tiene
        \begin{align*}
            S_{m}(k+)=s(m+k),\qquad S_{n}(S_{m}(k+))=n+s(m+k),
        \end{align*}
        y también
        \begin{align*}
            S_{n+m}(k+)=s((n+m)+k)
        \end{align*}
        De (1), obtenemos
        \begin{align*}
            n+s(m+k)=s(n+(m+k))=s((n+m)+k).
        \end{align*}
        Se sigue que
        \begin{align*}
            n+(m+k+)=(n+m)+k+,
        \end{align*}
        lo cual prueba que $k+\in S$. Así, $S$ es un conjunto de sucesores que
        contiene al 0. Por el teorema 4.1, se concluye que $S=\mathbb{N}_{0}$.
        Por lo tanto, para todo $k\in\mathbb{N}_{0}$ se cumple
        \begin{align*}
            n+(m+k)=(n+m)+k.
        \end{align*}
    \end{proof}

    \vspace{1cm}

    \textbf{Ejercicio 2, Inciso a} 
    \begin{proof}
        Primero probemos que $1\cdot n=n$. Consideremos el conjunto
        \begin{align*}
            S=\{n\in\mathbb{N}_{0}:1\cdot n=n\}.
        \end{align*}
        Por definición del producto,
        \begin{align*}
            1\cdot0=0,
        \end{align*}
        de modo que $0\in S$. Supongamos $n\in S$ esto es,
        \begin{align*}
            1\cdot n=n.
        \end{align*}
        Entonces, por la definición del producto y de la suma,
        \begin{align*}
            1\cdot s(n)=1\cdot n+1=n+1=s(n).
        \end{align*}
        Así, $s(n)\in S$. Por lo tanto, $S$ es un conjunto de sucesores que contiene
        al 0. Por el teorema 4.1, se sigue que $S=\mathbb{N}_{0}$, y por ende
        \begin{align*}
            1\cdot n=n\text{ para todo }n\in\mathbb{N}_{0}.
        \end{align*}
        Ahora probemos que $n\cdot1=n$. como $1=s(0)$, por la definición del producto,
        \begin{align*}
            n\cdot1=n\cdot s(0)=n\cdot 0+n.
        \end{align*}
        Pero $n\cdot0=0$ y $0+n=n$, de modo que
        \begin{align*}
            n\cdot1=0+n=n.
        \end{align*} 
        De donde se sigue el resultado.
    \end{proof}

    \textbf{Ejercicio 2, Inciso b}
    \begin{proof}
        Recordemos que $n+1=n+$ por el teorema 4.3. Usando esta igualdad y la 
        definición del producto, obtenemos
        \begin{align*}
            m(n+1)=m(n+)=P_{m}(n+).
        \end{align*}
        Por como se define a $P_{m}$, se tiene
        \begin{align*}
            P_{m}(n+)=S_{m}(P_{m}(n)),
        \end{align*}
        y por la definición de $S_{m}$,
        \begin{align*}
            S_{m}(P_{m}(n))=m+P_{m}(n).
        \end{align*}
        Como $P_{m}(n)=mn$, entonces
        \begin{align*}
            m(n+1)=m+mn.
        \end{align*}
        Dado que la suma es conmutativa, se sigue le resultado.
    \end{proof}

    \vspace{1cm}
    
    \textbf{Ejercicio 3, Inciso a}
    \begin{proof}
        Sean
        \begin{align*}
            a=[(a_1,a_2)],\qquad b=[(b_1,b_2)],\qquad c=[(c_1,c_2)]
        \end{align*}
        sus representaciones en $\mathbb{Z}$. Por definición de la suma en $\mathbb{Z}$,
        \begin{align*}
            a+b=[(a_1+b_1,a_2+b_2)].
        \end{align*}
        Entonces,
        \begin{align*}
            (a+b)+c=[(a_1+b_1)+c_1,(a_2+b_2)+c_2)].
        \end{align*}
        Por otra parte,
        \begin{align*}
            b+c=[(b_1+c_1,b_2+c_2)],
        \end{align*}
        y por la misma definición
        \begin{align*}
            a+(b+c)=[(a_1+(b_1+c_1),a_2+(b_2+c_2))].
        \end{align*}
        Así,
        \begin{align*}
            (a+b)+c=a+(b+c).
        \end{align*}
    \end{proof}

    \textbf{Ejercicio 3, Inciso b}
    \begin{proof}
        Sea $a=[(a_1,a_2)]$ su representación en $\mathbb{Z}$. El entero 0
        corresponde a la clase $[(0,0)]$. Por definición de la suma en $\mathbb{Z}$,
        \begin{align*}
            a+0=[(a_1,a_2)]+[(0,0)]=[(a_1+0,a_2+0)]=[(a_1,a_2)].
        \end{align*}
        Por lo tanto,
        \begin{align*}
            a+0=a.
        \end{align*}
    \end{proof}

    \textbf{Ejercicio 3, Inciso c}
    \begin{proof}
        \setcounter{equation}{0}
        Sean
        \begin{align*}
            a=[(a_1,a_2)],\qquad b=[(b_1,b_2)]\qquad c=[(c_1,c_2)]
        \end{align*}
        sus representaciones en $\mathbb{Z}$. Por definición de la suma en $\mathbb{Z}$,
        \begin{align*}
            a+b=[(a_1+b_1,a_2+b_2)],\qquad c+b=[(c_1+b_1,c_2+b_2)].
        \end{align*}
        Usando el criterio de igualdad en $\mathbb{Z}$,
        \begin{align*}
            [(x,y)]=[(u,v)] \Longleftrightarrow x+v=y+u,
        \end{align*}
        se tiene que
        \begin{align*}
            a+b=c+b
        \end{align*}
        si y solo si
        \begin{align}
            (a_1+b_1)+(c_2+b_2)=(a_2+b_2)+(c_1+b_1).
        \end{align}
        Aplicando asociatividad,
        \begin{align*}
            (a_1+b_1)+(c_2+b_2)=a_1+(b_1+(c_2+b_2)),\\
            (a_2+b_2)+(c_1+b_1)=a_2+(b_2+(c_1+b_1)).
        \end{align*}
        Aplicando conmutatividad,
        \begin{align*}
            b_1+(c_2+b_2)=c_2+(b_1+b_2),\\
            b_2+(c_1+b_1)=c_1+(b_2+b_1)=c_1+(b_1+b_2).
        \end{align*}
        sustituyendo esto en $(1)$, obtenemos
        \begin{align}
            a_1+(c_2+(b_1+b_2))=a_2+(c_1+(b_1+b_2)).
        \end{align}
        Por asociatividad nuevamente,
        \begin{align}
            a_1+c_2+(b_1+b_2)=a_2+c_1+(b_1+b_2).
        \end{align}
        Por el ejercicio 1 inciso d de esta tarea sabemos que
        \begin{align*}
            x+u=y+u \Longleftrightarrow x=y.
        \end{align*}
        Aplicando esto a $(3)$ con $u=b_1+b_2,x=a_1+c_1$ y $y=a_2+c_1$ se obtiene
        \begin{align*}
            a_1+c_2=a_2+c_1.
        \end{align*}
        Por definición de igualdad en $\mathbb{Z}$,
        \begin{align*}
            [(a_1,a_2)]=[(c_1,c_2)],
        \end{align*}
        es decir
        \begin{align*}
            a=c.
        \end{align*}
        La suficiencia es inmediata sustituyendo $a=c$ en la definición de la suma,
        concluyendo así el resultado.
    \end{proof}

    \vspace{1.4cm}

    \textbf{Ejercicio 4, Inciso a}
    \begin{proof}
        Sean $[(a,b)]_q$ y $[(c,d)]_q$ elementos de $\mathbb{Q}$ con $b\neq0$
        y $d\neq0$. Por definición de la suma en $\mathbb{Q}$,
        \begin{align*}
            [(a,b)]_q+[(c,d)]_q=[(ad+bc,bc)]_q,
        \end{align*}
        mientras que
        \begin{align*}
            [(c,d)]_q+[(a,b)]_q=[(cb+da,db)]_q.
        \end{align*}
        Como en $\mathbb{Z}$ la suma y el producto son conmutativos, se tiene
        $ad+bc=cb+da$ y $bd=db$. Entonces, por la proposición 4.12, ambas expresiones
        representan la misma clase, de modo que
        \begin{align*}
            [(a,b)]_q+[(c,d)]_q=[(c,d)]_q+[(a,b)]_q.
        \end{align*}
        Para el producto, de la definición se obtiene
        \begin{align*}
            [(a,b)]_q \cdot [(c,d)]_q=[(ac,bd)]_q,\qquad [(c,d)]_q \cdot [(a,b)]_q=[(ca,db)]_q.
        \end{align*}
        Dado que en $\mathbb{Z}$ se cumple $ac=ca$ y $bd=db$, estas dos parejas determinan la misma
        clase en $\mathbb{Q}$. Por lo tanto,
        \begin{align*}
            [(a,b)]_q\cdot[(c,d)]_q=[(c,d)]_q\cdot[(a,b)]_q.
        \end{align*}
        Así, las dos operaciones son conmutativas.
    \end{proof}

    \vspace{0.5cm}

    \textbf{Ejercicio 4, Inciso b}
    \begin{proof}
        Sean $[(a,b)]_q,[(c,d)]_q,[(e,f)]_q\in\mathbb{Q}$ con $b,d,f\neq0$.
        Por definición de la suma en $\mathbb{Q}$,
        \begin{align*}
            \big([(a,b)]_q+[(c,d)]_q\big)+[(e,f)]_q=[(ad+bc,bd)]_q+[(e,f)]_q=\big((ad+bc)f+bde,bdf\big)_q,
        \end{align*}
        mientras que
        \begin{align*}
            [(a,b)]_q+\big([(c,d)]_q+[(e,f)]_q\big)=[(a,b)]_q+[(cf+de,df)]_q=\big(a(df)+b(cf+de),bdf\big)_q.
        \end{align*}
        Por asociatividad y conmutatividad de la suma y el producto en $\mathbb{Z}$ se tiene
        \begin{align*}
            (ad+bc)f+bde=adf+bcf+bde=a(df)+b(cf+de),
        \end{align*}
        y ademas $(bd)f=b(df)$. Por la proposición 4.12, las parejas anteriores representan
        la misma clase, por lo que
        \begin{align*}
            \big([(a,b)]_q+[(c,d)]_q\big)+[(e,f)]_q=[(a,b)]_q+\big([(c,d)]_q+[(e,f)]_q\big).
        \end{align*}
        Por la definición del producto en $\mathbb{Q}$,
        \begin{align*}
            \big([(a,b)]_q\cdot[(c,d)]_q\big)\cdot[(e,f)]_q=[(ac,bd)]_q\cdot[(e,f)]_q=\big((ac)e,(bd)f\big)_q,
        \end{align*}
        y
        \begin{align*}
            [(a,b)]_q\cdot\big([(c,d)]_q\cdot[(e,f)]_q\big)=[(a,b)]_q\cdot[(cd,df)]_q=\big(a(ce),b(df)\big)_q.
        \end{align*}
        Por la asociatividad del producto en $\mathbb{Z}$ se cumple $(ac)e=a(ce)$ y
        $(bd)f=b(df)$, luego las dos parejas son la misma clase y
        \begin{align*}
            \big([(a,b)]_q\cdot[(c,d)]_q\big)\cdot[(e,f)]_q=[(a,b)]_q\cdot\big([(c,d)]_q\cdot[(e,f)]_q\big)
        \end{align*}
        Por lo tanto ambas operaciones son asociativas.
    \end{proof}

    \vspace{0.5cm}

    \textbf{Ejercicio 4, Inciso c}
    \begin{proof}
        Sean $[(a,b)]_q\in\mathbb{Q}$ con $b\neq0$. Consideremos las clases
        $[(0,1)]_q$ y $[(1,1)]_q$. Por definición de la suma en $\mathbb{Q}$,
        \begin{align*}
            [(a,b)]+[(0,1)]_q=[(a\cdot1+b\cdot0,b\cdot1)]_q=[(a,b)]_q.
        \end{align*}
        Por lo tanto $[(0,1)]_q$ es neutro aditivo. Por la definición del 
        producto en $\mathbb{Q}$,
        \begin{align*}
            [(a,b)]_q\cdot[(1,1)]_q=[(a\cdot1,b\cdot1)]_q=[(a,b)]_q.
        \end{align*}
        Así, $[(1,1)]_q$ es neutro multiplicativo.
    \end{proof}

    \vspace{0.5cm}

    \textbf{Ejercicio 4, Inciso d}
    \begin{proof}
        Sea $e\in\mathbb{Q}$ un neutro aditivo, para todo $a\in\mathbb{Q}$ se cumple
        \begin{align*}
            a+e=a.
        \end{align*}
        Sea $e^{\prime}\in\mathbb{Q}$ otro neutro aditivo, de modo que también
        \begin{align*}
            a+e^{\prime}=a
        \end{align*}
        para todo $a\in\mathbb{Q}$. Tomando $a=e^{\prime}$ en la igualdad que define a $e$, obtenemos
        \begin{align*}
            e^{\prime}+e=e^{\prime}.
        \end{align*}
        Tomando ahora $a=e$ en al igualdad que define a $e^{\prime}$, obtenemos
        \begin{align*}
            e+e^{\prime}=e.
        \end{align*}
        Como la suma en conmutativa en $\mathbb{Q}$, $e+e^{\prime}=e^{\prime}+e$.
        Sustituyendo en las igualdades anteriores se obtiene
        \begin{align*}
            e=e^{\prime}.
        \end{align*}
        Por lo tanto el neutro aditivo es único. Sea ahora $u\in\mathbb{Q}$ un neutro
        multiplicativo, para todo $a\in\mathbb{Q}$,
        \begin{align*}
            a\cdot u=a.
        \end{align*}
        Sea $u^{\prime}\in\mathbb{Q}$ otro neutro multiplicativo, con
        \begin{align*}
            a\cdot u^{\prime}=a
        \end{align*}
        para todo $a\in\mathbb{Q}$. Tomando $a=u^{\prime}$ en la igualdad para $u$, se obtiene
        \begin{align*}
            u^{\prime}\cdot u=u^{\prime}.
        \end{align*}
        Tomando $a=u$ en la igualdad para $u^{\prime}$,
        \begin{align*}
            u\cdot u^{\prime}=u.
        \end{align*}
        Como el producto en $\mathbb{Q}$ es conmutativo, $u\cdot u^{\prime}=u^{\prime}\cdot u$.
        Sustituyendo en las igualdades anteriores se obtiene
        \begin{align*}
            u=u^{\prime}.
        \end{align*}
        Asi, el neutro multiplicativo es único.
    \end{proof}
\end{document}