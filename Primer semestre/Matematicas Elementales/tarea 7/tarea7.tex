\documentclass[12pt,a4paper]{article}
\newcommand{\grupo}{}
% ------------------------------------------------------
% Idioma y codificación
% ------------------------------------------------------
\usepackage[spanish]{babel}
\usepackage[utf8]{inputenc}
\usepackage[T1]{fontenc}

% ------------------------------------------------------
% Tipografía y formato
% ------------------------------------------------------
\usepackage{lmodern}    % Fuente Latin Modern
\usepackage{microtype}  % Mejor espaciado
\usepackage{xcolor}     % Colores
\definecolor{UGblue}{RGB}{0,33,71}
\definecolor{UGgold}{RGB}{212,175,55}
\usepackage{graphicx}
\usepackage{setspace}
\usepackage{titlesec}
% Añade esto en el preámbulo (antes de \begin{document})
\usepackage{bookmark}
\bookmarksetup{open,numbered}
\usepackage{tikz}
\usetikzlibrary{calc,decorations.pathreplacing,angles,quotes}
\usepackage{pgfplots}
\pgfplotsset{compat=1.18}
\usepackage{caption}

% ------------------------------------------------------
% Matemáticas
% ------------------------------------------------------
\usepackage{amsmath,amssymb,amsthm,mathtools}
\usepackage{mathrsfs}
\usepackage{stmaryrd}
\usepackage{physics}

% ------------------------------------------------------
% Otros paquetes útiles
% ------------------------------------------------------
\usepackage{enumitem}
\usepackage{csquotes}
\usepackage{hyperref}
\hypersetup{
    colorlinks=true,
    linkcolor=UGblue,
    urlcolor=UGgold,
    citecolor=UGblue
}
\usepackage{geometry}
\geometry{margin=2.5cm}
\usepackage{multicol}

% ------------------------------------------------------
% Encabezados y pies de página
% ------------------------------------------------------
% ------------------------------------------------------
% Teoremas, definiciones, etc.
% ------------------------------------------------------
\theoremstyle{plain}
\newtheorem{teo}{Teorema}[section]
\newtheorem{prop}[teo]{Proposición}
\newtheorem{lema}[teo]{Lema}
\theoremstyle{definition}
\newtheorem{defi}[teo]{Definición}
\theoremstyle{remark}
\newtheorem{ejemplo}[teo]{Ejemplo}
\newtheorem{obs}[teo]{Observación}

% ------------------------------------------------------
% Comandos útiles
% ------------------------------------------------------
\newcommand{\N}{\mathbb{N}}
\newcommand{\Z}{\mathbb{Z}}
\newcommand{\Q}{\mathbb{Q}}
\newcommand{\R}{\mathbb{R}}
\newcommand{\C}{\mathbb{C}}
\newcommand{\pfrac}[2]{\left( \frac{#1}{#2} \right)} %parentesis
\newcommand{\bfrac}[2]{\left[ \frac{#1}{#2} \right]} %corchetes
\newcommand{\vecpar}[1]{\left( #1 \right)}
% En el preámbulo
\newcommand{\point}[2]{\left( #1, #2 \right)}
\newcommand{\pointfrac}[4]{\left( \frac{#1}{#2}, \frac{#3}{#4} \right)}
\newcommand{\pointmix}[4]{\left( #1, \frac{#3}{#4} \right)} % Primera coordenada normal, segunda fracción
\renewcommand{\qedsymbol}{$\blacksquare$}
\newcommand{\gt}{\ensuremath{>}}
\newcommand{\lt}{\ensuremath{<}}
\renewcommand{\div}{$÷$}

\renewcommand{\baselinestretch}{1.2}
\setlength{\parindent}{0pt}

\newenvironment{solucion}
{\par\noindent\textbf{Solución.}\ }
{\hfill$\blacktriangleleft$\par}

% ------------------------------------------------------
% Datos personales (modificar según tarea)
% ------------------------------------------------------
\newcommand{\alumno}{Ricardo León Martínez}
\newcommand{\materia}{Elementos de Geometría}
\newcommand{\profesor}{Valentina Muñoz Porras}
\newcommand{\tarea}{Tarea 5}
\newcommand{\fecha}{22/09/2025}

\setlength{\parindent}{0pt}

\begin{document}
\textbf{Ejercicio 1(a)} Para la sucesiones de conjuntos $\{A_n\}_n$ dadas en cada inciso,
demuestra que ellas convergen y halla su limite
(a)
\begin{align*}
    A_n=\left(\frac{-1}{n},\frac{1}{n}\right).
\end{align*}

\begin{proof}

Recordemos que por definicion:
\begin{align*}
    \liminf_{n \to \infty} A_n = \bigcup_{n=1}^\infty \bigcap_{k=n}^\infty A_k, \quad
 \limsup_{n \to \infty} A_n = \bigcap_{n=1}^\infty \bigcup_{k=n}^\infty A_k.
\end{align*}

$\liminf A_n = \{0\}$\\
Primero mostremos $\{0\} \subseteq \liminf A_n$. Para todo $n \in \mathbb{N}$, $0 \in A_k$ para todo $k \geq n$, pues $0 \in (-1/k, 1/k)$. 
Entonces $0 \in \bigcap_{k=n}^\infty A_k$ para todo $n$, luego $0 \in \liminf A_n$. Ahora 
mostremos $\liminf A_n \subseteq \{0\}$. Sea $x \in \liminf A_n$. Entonces existe $n_0$ tal 
que $x \in A_k$ para todo $k \geq n_0$. Es decir, $|x| < 1/k$ para todo $k \geq n_0$.
Si $x \neq 0$, entonces $|x| > 0$. Tomando $k > 1/|x|$, se tiene $1/k < |x|$, contradicción.
Luego $x = 0$
Por tanto, $\liminf A_n = \{0\}$.\\
$\limsup A_n = \{0\}$\\
Primero mostremos $\limsup A_n \subseteq \{0\}$. Sea $x \in \limsup A_n$. Entonces para todo
$n$ existe $k_n \geq n$ tal que $x \in A_{k_n}$, es decir, $|x| < 1/k_n$. Si $x \neq 0$,
$|x| > 0$. Tomemos $n > 1/|x|$. Existe $k_n \geq n$ con $|x| < 1/k_n \leq 1/n < |x|$, 
contradicción. Luego $x = 0$. Ahora mostremos $\{0\} \subseteq \limsup A_n$. Para cada $n$,
$0 \in A_k$ para todo $k$, en particular para algún $k \geq n$. Luego $0 \in 
\bigcup_{k=n}^\infty A_k$ para todo $n$, así $0 \in \limsup A_n$. Por tanto, $\limsup A_n = 
\{0\}$. Como los limites son iguales entonces la sucecion de conjuntos converge a $\{0\}$.
\end{proof}

\textbf{Ejercicio 2(a)} Para la suecion de conjuntos $\{A_n\}_n$ dadas en cada inciso, halla
\begin{align*}
    \inf_{k\geq n}A_k, \sup_{k\geq n}A_k.
\end{align*}
Ademas su la suecion converge y en caso afirmativo, especifica su limite.
\begin{align*}
    A_n=
    \begin{cases*}
        (-n,1) \text{ si } $n$ \text{ es impar},\\
        (1,n)\hfill \text{ si } $n$ \text{ es par}.
    \end{cases*}
\end{align*}

\begin{solucion}

Priero calculemos $\inf_{k \ge n} A_k$. Observemos que:
\begin{align*}
\text{Si }k \text{ es impar } A_k = (-k, 1) \quad
\text{Si } k \text{ es par } A_k = (1, k)
\end{align*}

Para $n$ fijo, consideremos:
\[
\bigcap_{k=n}^\infty A_k =\left[\bigcap_{k\geq n}(-k,1)\right]\cap\left[\bigcap_{k\geq n}(1,k)\right].
\]

 Primer término para $k$ impar:
\[
\bigcap_{k \ge n } (-k, 1) = (-\infty, 1)
\]
pues $-k \to -\infty$ cuando $k$ impar crece.

Segundo término:
\[
\bigcap_{k \ge n} (1, k) = \varnothing
\]
pues para cualquier $x \in \mathbb{R}$:\\
Si $x > 1$, existe $k$ par tal que $k < x$, luego $x \notin (1, k)$\\
Si $x \le 1$, $x \notin (1, k)$ para todo $k$ par



Por tanto:
\[
\inf_{k \ge n} A_k = \varnothing.
\]

Cálculo de $\sup_{k \ge n} A_k$

\[
\sup_{k \ge n} A_k = \bigcup_{k=n}^\infty A_k = \left[ \bigcup_{k \ge n} (-k, 1) \right] \cup \left[ \bigcup_{k \ge n} (1, k) \right].
\]

Primer término:
\[
\bigcup_{k \ge n} (-k, 1) = (-\infty, 1)
\]
Segundo término:
\[
\bigcup_{k \ge n} (1, k) = (1, \infty)
\]


Unión: $(-\infty, 1) \cup (1, \infty) = \mathbb{R} \setminus \{1\}$. Verifiquemos que 
$1 \notin A_k$ para todo $k$:\\
Si $k$ impar: $A_k = (-k, 1)$\\
Si $k$ par: $A_k = (1, k)$ 


Por tanto
\[
\sup_{k \ge n} A_k = \mathbb{R} \setminus \{1\}.
\]

 Calculo límite inferior y superior.

\[
\liminf_{n \to \infty} A_n = \bigcup_{n=1}^\infty \inf_{k \ge n} A_k = \bigcup_{n=1}^\infty \varnothing = \varnothing.
\]

\[
\limsup_{n \to \infty} A_n = \bigcap_{n=1}^\infty \sup_{k \ge n} A_k = \bigcap_{n=1}^\infty (\mathbb{R} \setminus \{1\}) = \mathbb{R} \setminus \{1\}.
\]

Ahora notemos que

\[
\liminf A_n = \varnothing \neq \mathbb{R} \setminus \{1\} = \limsup A_n.
\]

Por tanto, la sucesión no converge.
\end{solucion}

\textbf{Ejercicio 3(a)} Sea $\{x_n,n\in\mathbb{N}\}$ una coleccion de numeros reales.\\
Supon que existe $N$ tal que $x_n\neq x_m$ para cualesquiera $n,m\geq N$ distintos. Demuestra
que
\begin{align*}
    \lim_{n\to\infty}\{x_n\}=\varnothing
\end{align*}
\begin{proof}
    
Recordemos las definiciones:
\[
\liminf_{n \to \infty} C_n = \bigcup_{n=1}^\infty \bigcap_{k=n}^\infty C_k,
\]
\[
\limsup_{n \to \infty} C_n = \bigcap_{n=1}^\infty \bigcup_{k=n}^\infty C_k.
\]

En este caso, $C_n = \{x_n\}$.

Cálculo de $\limsup C_n$

Sea $y \in \limsup C_n$. Entonces, para todo $n \in \mathbb{N}$, existe 
$k \geq n$ tal que $y \in C_k = \{x_k\}$, es decir, $y = x_k$. En particular para $n = N$, 
existe $k_1 \geq N$ con $y = x_{k_1}$. Para $n = k_1 + 1$, existe $k_2 \geq k_1 + 1$ 
con $y = x_{k_2}$.


Pero $k_2 > k_1 \geq N$, y $x_{k_1} = x_{k_2} = y$, lo que contradice la hipótesis de 
que $x_n \neq x_m$ para $n, m \geq N$ distintos.

Por tanto, no existe tal $y$, es decir:
\[
\limsup_{n \to \infty} C_n = \varnothing.
\]

Cálculo de $\liminf C_n$

Sabemos que $\liminf C_n \subseteq \limsup C_n$. \\
Como $\limsup C_n = \varnothing$, entonces:
\[
\liminf_{n \to \infty} C_n = \varnothing.
\]

le sigue que

\[
\liminf C_n = \limsup C_n = \varnothing,
\]
luego
\[
\lim_{n \to \infty} \{x_n\} = \varnothing.
\]

\end{proof}

\textbf{Ejercicio 4(a)} Sean $\{A_n\}_n,\{B_n\}_n$ dos sucesiones de conjuntos tales que
$\lim_{n\to\infty}A_n=A$ y $\lim_{n\to\infty}B_n=B$. Demuestra o refuta las siguientes
igualdades
\begin{align*}
    \lim_{n\to\infty}(A_n\cap B_n)=A\cap B.
\end{align*}
\begin{solucion}
\textbf{Respuesta:}
La igualdad es falsa en general.

\textbf{Contraejemplo:}

Definamos:
\[
A_n = \left(0, 1 + \frac{(-1)^n}{n}\right), \quad 
B_n = \left(1 - \frac{(-1)^n}{n}, 2\right).
\]

Cálculo de $\lim A_n$:\\
Para $n$ par: $A_n = \left(0, 1 + \frac{1}{n}\right)$\\
Para $n$ impar: $A_n = \left(0, 1 - \frac{1}{n}\right)$

\[
\liminf A_n = \bigcup_{n=1}^\infty \bigcap_{k \ge n} A_k = (0,1], \quad
\limsup A_n = \bigcap_{n=1}^\infty \bigcup_{k \ge n} A_k = (0,1]
\]
Luego $\lim A_n = (0,1]$.

Cálculo de $\lim B_n$:\\
Para $n$ par: $B_n = \left(1 - \frac{1}{n}, 2\right)$\\
Para $n$ impar: $B_n = \left(1 + \frac{1}{n}, 2\right)$

\[
\liminf B_n = [1,2), \quad \limsup B_n = [1,2)
\]
Luego $\lim B_n = [1,2)$.

Entonces:
\[
A \cap B = (0,1] \cap [1,2) = \{1\}.
\]

Cálculo de $A_n \cap B_n$:\\
Si $n$ par: $A_n \cap B_n = \left(1 - \frac{1}{n}, 1 + \frac{1}{n}\right)$\\
Si $n$ impar: $A_n \cap B_n = \varnothing$


Cálculo de $\lim (A_n \cap B_n)$:
\[
\liminf (A_n \cap B_n) = \bigcup_{n=1}^\infty \bigcap_{k \ge n} (A_k \cap B_k) = \varnothing
\]
pues para todo $n$ existe $k \ge n$ impar con $A_k \cap B_k = \varnothing$.

\[
\limsup (A_n \cap B_n) = \bigcap_{n=1}^\infty \bigcup_{k \ge n} (A_k \cap B_k) = \{1\}
\]
Como $\liminf \neq \limsup$, $\lim (A_n \cap B_n)$ no existe.

\end{solucion}

\end{document}