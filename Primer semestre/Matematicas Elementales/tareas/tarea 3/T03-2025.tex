\documentclass[14pt]{extarticle}
\usepackage{extsizes}
\usepackage{enumerate}
\usepackage[english,activeacute]{babel}
\usepackage[latin1]{inputenc}
\usepackage{amsmath}
\usepackage{amsfonts}
\usepackage{amssymb,times}
\usepackage{graphicx}
\usepackage{latexsym}
\usepackage{color}
\usepackage{pifont}
\usepackage{bbm}
\usepackage{multicol}
\usepackage{appendix}
%\usepackage{dsfont}
\usepackage{mathrsfs}
%\spanishdecimal{.}
\usepackage[sort&compress]{natbib}
\bibliographystyle{plainnat}
%\usepackage{suthesis-2e}
%\usepackage{verbatim}
%\spanishdecimal{.}
\def\theequation{\thesection.\arabic{equation}}

\textwidth=6.5in
%\lineskip .25cm
%\lineskiplimit .25cm
\textheight=9.5in
\topmargin=-.75in
\topskip=-4pt
\evensidemargin=-2pt
\oddsidemargin=-1pt
%%%%%%%%%%%%%%%%%%%%
%\oddsidemargin 0in \textwidth 6.75in \topmargin 0in \textheight
%8.5in
\parindent 0em
\parskip 1ex

\def\CC{\mathbb{C}}
\newcommand{\rr}{\mathbb{R}}
\newcommand{\pai}{\left(}
\newcommand{\pad}{\right)}
\newcommand{\ci}{\left[}
\newcommand{\cd}{\right]}
\newcommand{\nn}{\mathbb{N}}
\newcommand{\Z}{\mathbb{Z}}
\newcommand{\p}{\mathbb{P}}
\newcommand{\E}{\mathbb{E}}
\newcommand{\lphi}{\widehat{\phi}}
\newcommand{\D}{\mathfrak{D}}
%\renewcommand{\qedsymbol}{\rule{1ex}{1ex}}

\begin{document}
 \begin{flushright}
  \begin{tabular}{|c|c|c|c|c|c|c|c|c|c|c|c|c|c|c|c|}
  \hline
  \ \ $E_1$\ \ &\ \ $E_2$\ \ &\ \ $E_3$\ \ &\ \ $E_4$\ \ &Calif.\\
  \hline
  & &  & & \\
    & & &&\\
   \hline
 \end{tabular}\\
 \end{flushright}
\begin{center}
\large
Matem�ticas Elementales
\normalsize

Agosto - Diciembre de 2025\\

Tarea 3
\end{center}

Resuelve los siguientes ejercicios seg�n el reglamento del curso y \textbf{entrega esta p�gina} con los nombres de los integrantes de equipo colocados en lugar de estas l�neas de instrucciones. 
Todas las demostraciones deben redactarse justificando formalmente todos los pasos necesarios.

Antes de iniciar cada demostraci�n, deber�s indicar claramente cu�les son las hip�tesis de cada ejercicio y qu� es lo que debes demostrar en cada ejercicio.

La tarea debe resolverse con la teor�a vista en el curso o con resultados a lo m�s de los cursos de preparatoria. Est� prohibido utilizar resultados de otros cursos de este mismo semestre y/o trucos de alguna olimpiada relacionada con computaci�n y/o matem�ticas.

\begin{enumerate}

\item Demuestras los siguientes resultados, utilizando el m�todo indicado en cada caso.

\begin{enumerate}
\item Para $n\in\nn$ definimos

$$n!:=n(n-1)(n-2)\dots (2)(1).$$

Adem�s, definimos $0!:=1$. Para $k,n\in\nn\cup\{0\}$ con $k\leq n$ definimos el coeficiente binomial $n$ en $k$ como

$$\binom{n}{k}:=\frac{n!}{(n-k)!k!}.$$

Para $n,k\in\nn$, prueba por v�a directa que 

$$\binom{n}{k}+\binom{n}{k-1}=\binom{n+1}{k}.$$

\item Sean $a,b,c\in\Z$. Prueba por contrapositiva que si $bc$ no es divisible entre $a$, entonces $b$ y $c$ no son divisibles entre $a$.

\item Sean $a,b,c\in\Z$. Sup�n que existe $d\in \Z$ tal que $a,b$ son divisibles entre $d$, pero $c$ no es divisible entre $d$. Demuestra por contradicci�n que no existen $x,y\in \Z$ tales que

$$ax+by=c.$$
\end{enumerate}

\item Sean $x,y\in\rr$. 

\begin{enumerate}
\item Demuestra que 

$$x^2+y^2-xy\geq 0\text{ y }x^2+y^2+xy\geq 0.$$

\item Si $x,y$ son ambos positivos, prueba que

$$\frac{x}{y}+\frac{y}{x}\geq 2.$$

\item Prueba que 

$$\frac{x^2+2}{\sqrt{x^2+1}}\geq 2.$$
\end{enumerate}

\item \begin{enumerate}
\item Sea $n\in \Z$ tal que $n^2$ es m�ltiplo de 3. Demuestra que $n$ tambi�n es m�ltiplo de 3.

\item Demuestra que $\sqrt{3}$ no es racional.
\end{enumerate}
\item Sean $a,b\in [0,1]$ tales que $a< b$. Prueba que 

\begin{enumerate}
\item $$0\leq \frac{b-a}{1-ab}\leq 1.$$
\item $$0\leq \frac{a}{1+b}+\frac{b}{1+a}\leq 1.$$
\end{enumerate}
\end{enumerate}
\end{document}
