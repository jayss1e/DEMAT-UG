\documentclass[14pt]{extarticle}
\usepackage{extsizes}
\usepackage{enumerate}
\usepackage[english,activeacute]{babel}
\usepackage[latin1]{inputenc}
\usepackage{amsmath}
\usepackage{amsfonts}
\usepackage{amssymb,times}
\usepackage{graphicx}
\usepackage{latexsym}
\usepackage{color}
\usepackage{pifont}
\usepackage{bbm}
\usepackage{multicol}
\usepackage{appendix}
%\usepackage{dsfont}
\usepackage{mathrsfs}
%\spanishdecimal{.}
\usepackage[sort&compress]{natbib}
\bibliographystyle{plainnat}
%\usepackage{suthesis-2e}
%\usepackage{verbatim}
%\spanishdecimal{.}
\def\theequation{\thesection.\arabic{equation}}

\textwidth=6.5in
%\lineskip .25cm
%\lineskiplimit .25cm
\textheight=9.5in
\topmargin=-.75in
\topskip=-4pt
\evensidemargin=-2pt
\oddsidemargin=-1pt
%%%%%%%%%%%%%%%%%%%%
%\oddsidemargin 0in \textwidth 6.75in \topmargin 0in \textheight
%8.5in
\parindent 0em
\parskip 1ex
\newcommand{\floor}[1]{\left\lfloor {#1} \right\rfloor}

\def\CC{\mathbb{C}}
\newcommand{\rr}{\mathbb{R}}
\newcommand{\pai}{\left(}
\newcommand{\pad}{\right)}
\newcommand{\ci}{\left[}
\newcommand{\cd}{\right]}
\newcommand{\nn}{\mathbb{N}}
\newcommand{\Z}{\mathbb{Z}}
\newcommand{\p}{\mathbb{P}}
\newcommand{\E}{\mathbb{E}}
\newcommand{\lphi}{\widehat{\phi}}
\newcommand{\D}{\mathfrak{D}}
%\renewcommand{\qedsymbol}{\rule{1ex}{1ex}}

\begin{document}
 \begin{flushright}
  \begin{tabular}{|c|c|c|c|c|c|c|c|c|c|c|c|c|c|c|c|}
  \hline
  \ \ $E_1$\ \ &\ \ $E_2$\ \ &\ \ $E_3$\ \ &\ \ $E_4$\ \ &Calif.\\
  \hline
  & &  & & \\
    & & &&\\
   \hline
 \end{tabular}\\
 \end{flushright}
\begin{center}
\large
Matem�ticas Elementales
\normalsize

Agosto - Diciembre de 2025\\

Tarea 6
\end{center}

Resuelve los siguientes ejercicios seg�n el reglamento del curso y \textbf{entrega esta p�gina} con los nombres de los integrantes de equipo colocados en lugar de estas l�neas de instrucciones. 
Todas las demostraciones deben redactarse justificando formalmente todos los pasos necesarios.

Antes de iniciar cada demostraci�n, deber�s indicar claramente cu�les son las hip�tesis de cada ejercicio y qu� es lo que debes demostrar en cada ejercicio.

La tarea debe resolverse con la teor�a vista en el curso o con resultados a lo m�s de los cursos de preparatoria. Est� prohibido utilizar resultados de otros cursos de este mismo semestre y/o trucos de alguna olimpiada relacionada con computaci�n y/o matem�ticas.

En cada ejercicio, cada inciso distinto debe ser resuelto y redactado por un miembro distinto del equipo.

\begin{enumerate}

\item Sea $f:\rr\to \rr$ dada por $f(x)=(x-1)^2$. Realiza lo siguiente:

\begin{enumerate}
\item Halla el rango de $f$.
\item Halla la imagen directa bajo $f$ de

$$A=[0,\infty), B=\nn, C=\mathbb{Z}, D=[-1,1].$$

\item Halla la imagen inversa bajo $f$ de  

$$A=[0,\infty), B=\nn, C=\mathbb{Z}, D=[-1,1].$$
\item Determina si $f$ es inyectiva y/o suprayectiva.
\end{enumerate}

\item Sean $f:A\to B$ y $g:B\to C$ funciones biyectivas.
\begin{enumerate}
\item  Demuestra que $g\circ f$ es biyectiva.

\item Demuestra que para todo $G\subseteq C$, se cumple

$$(g\circ f)^{-1}(G)=(f^{-1}\circ g^{-1})(G).$$
\end{enumerate}

\item Sean $f:A\to B$ y $A'\subset A$, $B'\subset B$.

\begin{enumerate}
\item Demuestra que si $f$ es inyectiva, entonces 

$$f^{-1}(f(A'))=A'.$$

\item Demuestra que si $f$ es suprayectiva, entonces

$$f(f^{-1}(B'))=B'.$$

\item Demuestra que, en general,

$$A'\subseteq f^{-1}(f(A')), f(f^{-1}(B'))\subseteq B'.$$
\end{enumerate}


\item \leavevmode
\begin{enumerate}
\item Sea $A$ un conjunto no vac�o. Demuestra por definici�n que 
$$\{\{x\}:x\in A\},$$
 es una partici�n de $A$.
\item Sea $f:\rr\to(0,1)$ dada por $f(x)=\frac{1}{1+e^{-x}}$. Demuestra que $f$ es una biyecci�n.
\item Considera la colecci�n de subconjuntos de $\rr$ dada por

$$\{[-n,n],n\in\nn\}.$$

Determina formalmente si esta colecci�n es una partici�n de $\rr$.

\item Para cada $x\in\rr$ definamos

$$E_x:=\{y\in\rr: x-y\in\mathbb{Q}\}.$$

Determina formalmente si $\{E_x:x\in\rr\}$ es una partici�n de $\rr$. \textbf{Nota}: Si para $x,y$ distintos se cumple que $E_x=E_y$, solo se considera una vez al conjunto dado. Es decir, se debe entender que la colecci�n $\{E_x:x\in\rr\}$ consta de todos los conjuntos $E_x$ distintos.
\end{enumerate}
\end{enumerate}
\end{document}
