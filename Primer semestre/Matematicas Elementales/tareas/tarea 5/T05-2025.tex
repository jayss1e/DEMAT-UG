\documentclass[14pt]{extarticle}
\usepackage{extsizes}
\usepackage{enumerate}
\usepackage[english,activeacute]{babel}
\usepackage[latin1]{inputenc}
\usepackage{amsmath}
\usepackage{amsfonts}
\usepackage{amssymb,times}
\usepackage{graphicx}
\usepackage{latexsym}
\usepackage{color}
\usepackage{pifont}
\usepackage{bbm}
\usepackage{multicol}
\usepackage{appendix}
%\usepackage{dsfont}
\usepackage{mathrsfs}
%\spanishdecimal{.}
\usepackage[sort&compress]{natbib}
\bibliographystyle{plainnat}
%\usepackage{suthesis-2e}
%\usepackage{verbatim}
%\spanishdecimal{.}
\def\theequation{\thesection.\arabic{equation}}

\textwidth=6.5in
%\lineskip .25cm
%\lineskiplimit .25cm
\textheight=9.5in
\topmargin=-.75in
\topskip=-4pt
\evensidemargin=-2pt
\oddsidemargin=-1pt
%%%%%%%%%%%%%%%%%%%%
%\oddsidemargin 0in \textwidth 6.75in \topmargin 0in \textheight
%8.5in
\parindent 0em
\parskip 1ex
\newcommand{\floor}[1]{\left\lfloor {#1} \right\rfloor}

\def\CC{\mathbb{C}}
\newcommand{\rr}{\mathbb{R}}
\newcommand{\pai}{\left(}
\newcommand{\pad}{\right)}
\newcommand{\ci}{\left[}
\newcommand{\cd}{\right]}
\newcommand{\nn}{\mathbb{N}}
\newcommand{\Z}{\mathbb{Z}}
\newcommand{\p}{\mathbb{P}}
\newcommand{\E}{\mathbb{E}}
\newcommand{\lphi}{\widehat{\phi}}
\newcommand{\D}{\mathfrak{D}}
%\renewcommand{\qedsymbol}{\rule{1ex}{1ex}}

\begin{document}
 \begin{flushright}
  \begin{tabular}{|c|c|c|c|c|c|c|c|c|c|c|c|c|c|c|c|}
  \hline
  \ \ $E_1$\ \ &\ \ $E_2$\ \ &\ \ $E_3$\ \ &\ \ $E_4$\ \ &Calif.\\
  \hline
  & &  & & \\
    & & &&\\
   \hline
 \end{tabular}\\
 \end{flushright}
\begin{center}
\large
Matem�ticas Elementales
\normalsize

Agosto - Diciembre de 2025\\

Tarea 5
\end{center}

Resuelve los siguientes ejercicios seg�n el reglamento del curso y \textbf{entrega esta p�gina} con los nombres de los integrantes de equipo colocados en lugar de estas l�neas de instrucciones. 
Todas las demostraciones deben redactarse justificando formalmente todos los pasos necesarios.

Antes de iniciar cada demostraci�n, deber�s indicar claramente cu�les son las hip�tesis de cada ejercicio y qu� es lo que debes demostrar en cada ejercicio.

La tarea debe resolverse con la teor�a vista en el curso o con resultados a lo m�s de los cursos de preparatoria. Est� prohibido utilizar resultados de otros cursos de este mismo semestre y/o trucos de alguna olimpiada relacionada con computaci�n y/o matem�ticas.

\begin{enumerate}

\item Calcula las siguientes sumas:
\begin{enumerate}
\item $\sum\limits_{j=1}^n\binom{n}{j}j.$
\item $\sum\limits_{j=0}^n\binom{n}{j}(j+1)^{-1}.$
\item $\sum\limits_{j=1}^n\frac{1}{j(j+1)}.$
\end{enumerate}

\item Demuestra lo siguiente:

\begin{enumerate}
\item $\binom{n}{m}\binom{m}{r}=\binom{n}{r}\binom{n-r}{m-r}$.
\item $\sum\limits_{k=m}^n\binom{k}{m}=\binom{n+1}{m+1}.$
\item Para n�meros reales no negativos $x_1, \cdots , x_n$ es v�lida la desigualdad
\begin{equation*}
    \frac{x_1 + x_2 + \cdots + x_n}{n} \geq \sqrt[n]{x_1x_2\cdots x_n}.
\end{equation*}
\end{enumerate}


\item \begin{enumerate}
\item Demuestra que para cada primo $p\geq 3$, $\binom{2p-1}{p-1}-1$ es divisible entre $p^2$.

\item Muestra que $3^{n+1}$ divide a $2^{3^{n}} + 1$ para cada entero $n \geq 0$.

\item Demuestra que para $n \geq 1$ se cumple que 
$$\floor{\frac{1}{2}} + \floor{\frac{2}{2}} + \floor{\frac{3}{2}} + \cdots + \floor{\frac{n}{2}} = \floor{\frac{n}{2}}\floor{\frac{n+1}{2}},$$

donde $\lfloor x\rfloor$ es la parte entera inferior de $x$ (funci�n piso evaluada en $x$).
\end{enumerate}

\item Realiza lo siguiente:

\begin{enumerate}
\item Considera el arreglo $1,2,\dots,n$. Denotemos por $\sigma(1),\sigma(2),\dots,\sigma(n)$ un \textbf{reordenamiento o permutaci�n} de $1,2,\dots,n$.

Por ejemplo, si tomamos $1,2,3$ y definimos $\sigma$ como 

$$\sigma(1)=3,\sigma(2)=1,\sigma(3)=2,$$
$\sigma$ es el reordenamiento que convierte a $1,2,3$ en $3,1,2$. En cambio, si definimos $s$ como $s(1)=2$, $s(2)=3$ y $s(3)=2$, $s$ no es un reordenamiento, ya que no asigna el valor 1 a ninguno de los n�meros en el arreglo original.

Sup�n que tenemos $n$ de afirmaciones, $P(1),P(2),\dots,P(n)$. Demuestra que todas las afirmaciones son equivalentes dos a dos si y s�lo si para cualquier reordenamiento $\sigma$, aplicado a los �ndices, se cumple que $P(\sigma(n))\Rightarrow P(\sigma(1))$ y $P(\sigma(k))\Rightarrow P(\sigma(k+1))$ para todo $k=1,2,\dots,n-1$.



\item Jon�s tiene un conejo al que apoda Che�ol Bunny�o. El Che�ol Bunny�o tiene un juguete de madera (para morder) que consta de $n$ cuadrados acomodados en un patr�n rectangular. El juguete completo o cualquier pieza m�s peque�a, puede partirse sobre la recta vertical o sobre la recta horizontal que separa los cuadrados.

El Che�ol Bunny�o es tan listo que, cada vez que muerde el juguete, separa un pedazo seg�n las reglas descritas anteriormente (s�lo se\-para un pedazo en cada mordida). Determina cu�ntas veces el Che�ol Bunny�o debe morder el juguete para que �ste se parta en $n$ piezas distintas (por supuesto, considera que el Che�ol Bunny�o \textbf{no} se come las piezas una vez separadas y no puede romper los cuadrados que forman el juguete).
\end{enumerate}


\item (150 puntos extra sobre el total de las tareas, previo al promedio). Para $n\geq 2$, sean $a_1,\dots,a_n$ n�meros reales distintos entre s� y distintos de cero. Calcula la suma

$$\sum\limits_{j=1}^n\frac{p(a_j)}{\prod\limits_{k=1,k\neq j}^n(a_k-a_j)},$$

donde $p$ es un polinomio arbitrario (con coeficientes reales) cuyo grado es $n-1$.

\textbf{Sugerencia}: considera primero el caso $p(x)=C x^k$ para alguna constante real $C$ y $k\leq n-1$.
\end{enumerate}
\end{document}
