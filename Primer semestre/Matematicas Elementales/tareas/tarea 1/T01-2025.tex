\documentclass[14pt]{extarticle}
\usepackage{extsizes}
\usepackage{enumerate}
\usepackage[english,activeacute]{babel}
\usepackage[latin1]{inputenc}
\usepackage{amsmath}
\usepackage{amsfonts}
\usepackage{amssymb,times}
\usepackage{graphicx}
\usepackage{latexsym}
\usepackage{color}
\usepackage{pifont}
\usepackage{bbm}
\usepackage{multicol}
\usepackage{appendix}
%\usepackage{dsfont}
\usepackage{mathrsfs}
%\spanishdecimal{.}
\usepackage[sort&compress]{natbib}
\bibliographystyle{plainnat}
%\usepackage{suthesis-2e}
%\usepackage{verbatim}
%\spanishdecimal{.}
\def\theequation{\thesection.\arabic{equation}}

\textwidth=6.5in
%\lineskip .25cm
%\lineskiplimit .25cm
\textheight=9.5in
\topmargin=-.75in
\topskip=-4pt
\evensidemargin=-2pt
\oddsidemargin=-1pt
%%%%%%%%%%%%%%%%%%%%
%\oddsidemargin 0in \textwidth 6.75in \topmargin 0in \textheight
%8.5in
\parindent 0em
\parskip 1ex

\def\CC{\mathbb{C}}
\newcommand{\rr}{\mathbb{R}}
\newcommand{\pai}{\left(}
\newcommand{\pad}{\right)}
\newcommand{\ci}{\left[}
\newcommand{\cd}{\right]}
\newcommand{\N}{\mathbb{N}}
\newcommand{\B}{\mathcal{B}}
\newcommand{\p}{\mathbb{P}}
\newcommand{\E}{\mathbb{E}}
\newcommand{\lphi}{\widehat{\phi}}
\newcommand{\D}{\mathfrak{D}}
%\renewcommand{\qedsymbol}{\rule{1ex}{1ex}}

\begin{document}
 \begin{flushright}
  \begin{tabular}{|c|c|c|c|c|c|c|c|c|c|c|c|c|c|c|c|}
  \hline
  \ \ $E_1$\ \ &\ \ $E_2$\ \ &\ \ $E_3$\ \ &\ \ $E_4$\ \ &Calif.\\
  \hline
  & &  & & \\
    & & &&\\
   \hline
 \end{tabular}\\
 \end{flushright}
\begin{center}
\large
Matem�ticas Elementales
\normalsize

Agosto - Diciembre de 2025\\

Tarea 1
\end{center}

Resuelve los siguientes ejercicios seg�n el reglamento del curso y \textbf{entrega esta p�gina} con los nombres de los integrantes de equipo colocados en lugar de estas l�neas de instrucciones.


En toda esta tarea, $P,Q$ son afirmaciones.

\begin{enumerate}

\item Demuestra los numerales 11 y 12 de la Proposici�n 1.2 (Leyes de De Morgan).

\item Demuestra el numeral 8 de la Proposici�n 1.4 (Dilema constructivo).

\item Escribe en s�mbolos los siguientes enunciados:

\begin{enumerate}
\item No me gusta el chile, pero s� el chocolate.
\item Si a Jon�s le gusta la michelada, tiene sentido decir que a Jon�s le gusta la cerveza.
\item Los gatos son peludos. Vladimir es peludo. Luego, Vladimir es un gato.
\item Los dulces contienen az�car. Este chocolate no es dulce, as� que este chocolate no contiene az�car.
\end{enumerate}

\item Realiza lo siguiente:

\begin{enumerate}
\item Utilizando tablas de verdad, demuestra que $(P\vee Q)\wedge Q\Longleftrightarrow Q$.
\item Prueba de dos formas (con tablas de verdad y despu�s con reglas de inferencia) que

$$[(P\longrightarrow Q)\wedge Q]\longrightarrow P \Longleftrightarrow Q\longrightarrow P.$$

\item Prueba que 

$$P\wedge (Q\vee P)\Longleftrightarrow P.$$

\item Utilizando reglas de inferencia, demuestra que

$$[(P\vee Q)\longrightarrow (\neg P\wedge Q)]\wedge (P\longrightarrow Q) \Longleftrightarrow \neg P.$$
\end{enumerate}

\end{enumerate}
\end{document}
