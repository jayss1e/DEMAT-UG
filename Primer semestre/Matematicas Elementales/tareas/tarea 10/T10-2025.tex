\documentclass[14pt]{extarticle}
\usepackage{extsizes}
\usepackage{enumerate}
\usepackage[english,activeacute]{babel}
\usepackage[latin1]{inputenc}
\usepackage{amsmath}
\usepackage{amsfonts}
\usepackage{amssymb,times}
\usepackage{graphicx}
\usepackage{latexsym}
\usepackage{color}
\usepackage{pifont}
\usepackage{bbm}
\usepackage{multicol}
\usepackage{appendix}
%\usepackage{dsfont}
\usepackage{mathrsfs}
%\spanishdecimal{.}
\usepackage[sort&compress]{natbib}
\bibliographystyle{plainnat}
%\usepackage{suthesis-2e}
%\usepackage{verbatim}
%\spanishdecimal{.}
\def\theequation{\thesection.\arabic{equation}}

\textwidth=6.5in
%\lineskip .25cm
%\lineskiplimit .25cm
\textheight=9.5in
\topmargin=-.75in
\topskip=-4pt
\evensidemargin=-2pt
\oddsidemargin=-1pt
%%%%%%%%%%%%%%%%%%%%
%\oddsidemargin 0in \textwidth 6.75in \topmargin 0in \textheight
%8.5in
\parindent 0em
\parskip 1ex
\newcommand{\floor}[1]{\left\lfloor {#1} \right\rfloor}

\def\CC{\mathbb{C}}
\newcommand{\rr}{\mathbb{R}}
\newcommand{\pai}{\left(}
\newcommand{\pad}{\right)}
\newcommand{\ci}{\left[}
\newcommand{\cd}{\right]}
\newcommand{\nn}{\mathbb{N}}
\newcommand{\Z}{\mathbb{Z}}
\newcommand{\p}{\mathbb{P}}
\newcommand{\E}{\mathbb{E}}
\newcommand{\lphi}{\widehat{\phi}}
\newcommand{\D}{\mathfrak{D}}
\newcommand{\Bmod}{\bmod \hspace{1mm}}
\newcommand{\Vmod}[2]{ {#1} \mom{\Bmod{#2}} }
%\renewcommand{\qedsymbol}{\rule{1ex}{1ex}}

\begin{document}
 \begin{flushright}
  \begin{tabular}{|c|c|c|c|c|c|c|c|c|c|c|c|c|c|c|c|}
  \hline
  \ \ $E_1$\ \ &\ \ $E_2$\ \ &\ \ $E_3$\ \ &\ \ $E_4$\ \ &Calif.\\
  \hline
  & &  & & \\
    & & &&\\
   \hline
 \end{tabular}\\
 \end{flushright}
\begin{center}
\large
Matem�ticas Elementales
\normalsize

Agosto - Diciembre de 2025\\

Tarea 10
\end{center}

Resuelve los siguientes ejercicios seg�n el reglamento del curso y \textbf{entrega esta p�gina} con los nombres de los integrantes de equipo colocados en lugar de estas l�neas de instrucciones. 
Todas las demostraciones deben redactarse justificando formalmente todos los pasos necesarios.

La tarea debe resolverse \textbf{con la teor�a vista en el curso o con resultados a lo m�s de los cursos de preparatoria}. Est� prohibido utilizar resultados de otros cursos de este mismo semestre o de semestres posteriores. Tampoco est� permitido utilizar trucos de alguna olimpiada relacionada con computaci�n y/o matem�ticas.

Cada inciso distinto debe ser resuelto por un alumno distinto y, en cada caso, deber� colocarse el nombre de la persona que redact� el inciso.

\begin{enumerate}
\item Considera $(\rr,+,\cdot)$ como se definieron en clase. 

Dado un entero $n\geq2$ y un real $x>0$, consideramos

$$\alpha_n:=\{r\in\rr:r^n<x\}.$$

Definimos

$$\sqrt[n]{x}:=\sup\alpha_n. $$

Demuestra que para $\alpha, \beta \in \rr^{+}, n \in \Z^{+}, m \in \Z^+$, se cumple lo siguiente:
                
                    \begin{enumerate}
                    	\item
                    	$\sqrt[n]{\alpha} \sqrt[n]{\beta} = \sqrt[n]{\alpha \beta}$
                    	\item
                    	$\sqrt[m]{\sqrt[n]{\alpha}} = \sqrt[mn]{\alpha}$
                    	\item
                    	$\pai\sqrt[n]{\alpha}\pad^{m} = \sqrt[n]{\alpha^{m}}$
                    	\item
                    	$\sqrt[n]{\alpha^{m}} = \sqrt[s]{\alpha^{r}} \Longleftrightarrow \frac{m}{n} = \frac{r}{s}.$
                    \end{enumerate}

\item Considera $\rr$, $\mathbb{Q}$, $\mathbb{Z}$ y $\nn_0$ como se construyeron en clase. Definimos una funci�n $F:\nn_0\to \nn$ tal que $F(0)=1$ y para $n\in\nn$, $F(n)=n\cdot F(n-1)$. a $F$ le llamaremos \textbf{factorial} de $n$ y escribiremos $n!=F(n)$.

Sea

$$a_n=\sum\limits_{k=0}^n\frac{1}{k!}.$$

\begin{enumerate}
\item Considera $b_1=1$ y $b_n$ para $n\in\nn\backslash\{1\}$ tal que $b_n=-\frac{1}{n!}+b_{n-1}$ y

$$\frac{1}{(n+1)!}<b_n<3.$$

Demuestra que $a_n<3-b_n$ para todo $n\in\nn$.

\item Sea 

$$A_n:=\{q\in\mathbb{Q}:q<a_n\}.$$

Demuestra que el conjunto 

$$e:=\bigcup_{n\in\nn}A_n,$$

es un n�mero real. Dicho n�mero real es la base del logaritmo nepe\-riano.
\end{enumerate}

\item \leavevmode

\begin{enumerate}
 \item
                    Sean $a,b \in \Z$. Demuestra que
                    \[
                        (a,b) = \pai|a|,|b|\pad.
                    \]
                    \item
                    Demuestra  que si $d = (a,b)$ y $d = ar + bs$, entonces $(r,s)=1$.
                    
                    \item Sean $a,b\in\mathbb{Z}$ y $d=(a,b)$. Sean $a',b'$ tales que $a=da'$, $b=db'$. Demuestra que si $a|c$ y $b|c$, entonces $a'b'd | c$.
                    
\item Sea $k$ un entero positivo, demuestra que

$$(ka,kb)=k(a,b),\quad [ka,kb]=k[a,b],\quad \forall a,b\in\mathbb{Z}.$$

\item Utiliza el algoritmo de Euclides para hallar el MCD de las siguientes parejas de n�meros. Adem�s, utiliza este mismo algoritmo para escribir al MCD de cada pareja como combinaci�n lineal de los n�meros en dicha pareja.

\begin{enumerate}
\item 329, 1005.
\item 1302, 1224.
\item 1816, -1789.
\item -666, -12309.
\end{enumerate}

\item Determina cu�les de las siguientes ecuaciones diofantinas tienen soluciones enteras. Para aquellas en las que sea posible, halla el conjunto de todas sus soluciones enteras.
                       
\begin{enumerate}
 \item $35x + 17y = 14$;
                        \item
                        $1242x + 1476y = 49$;
                        \item
                        $15x + 21y = 10.$
                         \item
                        $696 x+408 y=48$
                        \item
                        $(6 n+1) x+3 n y=12$
\end{enumerate}                       

\end{enumerate}

\item \leavevmode

\begin{enumerate}
\item Utilizando el principio del buen orden, demuestra que todo entero mayor a 1 es divisible entre al menos un n�mero primo.

\item Demuestra que el conjunto de n�meros primos no es finito.

\item Demuestra que si $a\equiv b(\Bmod m)$ y $b \equiv c(\Bmod m)$, entonces  $$a \equiv c(\Bmod m).$$
                    \item
                    Prueba que $a \equiv 0(\Bmod m)$ si y solo si $m|a$.
                    \item
                    Demuestra que  que si $a \equiv b(\Bmod m)$ entonces  $a c \equiv b c(\Bmod m)$ para cualquier entero $c$.
                    \item
                    Demuestra que si $a+c \equiv b+c(\Bmod m),$ entonces $$a \equiv b (\Bmod m).$$
                    
                    \item
                    Demuestra que si $a c \equiv b c(\Bmod m)$ y $m$ y $c$ son primos entre s� entonces $a \equiv b(\Bmod m)$.
                    
                    \item
                    Sup�n que $m_{1}, m_{2}, \dots, m_{n}$ son primos relativos dos a dos. Demuestra que las congruencias
                    \[
                        x \equiv a_{1}\left(\bmod m_{1}\right), x \equiv a_{2}\left(\bmod m_{2}\right), \cdots, x \equiv a_{n}\left(\bmod m_{n}\right),
                    \]
                    tienen soluci�n com�n.
                    \item
                    Resuelve las siguientes congruencias:
                    \begin{enumerate}
                        \item
                        $16 x-9\equiv 0(\bmod 35)$.
                        \item
                        $200 x+315 \equiv 0(\bmod 441)$.
                        \item
                        $(2 n+1) x+7 \equiv 0(\bmod 4 n)$.
                        \item
                        $(3 n-2) x+5 n\equiv 0(\bmod 9 n-9)$.
                    \end{enumerate}
                    \item
                    Resuelve los siguientes sistemas de congruencias:
                        \begin{multicols}{3}
                            \begin{enumerate}
                                \item
                                $x \equiv 0 (\Bmod 3)$
                                
                                $x \equiv 0(\Bmod 8)$
                                \item
                                $x \equiv 1 (\Bmod 25)$
                                
                                $x \equiv 7(\Bmod 35)$
                                \item
                                $x \equiv 3 (\Bmod 17)$
                                
                                $x \equiv 4(\Bmod 21)$
                                
                                
                                $x \equiv 5 (\Bmod 25)$
                            \end{enumerate}
                        \end{multicols}
\end{enumerate}
\end{enumerate}

 
\end{document}
