\documentclass[14pt]{extarticle}
\usepackage{extsizes}
\usepackage{enumerate}
\usepackage[english,activeacute]{babel}
\usepackage[latin1]{inputenc}
\usepackage{amsmath}
\usepackage{amsfonts}
\usepackage{amssymb,times}
\usepackage{graphicx}
\usepackage{latexsym}
\usepackage{color}
\usepackage{pifont}
\usepackage{bbm}
\usepackage{multicol}
\usepackage{appendix}
%\usepackage{dsfont}
\usepackage{mathrsfs}
%\spanishdecimal{.}
\usepackage[sort&compress]{natbib}
\bibliographystyle{plainnat}
%\usepackage{suthesis-2e}
%\usepackage{verbatim}
%\spanishdecimal{.}
\def\theequation{\thesection.\arabic{equation}}

\textwidth=6.5in
%\lineskip .25cm
%\lineskiplimit .25cm
\textheight=9.5in
\topmargin=-.75in
\topskip=-4pt
\evensidemargin=-2pt
\oddsidemargin=-1pt
%%%%%%%%%%%%%%%%%%%%
%\oddsidemargin 0in \textwidth 6.75in \topmargin 0in \textheight
%8.5in
\parindent 0em
\parskip 1ex
\newcommand{\floor}[1]{\left\lfloor {#1} \right\rfloor}

\def\CC{\mathbb{C}}
\newcommand{\rr}{\mathbb{R}}
\newcommand{\pai}{\left(}
\newcommand{\pad}{\right)}
\newcommand{\ci}{\left[}
\newcommand{\cd}{\right]}
\newcommand{\nn}{\mathbb{N}}
\newcommand{\Z}{\mathbb{Z}}
\newcommand{\p}{\mathbb{P}}
\newcommand{\E}{\mathbb{E}}
\newcommand{\lphi}{\widehat{\phi}}
\newcommand{\D}{\mathfrak{D}}
%\renewcommand{\qedsymbol}{\rule{1ex}{1ex}}

\begin{document}
 \begin{flushright}
  \begin{tabular}{|c|c|c|c|c|c|c|c|c|c|c|c|c|c|c|c|}
  \hline
  \ \ $E_1$\ \ &\ \ $E_2$\ \ &\ \ $E_3$\ \ &\ \ $E_4$\ \ &Calif.\\
  \hline
  & &  & & \\
    & & &&\\
   \hline
 \end{tabular}\\
 \end{flushright}
\begin{center}
\large
Matem�ticas Elementales
\normalsize

Agosto - Diciembre de 2025\\

Tarea 4
\end{center}

Resuelve los siguientes ejercicios seg�n el reglamento del curso y \textbf{entrega esta p�gina} con los nombres de los integrantes de equipo colocados en lugar de estas l�neas de instrucciones. 
Todas las demostraciones deben redactarse justificando formalmente todos los pasos necesarios.

Antes de iniciar cada demostraci�n, deber�s indicar claramente cu�les son las hip�tesis de cada ejercicio y qu� es lo que debes demostrar en cada ejercicio.

La tarea debe resolverse con la teor�a vista en el curso o con resultados a lo m�s de los cursos de preparatoria. Est� prohibido utilizar resultados de otros cursos de este mismo semestre y/o trucos de alguna olimpiada relacionada con computaci�n y/o matem�ticas.

\begin{enumerate}

\item Dadas dos funciones $f,g:\rr\to\rr$, definimos la \textbf{composici�n} de $f,g$, denotada por $f\circ g$, como
$$(f\circ g)(x):=f(g(x)),\forall\ x\in\rr.$$

Sup�n que $g$ es continua en $z$ y $f$ es continua en $g(z)$. Demuestra que $f\circ g$ es continua en $z$.

\item \begin{enumerate}
\item Define formalmente $\lim\limits_{x\to y} f(x)=\pm \infty$ y demuestra que los l�mites laterales de $f(x)=1/x$, cuando $x\to0$, no coinciden y ninguno es finito.

\item Dadas dos funciones $f,g:\rr\to\rr$ tales que $\lim\limits_{x\to z}f(x)g(x)$ existe y es finito �se cumple que $\lim\limits_{x\to z}f(x)$ y $\lim\limits_{x\to z}g(x)$ existen y son finitos? Justifica formalmete tu respuesta.


\end{enumerate}

\item Sean $f,g:\rr\to [0,\infty)$ funciones continuas, demuestra que $h:\rr\to [0,\infty)$ dada por

$$h(x)=\ci 1+g(x)\cd^{f(x)},$$

es continua.

\item Demuestra utilizando la definici�n que si $f:\rr\to\rr$ y
$g:\rr\to [0,\infty)$ son funciones continuas, entonces $h:\rr\to\rr$ dada por
$$h(x)=\frac{f(x)}{1+g(x)},$$
es continua.

\end{enumerate}
\end{document}
