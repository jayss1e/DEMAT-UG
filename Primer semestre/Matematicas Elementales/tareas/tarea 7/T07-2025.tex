\documentclass[14pt]{extarticle}
\usepackage{extsizes}
\usepackage{enumerate}
\usepackage[english,activeacute]{babel}
\usepackage[latin1]{inputenc}
\usepackage{amsmath}
\usepackage{amsfonts}
\usepackage{amssymb,times}
\usepackage{graphicx}
\usepackage{latexsym}
\usepackage{color}
\usepackage{pifont}
\usepackage{bbm}
\usepackage{multicol}
\usepackage{appendix}
%\usepackage{dsfont}
\usepackage{mathrsfs}
%\spanishdecimal{.}
\usepackage[sort&compress]{natbib}
\bibliographystyle{plainnat}
%\usepackage{suthesis-2e}
%\usepackage{verbatim}
%\spanishdecimal{.}
\def\theequation{\thesection.\arabic{equation}}

\textwidth=6.5in
%\lineskip .25cm
%\lineskiplimit .25cm
\textheight=9.5in
\topmargin=-.75in
\topskip=-4pt
\evensidemargin=-2pt
\oddsidemargin=-1pt
%%%%%%%%%%%%%%%%%%%%
%\oddsidemargin 0in \textwidth 6.75in \topmargin 0in \textheight
%8.5in
\parindent 0em
\parskip 1ex
\newcommand{\floor}[1]{\left\lfloor {#1} \right\rfloor}

\def\CC{\mathbb{C}}
\newcommand{\rr}{\mathbb{R}}
\newcommand{\pai}{\left(}
\newcommand{\pad}{\right)}
\newcommand{\ci}{\left[}
\newcommand{\cd}{\right]}
\newcommand{\nn}{\mathbb{N}}
\newcommand{\Z}{\mathbb{Z}}
\newcommand{\p}{\mathbb{P}}
\newcommand{\E}{\mathbb{E}}
\newcommand{\lphi}{\widehat{\phi}}
\newcommand{\D}{\mathfrak{D}}
%\renewcommand{\qedsymbol}{\rule{1ex}{1ex}}

\begin{document}
 \begin{flushright}
  \begin{tabular}{|c|c|c|c|c|c|c|c|c|c|c|c|c|c|c|c|}
  \hline
  \ \ $E_1$\ \ &\ \ $E_2$\ \ &\ \ $E_3$\ \ &\ \ $E_4$\ \ &Calif.\\
  \hline
  & &  & & \\
    & & &&\\
   \hline
 \end{tabular}\\
 \end{flushright}
\begin{center}
\large
Matem�ticas Elementales
\normalsize

Agosto - Diciembre de 2025\\

Tarea 6
\end{center}

Resuelve los siguientes ejercicios seg�n el reglamento del curso y \textbf{entrega esta p�gina} con los nombres de los integrantes de equipo colocados en lugar de estas l�neas de instrucciones. 
Todas las demostraciones deben redactarse justificando formalmente todos los pasos necesarios.

Antes de iniciar cada demostraci�n, deber�s indicar claramente cu�les son las hip�tesis de cada ejercicio y qu� es lo que debes demostrar en cada ejercicio.

La tarea debe resolverse con la teor�a vista en el curso o con resultados a lo m�s de los cursos de preparatoria. Est� prohibido utilizar resultados de otros cursos de este mismo semestre y/o trucos de alguna olimpiada relacionada con computaci�n y/o matem�ticas.

Cada inciso distinto debe ser resuelto por un alumno distinto y, en cada caso, deber� colocarse el nombre de la persona que redact� el inciso.

En cada ejercicio, cada inciso distinto debe ser resuelto y redactado por un miembro distinto del equipo.

\begin{enumerate}
\item Para las sucesiones de conjuntos $\{A_n\}_n$ dadas en cada inciso, demuestra que ellas convergen y halla su l�mite.

\begin{enumerate}
\item $$A_n=\pai\frac{-1}{n},\frac{1}{n}\pad.$$

\item $$A_n=\ci 1-\frac{1}{n},2-\frac{1}{n}\cd.$$

\item $$A_n=[0,g(n)],$$

donde $g(n)=1+1/n^2$.

\item $$A_n=\{k\in\nn: k\leq n\}.$$
\end{enumerate}

\item Para las sucesiones de conjuntos $\{A_n\}_n$ dadas en cada inciso, halla 
$$\inf_{k\geq n}A_k,\sup_{k\geq n}A_k.$$
 Adem�s, determina si la sucesi�n converge y, en caso afirmativo, especifica su l�mite.

\begin{enumerate}
\item $$A_n=\left\{
\begin{array}{cc}
(-n,1)&\text{ si $n$ es impar},\\
(1,n)&\text{ si $n$ es par}.
\end{array}
\right.$$

\item  $$A_1=\{1\}, A_n=\left\{\frac{n}{n+1}\right\}\text{ para }n\geq 2.$$

\item $$A_1=(0,1/2], A_n=\pai \frac{1}{n-1}+\frac{n-2}{2},\frac{1}{n-1}+\frac{n-2}{2}\cd, n\geq 2.$$

\item $$A_1=\varnothing, A_n=\pai \frac{1}{n},\frac{1}{n-1}\pad, \text{ para }n\geq 2.$$
\end{enumerate}

\item Sea $\{x_n,n\in\nn\}$ una colecci�n de n�meros reales.

\begin{enumerate}
\item Sup�n que existe $N$ tal que $x_n\neq x_m$ para cualesquiera $n,m\geq N$ distintos. Demuestra que
$$\lim\limits_{n\to\infty}\{x_n\}=\varnothing.$$
\item Sea $f:\nn\to\rr$ tal que $f(n)=x_n$. Sup�n que existe $x\in\rr$ tal que $\lim_{n\to\infty}f(n)=x$.
Demuestra que el l�mite $\lim_{n\to\infty}\{x,x_n\}$ existe y h�llalo expl�citamente.
\end{enumerate}

\item Sean $\{A_n\}_n, \{B_n\}_n$ dos sucesiones de conjuntos tales que $\lim_{n\to\infty}A_n=A$ y $\lim_{n\to\infty}B_n=B$. Demuestra o refuta las siguientes igualdades:

\begin{enumerate}
\item $$\lim\limits_{n\to\infty}(A_n\cap B_n)=A\cap B.$$
\item $$\lim\limits_{n\to\infty}(A_n\cup B_n)=A\cup B.$$
\end{enumerate}

\item Ejercicio extra: Sea $f:\nn\to\rr$ tal que $\lim_{n\to\infty}f(n)=x$ para alg�n $x\in(0,\infty]$ ($x$ puede ser infinito). Sea $A_n:=[0,f(n)]$ para $n\in\nn$, demuestra o refuta que 

$$\lim\limits_{n\to\infty}A_n=[0,x].$$

\textbf{Nota}: Este ejercicio contar� como una tarea cuya calificaci�n s�lo podr� ser 0 � 100 (100 si todos los argumentos son correctos y no hay errores graves de redacci�n. En caso contrario, se calificar� con cero). Deber� entregarse en f�sico (escrito a mano o impreso) de manera individual (s�lo se dar�n puntos a las personas que lo entreguen) y la fecha m�xima de entrega ser� el 17 de noviembre a las 9:30. Se aplicar�n los mismos criterios del reglamento que se aplican a las tareas.
\end{enumerate}
\end{document}
