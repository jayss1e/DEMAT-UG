\documentclass[14pt]{extarticle}
\usepackage{extsizes}
\usepackage{enumerate}
\usepackage[english,activeacute]{babel}
\usepackage[latin1]{inputenc}
\usepackage{amsmath}
\usepackage{amsfonts}
\usepackage{amssymb,times}
\usepackage{graphicx}
\usepackage{latexsym}
\usepackage{color}
\usepackage{pifont}
\usepackage{bbm}
\usepackage{multicol}
\usepackage{appendix}
%\usepackage{dsfont}
\usepackage{mathrsfs}
%\spanishdecimal{.}
\usepackage[sort&compress]{natbib}
\bibliographystyle{plainnat}
%\usepackage{suthesis-2e}
%\usepackage{verbatim}
%\spanishdecimal{.}
\def\theequation{\thesection.\arabic{equation}}

\textwidth=6.5in
%\lineskip .25cm
%\lineskiplimit .25cm
\textheight=9.5in
\topmargin=-.75in
\topskip=-4pt
\evensidemargin=-2pt
\oddsidemargin=-1pt
%%%%%%%%%%%%%%%%%%%%
%\oddsidemargin 0in \textwidth 6.75in \topmargin 0in \textheight
%8.5in
\parindent 0em
\parskip 1ex
\newcommand{\floor}[1]{\left\lfloor {#1} \right\rfloor}

\def\CC{\mathbb{C}}
\newcommand{\rr}{\mathbb{R}}
\newcommand{\pai}{\left(}
\newcommand{\pad}{\right)}
\newcommand{\ci}{\left[}
\newcommand{\cd}{\right]}
\newcommand{\nn}{\mathbb{N}}
\newcommand{\Z}{\mathbb{Z}}
\newcommand{\p}{\mathbb{P}}
\newcommand{\E}{\mathbb{E}}
\newcommand{\lphi}{\widehat{\phi}}
\newcommand{\D}{\mathfrak{D}}
%\renewcommand{\qedsymbol}{\rule{1ex}{1ex}}

\begin{document}
 \begin{flushright}
  \begin{tabular}{|c|c|c|c|c|c|c|c|c|c|c|c|c|c|c|c|}
  \hline
  \ \ $E_1$\ \ &\ \ $E_2$\ \ &\ \ $E_3$\ \ &\ \ $E_4$\ \ &Calif.\\
  \hline
  & &  & & \\
    & & &&\\
   \hline
 \end{tabular}\\
 \end{flushright}
\begin{center}
\large
Matem�ticas Elementales
\normalsize

Agosto - Diciembre de 2025\\

Tarea 9
\end{center}

Resuelve los siguientes ejercicios seg�n el reglamento del curso y \textbf{entrega esta p�gina} con los nombres de los integrantes de equipo colocados en lugar de estas l�neas de instrucciones. 
Todas las demostraciones deben redactarse justificando formalmente todos los pasos necesarios.

La tarea debe resolverse \textbf{con la teor�a vista en el curso o con resultados a lo m�s de los cursos de preparatoria}. Est� prohibido utilizar resultados de otros cursos de este mismo semestre o de semestres posteriores. Tampoco est� permitido utilizar trucos de alguna olimpiada relacionada con computaci�n y/o matem�ticas.

Cada inciso distinto debe ser resuelto por un alumno distinto y, en cada caso, deber� colocarse el nombre de la persona que redact� el inciso.

\begin{enumerate}
\item Con base en la definici�n de suma en $\nn_0$ vista en clase, demuestra que se cumplen las siguientes propiedades para cualesquiera $n,m,k\in\nn_0$.

\begin{enumerate}
\item $n+m=m+n$.
\item $n+(m+k)=(n+m)+k$.
\item Se cumple que $n<m$ si y solo si existe $k\in\nn$ tal que $m=n+k$.

\item $n+k=m+k\Longleftrightarrow n=m$.
\end{enumerate}

\item Con base en las definiciones de suma y producto en $\nn_0$ vistas en clase, demuestra que se cumplen las siguientes propiedades para cualesquiera $n,m,k\in\nn_0$

\begin{enumerate}
\item $1\cdot n= n=n\cdot 1$.

\item $m(n+1)=mn+m$.

\item $m(n+k)=mn+mk$.
\item $(n+k)m=nm+km$.
\item $nm=mn$.
\item $k(nm)=(kn)m$.
\end{enumerate}

\item Demuestra que las operaciones de suma y producto en $\mathbb{Z}$ vistas en clase, satisfacen las siguientes propiedades para cualesquiera $a,b,c\in\mathbb{Z}$.

\begin{enumerate}
\item $(a+b)+c=a+(b+c)$.
\item $a+0=a$.
\item $a+b=c+b\Longleftrightarrow a=c$.
\item $a(-b)=-(ab)=(-a)b$.
\item $a(bc)=(ab)c$.
\item $a(1)=1(a)=a$.
\item $a(0)=0=0(a)$.
\item $a(b+c)=ab+ac$.
\item $ab=ac\Longleftrightarrow b=c,\quad \forall a\neq 0$.
\item El inverso aditivo es �nico.
\end{enumerate}

\item Dado $(\mathbb{Q},+,\cdot)$ como se defini� en clase, demuestra lo siguiente:

\begin{enumerate}
\item $+,\cdot$ son operaciones conmutativas.
\item $+,\cdot$ son asociativas.
\item Existen los neutros aditivos  multiplicativo en $(\mathbb{Q},+,\cdot)$.
\item El neutro aditivo y el neutro multiplicativo en $(\mathbb{Q},+,\cdot)$ son �nicos.
\item $+,\cdot$ cumplen las leyes de la cancelaci�n.
\item $\mathbb{Q}_{++}$ es cerrado bajo suma y producto.
\item
                \[
                \frac{a}{b}>\frac{a'}{b'}, \frac{c}{d}>\frac{c'}{d'} \Longrightarrow \frac{a}{b}+\frac{c}{d}>\frac{a'}{b'}+\frac{c'}{d'},
                \]
                \item 
                \[
                \frac{a}{b}>\frac{a'}{b'} \Rightarrow \frac{a}{b}+\frac{c}{d}>\frac{a'}{b'}+\frac{c}{d},
                \]
                \item
                \[ 
                \frac{a}{b}>\frac{a'}{b'} \geq 0\ \text{   y   }\ \frac{c}{d}>\frac{c'}{d'} \geq 0 \Longrightarrow \frac{a}{b} \cdot \frac{c}{d}>\frac{a'}{b'} \cdot \frac{c'}{d'},
                \]
                \item
                \[
                 \frac{a}{b}>\frac{a'}{b'}, \frac{c}{d}>0\Longrightarrow\frac{a}{b} \cdot \frac{c}{d}>\frac{a'}{b'} \cdot \frac{c}{d},
                 \]
                \item
                 \[
                 \frac{a}{b}>\frac{c}{d} \Longleftrightarrow-\frac{c}{d}>-\frac{a}{b}.
                 \]
\end{enumerate}
\end{enumerate}

 
\end{document}
