\documentclass[14pt]{extarticle}
\usepackage{extsizes}
\usepackage{enumerate}
\usepackage[english,activeacute]{babel}
\usepackage[latin1]{inputenc}
\usepackage{amsmath}
\usepackage{amsfonts}
\usepackage{amssymb,times}
\usepackage{graphicx}
\usepackage{latexsym}
\usepackage{color}
\usepackage{pifont}
\usepackage{bbm}
\usepackage{multicol}
\usepackage{appendix}
%\usepackage{dsfont}
\usepackage{mathrsfs}
%\spanishdecimal{.}
\usepackage[sort&compress]{natbib}
\bibliographystyle{plainnat}
%\usepackage{suthesis-2e}
%\usepackage{verbatim}
%\spanishdecimal{.}
\def\theequation{\thesection.\arabic{equation}}

\textwidth=6.5in
%\lineskip .25cm
%\lineskiplimit .25cm
\textheight=9.5in
\topmargin=-.75in
\topskip=-4pt
\evensidemargin=-2pt
\oddsidemargin=-1pt
%%%%%%%%%%%%%%%%%%%%
%\oddsidemargin 0in \textwidth 6.75in \topmargin 0in \textheight
%8.5in
\parindent 0em
\parskip 1ex
\newcommand{\floor}[1]{\left\lfloor {#1} \right\rfloor}

\def\CC{\mathbb{C}}
\newcommand{\rr}{\mathbb{R}}
\newcommand{\pai}{\left(}
\newcommand{\pad}{\right)}
\newcommand{\ci}{\left[}
\newcommand{\cd}{\right]}
\newcommand{\nn}{\mathbb{N}}
\newcommand{\Z}{\mathbb{Z}}
\newcommand{\p}{\mathbb{P}}
\newcommand{\E}{\mathbb{E}}
\newcommand{\lphi}{\widehat{\phi}}
\newcommand{\D}{\mathfrak{D}}
%\renewcommand{\qedsymbol}{\rule{1ex}{1ex}}

\begin{document}
 \begin{flushright}
  \begin{tabular}{|c|c|c|c|c|c|c|c|c|c|c|c|c|c|c|c|}
  \hline
  \ \ $E_1$\ \ &\ \ $E_2$\ \ &\ \ $E_3$\ \ &\ \ $E_4$\ \ &Calif.\\
  \hline
  & &  & & \\
    & & &&\\
   \hline
 \end{tabular}\\
 \end{flushright}
\begin{center}
\large
Matem�ticas Elementales
\normalsize

Agosto - Diciembre de 2025\\

Tarea 8
\end{center}

Resuelve los siguientes ejercicios seg�n el reglamento del curso y \textbf{entrega esta p�gina} con los nombres de los integrantes de equipo colocados en lugar de estas l�neas de instrucciones. 
Todas las demostraciones deben redactarse justificando formalmente todos los pasos necesarios.

La tarea debe resolverse \textbf{con la teor�a vista en el curso o con resultados a lo m�s de los cursos de preparatoria}. Est� prohibido utilizar resultados de otros cursos de este mismo semestre o de semestres posteriores. Tampoco est� permitido utilizar trucos de alguna olimpiada relacionada con computaci�n y/o matem�ticas.

Cada inciso distinto debe ser resuelto por un alumno distinto y, en cada caso, deber� colocarse el nombre de la persona que redact� el inciso.

\begin{enumerate}
\item Realiza lo siguiente:
\begin{enumerate}
\item Sea $A$ un conjunto no vac�o arbitrario y sea
$$R=\{(a,a): a\in A\}.$$

Demuestra que $R$ es una relaci�n de equivalencia sobre $A$ y que si $S$ es cualquier otra relaci�n de equivalencia sobre $A$, entonces $R\subseteq S$. Adem�s, halla la partici�n inducida por $R$.
\bigskip

\item Considera en $\rr$ la relaci\'on $R=\{(a,b) \in \rr \times \rr: a^2-b^2=a-b\}$. Demuestra que $R$ es una relaci\'on de equivalencia y halla la partici�n inducida por $R$.

\item Determina si la siguiente ``demostraci�n'' es correcta o no. En caso de que no sea correcta, halla el error.

\textbf{Proposici�n}. Sea $A$ un conjunto arbitrario no vac�o y sea $\sim$ una relaci�n binaria sobre $A$ que es sim�trica y transitiva, se cumple que $\sim$ es reflexiva.

\textbf{Prueba}: \textit{Sean $x,y\in A$ tales que $x\sim y$. Por simetr�a tenemos que $y\sim x$ y por transitividad obtenemos tambi�n que $x\sim x$}. $\blacksquare$
\end{enumerate}

 


\item  Sea $A=\Z \times (\Z - \{0\})$. Considera la relaci\'on en $A$ dada por 
$$R=\Bigg\{\Big((a,b),(c,d)\Big) \in A \times A: ad=bc \Bigg\}.$$

\begin{enumerate}
\item  Demuestra que $R$ es una relaci\'on de equivalencia y halla la partici�n inducida por ella.
\item Denotemos por $\sim$ a la relaci�n de equivalencia $R$ y sea

\begin{align*}
f: (A/\sim )\times (A/\sim)& \longrightarrow A/\sim \\
\big( [(a,b)], [(c,d)] \big) &\mapsto [(ad+bc,bd)].
\end{align*}

Demuestra que $f$ es una funci�n bien definida.

\item �Se cumple que $Rango(f)=A/\sim$ para $f$ como en el inciso anterior? Justifica formalmente tu respuesta.
\end{enumerate}


\item Sean $A=\nn \times \nn$ y 
$$R=\{((a,b),(c,d)) \in A \times A: a+d=b+c\}.$$

En todo el ejercicio se considerar� la relaci�n $R$.
\begin{enumerate}
\item Demuestra que $R$ es una relaci\'on de equivalencia. 

\item Denotemos por $\sim$ a la relaci�n $R$ y sea \begin{align*}
g: (A/\sim) \times (A/\sim) &\to (A/\sim) \\
\Big([(a,b)], [(c,d)] \Big) &\mapsto [(a \cdot c+b \cdot d,a \cdot d+ b \cdot c)].
\end{align*}
Determina si $g$ es una funci�n bien definida y si lo es, halla su rango.


\end{enumerate}

\item Demuestra que los siguientes conjuntos son numerables:
\begin{enumerate}
\item $\mathbb{Q}^n$.
\item $\Z^n$.
\item $\{ (a,b)\subseteq \rr: a<b, a,b\in\mathbb{Q}\}$.
\item $\{q\in\mathbb{Q}: q\in [0,1]\}$.
\end{enumerate}

\end{enumerate}
\end{document}
