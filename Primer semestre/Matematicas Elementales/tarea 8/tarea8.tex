\documentclass[12pt,a4paper]{article}
\newcommand{\grupo}{}
% ------------------------------------------------------
% Idioma y codificación
% ------------------------------------------------------
\usepackage[spanish]{babel}
\usepackage[utf8]{inputenc}
\usepackage[T1]{fontenc}

% ------------------------------------------------------
% Tipografía y formato
% ------------------------------------------------------
\usepackage{lmodern}    % Fuente Latin Modern
\usepackage{microtype}  % Mejor espaciado
\usepackage{xcolor}     % Colores
\definecolor{UGblue}{RGB}{0,33,71}
\definecolor{UGgold}{RGB}{212,175,55}
\usepackage{graphicx}
\usepackage{setspace}
\usepackage{titlesec}
% Añade esto en el preámbulo (antes de \begin{document})
\usepackage{bookmark}
\bookmarksetup{open,numbered}
\usepackage{tikz}
\usetikzlibrary{calc,decorations.pathreplacing,angles,quotes}
\usepackage{pgfplots}
\pgfplotsset{compat=1.18}
\usepackage{caption}

% ------------------------------------------------------
% Matemáticas
% ------------------------------------------------------
\usepackage{amsmath,amssymb,amsthm,mathtools}
\usepackage{mathrsfs}
\usepackage{stmaryrd}
\usepackage{physics}

% ------------------------------------------------------
% Otros paquetes útiles
% ------------------------------------------------------
\usepackage{enumitem}
\usepackage{csquotes}
\usepackage{hyperref}
\hypersetup{
    colorlinks=true,
    linkcolor=UGblue,
    urlcolor=UGgold,
    citecolor=UGblue
}
\usepackage{geometry}
\geometry{margin=2.5cm}
\usepackage{multicol}

% ------------------------------------------------------
% Encabezados y pies de página
% ------------------------------------------------------
% ------------------------------------------------------
% Teoremas, definiciones, etc.
% ------------------------------------------------------
\theoremstyle{plain}
\newtheorem{teo}{Teorema}[section]
\newtheorem{prop}[teo]{Proposición}
\newtheorem{lema}[teo]{Lema}
\theoremstyle{definition}
\newtheorem{defi}[teo]{Definición}
\theoremstyle{remark}
\newtheorem{ejemplo}[teo]{Ejemplo}
\newtheorem{obs}[teo]{Observación}

% ------------------------------------------------------
% Comandos útiles
% ------------------------------------------------------
\newcommand{\N}{\mathbb{N}}
\newcommand{\Z}{\mathbb{Z}}
\newcommand{\Q}{\mathbb{Q}}
\newcommand{\R}{\mathbb{R}}
\newcommand{\C}{\mathbb{C}}
\newcommand{\pfrac}[2]{\left( \frac{#1}{#2} \right)} %parentesis
\newcommand{\bfrac}[2]{\left[ \frac{#1}{#2} \right]} %corchetes
\newcommand{\vecpar}[1]{\left( #1 \right)}
% En el preámbulo
\newcommand{\point}[2]{\left( #1, #2 \right)}
\newcommand{\pointfrac}[4]{\left( \frac{#1}{#2}, \frac{#3}{#4} \right)}
\newcommand{\pointmix}[4]{\left( #1, \frac{#3}{#4} \right)} % Primera coordenada normal, segunda fracción
\renewcommand{\qedsymbol}{$\blacksquare$}
\newcommand{\gt}{\ensuremath{>}}
\newcommand{\lt}{\ensuremath{<}}
\renewcommand{\div}{$÷$}

\renewcommand{\baselinestretch}{1.2}
\setlength{\parindent}{0pt}

\newenvironment{solucion}
{\par\noindent\textbf{Solución.}\ }
{\hfill$\blacktriangleleft$\par}

% ------------------------------------------------------
% Datos personales (modificar según tarea)
% ------------------------------------------------------
\newcommand{\alumno}{Ricardo León Martínez}
\newcommand{\materia}{Elementos de Geometría}
\newcommand{\profesor}{Valentina Muñoz Porras}
\newcommand{\tarea}{Tarea 5}
\newcommand{\fecha}{22/09/2025}

\setlength{\parindent}{0pt}

\begin{document}
    \textbf{Ejercicio 1, Inciso (a)}\\
    Sea $A$ un conjunto no vacio 
    arbitrario y sea
    \begin{align*}
        R={(a,a):a\in A}.
    \end{align*}
    Demuestra que $R$ es una relacion de equivalencia sobre $A$ y que si
    $S$ es cualquier otra relacion de equivalencia sobre $A$, entonces
    $R\subseteq S$. Ademas, halla la particion inducida por $R$.
    \begin{proof}
        Primero, veamos que $R$ es una relacion de equivalencia. Para 
        todo $a \in A$, $(a, a) \in R$ por definición de $R$, por lo que
        $R$ es reflexiva. Supongamos $(a, b) \in R$. Por definición
    de $R$, $a = b$. Entonces, por analogía con el caso anterior, 
    $(b, a) = (a, a) \in R$. Así, $R$ es simétrica.
        Supongamos $(a, b) \in R$ y $(b, c)
    \in R$. Por definición de $R$, $a = b$ y $b = c$, luego $a = c$.
    Entonces $(a, c) = (a, a) \in R$. Con esto hemos probado que 
    $R$ es transitiva. Asi, hemos probado que $R$ es una relacion
        de equivalencia.\\
        Ahora sea $S$ una relacion de equivalencia sobre $A$, como $S$
        es reflexiva, para todo $a\in A$, $(a,a)\in S$. Pero, $R=
        \{(a,a):a\in A\}$, luego $R\subseteq S$.\\
        Por ultimo la clase de equivalencia de $a\in A$ es
        \begin{align*}
            [a]=\{x\in A,(a,x)\in R\}
        \end{align*}
        Pero $(a,x)\in\mathbb{R}$ implica $a=x$, luego $[a]=\{a\}$. Por
        lo que la particion inducida por $R$ es
        \begin{align*}
            \mathcal{P}=\{\{a\}:a\in A\}.
        \end{align*}
    \end{proof}
    \textbf{Ejercicio 2, inciso (a)}\\
    Sea $A=\mathbb{Z}\times(\mathbb{Z}\setminus{0})$.
    Considera la realacion en $A$ dada por
    \begin{align*}
        R=\{((a,b),(c,d))\in A\times A:ad=bc\}
    \end{align*}
    Demuestra que $R$ es una relacion de equivalencia y halla la particion
    inducida por ella.
    \begin{proof}
        Sea $(a,b)\in R$, tenemos que mostrar que $((a,b),(a,b))\in R$. La
        condicion es $ab=ba$ lo cual es cierto por la conmutatividad 
        del producto en $\mathbb{Z}$ por lo tanto $R$ es reflexiva.
        Ahora supongamos que $((a,b),(c,d))\in\mathbb{R}$
        es decir, $ad=bc$, queremos ver que $cb=da$ pero $ad=bc$ implica
        $cb=da$ por la conmutatividad del producto en $\mathbb{Z}$, asi, $R$
        es simetrica. Por ultimo supongamos $((a,b),(c,d))\in\mathbb{R}$ y
        $((c,d),(e,f))\in\mathbb{R}$ es decir $ad=bc$ y $cf=dc$. De $ad=bc$
        tenemos $adf=bcf$ y de igual forma de $cf=dc$ se tiene que $bcf=bdc$
        entonces $adf=bdc$ con $d\neq0$, cancelamos d, asi teniendo $af=bc$
        por lo tanto $R$ es transitiva.
        Asi, $R$ es una relacion de equivalencia.\\
        Por último, la clase de equivalencia de $(a,b)\in A$ es
        \begin{align*}
        [(a,b)] = \{(c,d)\in A : a d = b c\}.
        \end{align*}
            Por lo que la partición inducida por $R$ es
            \begin{align*}
                \mathcal{P} = \big\{\,\{\{(a,b) \in \mathbb{N}\times\mathbb{N} : \frac{c}{d}=r\}:r\in\mathbb{Q}\}\big\}.
            \end{align*}
    \end{proof}
    \textbf{Ejercicio 3, inciso (a)}\\
    Sean $A=\mathbb{N}\times\mathbb{N}$ y
    \begin{align*}
        R=\{((a,b),(c,d))\in A\times A:a+d=b+c\}
    \end{align*}
    demuestra que $R$ es una relacion de equivalencia.
    \begin{proof}
        Primero, veamos que $R$ es una relación de equivalencia.
        Sea $(a,b) \in A$. Entonces
            \begin{align*}
                a+b = b+a,                
            \end{align*}
        por la conmutatividad de la suma en $\mathbb{N}$, por lo que $((a,b),(a,b)) \in R$. Así, $R$ es reflexiva.
        Ahora supongamos que $((a,b),(c,d)) \in R$, es decir,
            \begin{align*}
                a + d = b + c.
            \end{align*}
        Reordenando términos se obtiene
            \begin{align*}
                c + b = d + a,
            \end{align*}
        lo cual muestra que $((c,d),(a,b)) \in R$, por lo tanto $R$ es simétrica.
        Por último, supongamos que $((a,b),(c,d)) \in R$ y $((c,d),(e,f)) \in R$, es decir,
            \begin{align*}
                a + d = b + c \qquad\text{y}\qquad c + f = d + e.
            \end{align*}
        sumando se tiene que
            \begin{align*}
                (a+d)+(c+f)=(b+c)+(d+e)
            \end{align*}
        Por la asociatividad de la suma en $\mathbb{N}$ se sigue que
            \begin{align*}
                (a+f)+(d+c)=(b+c)+(d+c).
            \end{align*}
        De esta manera
            \begin{align*}
                a + f = b + e,
            \end{align*}
        por lo que $((a,b),(e,f)) \in R$. Así, $R$ es transitiva.
        Con esto hemos probado que $R$ es una relación de equivalencia.
        Ahora, la clase de equivalencia de $(a,b) \in A$ es
            \begin{align*}
                [(a,b)] = \{(c,d) \in A : a + d = b + c\}.
            \end{align*}
        La partición inducida por $R$ es entonces
            \begin{align*}
                \mathcal{P} = \big\{\,\{\{(a,b) \in \mathbb{N}\times\mathbb{N} : a-b=k\}:k\in\mathbb{Z}\}\big\}.
            \end{align*}
    \end{proof}
    \textbf{Ejercicio 4, inciso a}\\
    Demuestra que $\mathbb{Q}^{n}$ es numerable.
    \begin{proof}
        Procedamos por induccion. El caso $n=1$ es claro ya que $\mathbb{Q}$
        es numerable. Supongamos para $n=k$. Probemos para $n=k+1$. Notemos que
    \[
        \mathbb{Q}^{k}\times\mathbb{Q}=\mathbb{Q}^{k+1}.
    \]
    Por el lema 1 se sigue que es verdadero para $n=k+1$. Por lo tanto
    $\mathbb{Q}^{n}$ es numerable.
    \end{proof}
\end{document}