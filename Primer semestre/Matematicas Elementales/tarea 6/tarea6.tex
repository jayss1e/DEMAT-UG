\documentclass[12pt,a4paper]{article}
\newcommand{\grupo}{}
% ------------------------------------------------------
% Idioma y codificación
% ------------------------------------------------------
\usepackage[spanish]{babel}
\usepackage[utf8]{inputenc}
\usepackage[T1]{fontenc}

% ------------------------------------------------------
% Tipografía y formato
% ------------------------------------------------------
\usepackage{lmodern}    % Fuente Latin Modern
\usepackage{microtype}  % Mejor espaciado
\usepackage{xcolor}     % Colores
\definecolor{UGblue}{RGB}{0,33,71}
\definecolor{UGgold}{RGB}{212,175,55}
\usepackage{graphicx}
\usepackage{setspace}
\usepackage{titlesec}
% Añade esto en el preámbulo (antes de \begin{document})
\usepackage{bookmark}
\bookmarksetup{open,numbered}
\usepackage{tikz}
\usetikzlibrary{calc,decorations.pathreplacing,angles,quotes}
\usepackage{pgfplots}
\pgfplotsset{compat=1.18}
\usepackage{caption}

% ------------------------------------------------------
% Matemáticas
% ------------------------------------------------------
\usepackage{amsmath,amssymb,amsthm,mathtools}
\usepackage{mathrsfs}
\usepackage{stmaryrd}
\usepackage{physics}

% ------------------------------------------------------
% Otros paquetes útiles
% ------------------------------------------------------
\usepackage{enumitem}
\usepackage{csquotes}
\usepackage{hyperref}
\hypersetup{
    colorlinks=true,
    linkcolor=UGblue,
    urlcolor=UGgold,
    citecolor=UGblue
}
\usepackage{geometry}
\geometry{margin=2.5cm}
\usepackage{multicol}

% ------------------------------------------------------
% Encabezados y pies de página
% ------------------------------------------------------

% ------------------------------------------------------
% Teoremas, definiciones, etc.
% ------------------------------------------------------
\theoremstyle{plain}
\newtheorem{teo}{Teorema}[section]
\newtheorem{prop}[teo]{Proposición}
\newtheorem{lema}[teo]{Lema}
\theoremstyle{definition}
\newtheorem{defi}[teo]{Definición}
\theoremstyle{remark}
\newtheorem{ejemplo}[teo]{Ejemplo}
\newtheorem{obs}[teo]{Observación}

% ------------------------------------------------------
% Comandos útiles
% ------------------------------------------------------
\newcommand{\N}{\mathbb{N}}
\newcommand{\Z}{\mathbb{Z}}
\newcommand{\Q}{\mathbb{Q}}
\newcommand{\R}{\mathbb{R}}
\newcommand{\C}{\mathbb{C}}

\newcommand{\vecpar}[1]{\left( #1 \right)}
\renewcommand{\qedsymbol}{$\blacksquare$}
\newcommand{\gt}{\ensuremath{>}}
\newcommand{\lt}{\ensuremath{<}}
\renewcommand{\div}{$÷$}

\renewcommand{\baselinestretch}{1.2}
\setlength{\parindent}{0pt}

\newenvironment{solucion}
{\par\noindent\textbf{Solución.}\ }
{\hfill$\blacktriangleleft$\par}

\begin{document}
\title{Tarea 6, Matemáticas elementales\\Ejercicio 1 inciso a), Ejericio 3 inciso a), Ejercicio 4, inciso a)}
\author{Said Huizar Dorantes. LIC. EN MATEMÁTICAS\\
        Axel Magaña Falcón. LIC. EN COMPUTACIÓN MATEMÁTICA\\
        Egny Alejandrina Madrid. LIC. EN MATEMÁTICAS\\
        Ricardo León Martínez. LIC. EN MATEMÁTICAS}
\date{28 de octubre de 2025}
\maketitle

\textbf{Ejercicio 1 Inciso a)} \\
Sea $f:\mathbb{R}\to\mathbb{R}$ dada por $f(x)=(x-1)^2$. Halla el rango de $f$.
\begin{solucion}
    Sea $f:\mathbb{R}\to\mathbb{R}$ definida por $f(x)=(x-1)^2$. 
Para determinar su rango, buscamos todos los valores reales $y$ que puede tomar la función. 
Dado que el cuadrado de cualquier número real es siempre no negativo, se tiene 
$(x-1)^2 \ge 0$ para todo $x \in \mathbb{R}$, de modo que $f(x) \ge 0$ en todo punto de su dominio. 
Por lo tanto, todos los valores de $f$ pertenecen al conjunto $[0,\infty)$.  

Para comprobar que cada número real no negativo se alcanza efectivamente, 
consideremos un $y \ge 0$ arbitrario y resolvamos la ecuación 
\[
y = (x - 1)^2.
\]
Al extraer raíz cuadrada en ambos lados se obtiene 
$x - 1 = \pm \sqrt{y}$, y por tanto 
$x = 1 \pm \sqrt{y}$, que son números reales para todo $y \ge 0$. 
Esto muestra que, para cualquier $y \ge 0$, existe al menos un $x \in \mathbb{R}$ tal que $f(x)=y$.  

En consecuencia, el conjunto de valores posibles de $f$ es 
\[
R_f = [0,\infty),
\]
por lo que el rango de la función está dado por $[0,\infty)$. 
\end{solucion}

\textbf{Ejercicio 3 Inciso a)}\\
Sean $f:A\to B$ y $A^\prime\subset A$, $B^\prime\subset B$. Demuestra que si $f$ es inyectiva,
entonces
\[
    f^{-1}(f(A^\prime))=A^\prime.
\]
\begin{proof}
    Veamos primero que $A^\prime\subset f^{-1}(f(A^\prime))$. Sea $a\in A^\prime$.
    Entonces $f(a)\in f(A^\prime)$ por definicion de imagen directa. Por definicion de imagen
    inversa, $a\in f^{-1}(f(A^\prime))$ puesto que $f(a)\in f(A^\prime)$. Como $a$ fue 
    arbitrario, se tiene $A^\prime\subset f^{-1}(f(A^\prime))$.\\
    Ahora veamos que $f^{-1}(f(A^\prime))\subset A^\prime$. Sea $x\in f^{-1}(f(A^\prime))$.
    Entonces $f(x)\in f(A^\prime)$. Por definicion de $f(A^\prime)$ existe $a^\prime\in A^\prime$
    tal que $f(x)=f(a^\prime)$. Como $f$ es inyectiva, de $f(x)=f(a^\prime)$ se deduce $x=a^\prime$.
    Por tanto $x\in A^\prime$. Como $x$ fue arbitrario, resulta $f^{-1}(f(A^\prime))\subset A^\prime$.
    Por doble contencion se obtiene
    \begin{align*}
        f^{-1}(f(A^\prime))=A^\prime.
    \end{align*}
\end{proof}

\textbf{Ejercicio 4 Inciso a)}\\
Sea $A$ un conjunto no vacio. Demuestra por definicion que
\begin{align*}
    \{\{x\}:x\in A\},
\end{align*}
es una particion de A.
\begin{proof}
    Sea $\Omega = A$ un conjunto no vacío y consideremos la clase 
$\mathcal{F} = \{A_x : x \in A\}$, donde para cada $x \in A$ se define 
$A_x = \{x\}$. Nótese primero que, por construcción, cada conjunto $A_x$ 
es no vacío, pues contiene al elemento $x$. Si tomamos dos índices distintos 
$x, y \in A$ con $x \neq y$, entonces 
$A_x \cap A_y = \{x\} \cap \{y\} = \varnothing$, 
de modo que los conjuntos de la clase son disjuntos dos a dos. 
Por otra parte, dado que para todo $a \in A$ se cumple 
$a \in \{a\} = A_a$, se tiene que 
\[
\bigcup_{x \in A} A_x = A = \Omega.
\]
En consecuencia, la clase 
$\mathcal{F} = \{A_x : x \in A\}$ 
constituye una partición de $\Omega$.

\end{proof}



\end{document}