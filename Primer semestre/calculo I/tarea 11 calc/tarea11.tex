\documentclass[12pt,a4paper]{article}
\newcommand{\grupo}{}
% ------------------------------------------------------
% Idioma y codificación
% ------------------------------------------------------
\usepackage[spanish]{babel}
\usepackage[utf8]{inputenc}
\usepackage[T1]{fontenc}

% ------------------------------------------------------
% Tipografía y formato
% ------------------------------------------------------
\usepackage{lmodern}    % Fuente Latin Modern
\usepackage{microtype}  % Mejor espaciado
\usepackage{xcolor}     % Colores
\definecolor{UGblue}{RGB}{0,33,71}
\definecolor{UGgold}{RGB}{212,175,55}
\usepackage{graphicx}
\usepackage{setspace}
\usepackage{titlesec}
% Añade esto en el preámbulo (antes de \begin{document})
\usepackage{bookmark}
\bookmarksetup{open,numbered}
\usepackage{tikz}
\usetikzlibrary{calc,decorations.pathreplacing,angles,quotes}
\usepackage{pgfplots}
\pgfplotsset{compat=1.18}
\usepackage{caption}

% ------------------------------------------------------
% Matemáticas
% ------------------------------------------------------
\usepackage{amsmath,amssymb,amsthm,mathtools}
\usepackage{mathrsfs}
\usepackage{stmaryrd}
\usepackage{physics}

% ------------------------------------------------------
% Otros paquetes útiles
% ------------------------------------------------------
\usepackage{enumitem}
\usepackage{csquotes}
\usepackage{hyperref}
\hypersetup{
    colorlinks=true,
    linkcolor=UGblue,
    urlcolor=UGgold,
    citecolor=UGblue
}
\usepackage{geometry}
\geometry{margin=2.5cm}
\usepackage{multicol}

% ------------------------------------------------------
% Encabezados y pies de página
% ------------------------------------------------------
\usepackage{fancyhdr}
\pagestyle{fancy}
\fancyhf{}
\lhead{Elementos de Geometría}
\rhead{Ricardo Leon Martinez}
\cfoot{\thepage}
% ------------------------------------------------------
% Teoremas, definiciones, etc.
% ------------------------------------------------------
\theoremstyle{plain}
\newtheorem{teo}{Teorema}[section]
\newtheorem{prop}[teo]{Proposición}
\newtheorem{lema}[teo]{Lema}
\theoremstyle{definition}
\newtheorem{defi}[teo]{Definición}
\theoremstyle{remark}
\newtheorem{ejemplo}[teo]{Ejemplo}
\newtheorem{obs}[teo]{Observación}

% ------------------------------------------------------
% graficas
% ------------------------------------------------------
\usepackage{graphicx}
\usepackage{caption}
\usepackage{subcaption}
\usepackage{pgfplots}
\pgfplotsset{compat=1.18}
\usepackage{siunitx}
% ------------------------------------------------------
% Comandos útiles
% ------------------------------------------------------
\newcommand{\N}{\mathbb{N}}
\newcommand{\Z}{\mathbb{Z}}
\newcommand{\Q}{\mathbb{Q}}
\newcommand{\R}{\mathbb{R}}
\newcommand{\C}{\mathbb{C}}
\newcommand{\pfrac}[2]{\left( \frac{#1}{#2} \right)} %parentesis
\newcommand{\bfrac}[2]{\left[ \frac{#1}{#2} \right]} %corchetes
\newcommand{\vecpar}[1]{\left( #1 \right)}
% En el preámbulo
\newcommand{\point}[2]{\left( #1, #2 \right)}
\newcommand{\pointfrac}[4]{\left( \frac{#1}{#2}, \frac{#3}{#4} \right)}
\newcommand{\pointmix}[4]{\left( #1, \frac{#3}{#4} \right)} % Primera coordenada normal, segunda fracción
\renewcommand{\qedsymbol}{$\blacksquare$}
\newcommand{\gt}{\ensuremath{>}}
\newcommand{\lt}{\ensuremath{<}}
\renewcommand{\div}{$÷$}

\renewcommand{\baselinestretch}{1.2}
\setlength{\parindent}{0pt}

\newenvironment{solucion}
{\par\noindent\textbf{Solución.}\ }
{\hfill$\blacktriangleleft$\par}

% ------------------------------------------------------
% Datos personales (modificar según tarea)
% ------------------------------------------------------
\newcommand{\alumno}{Ricardo León Martínez}
\newcommand{\materia}{Calculo Diferencial e Integral I}
\newcommand{\profesor}{Fernando Nuñez Medina}
\newcommand{\tarea}{Tarea 11}
\newcommand{\fecha}{06/11/2025}

\setlength{\parindent}{0pt}

\begin{document}
    \begin{center}
        {\large\textbf{UNIVERSIDAD DE GUANAJUATO}}\\[0.3cm]
        {\normalsize\textbf{DIVISIÓN DE CIENCIAS NATURALES Y EXACTAS}}\\
        {\normalsize\textbf{CAMPUS GUANAJUATO}}\\[1cm]

        {\Large\textbf{\tarea\ (\materia)}}\\[1cm]
    \end{center}
    \textbf{Nombre:} \alumno \hfill 
    \textbf{Fecha:} \fecha \hfill 
    \textbf{Calificación:} \rule{3cm}{0.4pt} \\[0.3cm]
    \begin{enumerate}[left=0pt, label=\textbf{\arabic*.}]
        \item Prueba las leyes de los exponentes cuando $m$ y $n$ son numeros racionales.
        \item \textbf{(Funciones trigonometricas Inversas)} Por el criterio de la recta
        horizontal para determinar si una funcion es 1-1, es claro que las funciones trigonometricas
        no son funciones 1-1 y, en consecuancia, no son invertibles, sin embargo, restringiremos
        su dominio para que si lo sean. Las restricciones que se suelen utilizar  se muestran
        en el cuadro siguiente \\
        \begin{center}
            \begin{tabular}{|c|c|c|}
                \hline
                \textbf{Funcion} & \textbf{Restriccion} & \textbf{Rango de la restriccion} \\
                \hline
                $\sin(x)$ & $[-\pi/2,\pi/2]$ & $[-1,1]$ \\
                \hline
                $\cos(x)$ & $[0,\pi]$ & $[-1,1]$ \\
                \hline
                $\tan(x)$ & $(-\pi/2,\pi/2)$ & $\mathbb{R}$ \\
                \hline
                $\cot(x)$ & $(0,\pi)$ & $\mathbb{R}$ \\
                \hline
                $\sec(x)$ & $[0,\pi/2)\cup(\pi/2,\pi]$ & $(-\infty,-1]\cup[1,\infty)$ \\
                \hline
                $\csc(x)$ & $[-\pi/2,0)\cup(0,\pi/2]$ & $(-\infty,-1]\cup[1,\infty)$ \\
                \hline
            \end{tabular}
        \end{center}
        Las restricciones de las funciones trigonometricas señaladas en el cuadro anterior son
        1-1 y sobre (su rango) y, en consecuencia, invertibles. A las funciones inversas de las
        restricciones señaladas en el cuadro anterior se les llama arcoseno, arcocoseno, arcotangente,
        arcocotangente, arcosecante, arcocosecante, y se les denota por $\arcsin(x)$, $\arccos(x)$,
        $\arctan(x)$, $\arccot(x)$, $\arcsec(x)$ y $\arccsc(x)$, respectivamente; son las llamadas
        funciones trigonometricas inversas.
        \begin{enumerate}[label=(\alph*)]
            \item Dibuja las graficas de las restricciones del cuadro anterior.
            \item A partir de las graficas del inciso (a) dibuja las graficas de las funciones
            trigonometricas inversas.
        \end{enumerate}
        \begin{solucion}
            (a) Gráficas de las restricciones.\\[2mm]
            % --- SIN(x) y COS(x) ---
            \begin{minipage}[b]{0.48\textwidth}
                \centering
                \begin{tikzpicture}
                    \begin{axis}[
                    domain=-1.7:1.7, samples=200,
                    xmin=-1.8, xmax=1.8, ymin=-1.2, ymax=1.2,
                    axis lines=middle, xlabel={$x$}, ylabel={$y$},
                    xtick={-1.5708,-0.7854,0,0.7854,1.5708},
                    xticklabels={\(-\tfrac{\pi}{2}\),\(-\tfrac{\pi}{4}\),0,\(\tfrac{\pi}{4}\),\(\tfrac{\pi}{2}\)},
                    ytick={-1,0,1}, title={\(\sin(x)\) en \([-\tfrac{\pi}{2},\tfrac{\pi}{2}]\)}
                    ]
                    \addplot+[thick]{sin(deg(x))};
                    \addplot[only marks,mark=*,mark size=1pt] coordinates {(-1.5708,-1) (1.5708,1)};
                    \draw[dashed](axis cs:-1.5708,-1.2)--(axis cs:-1.5708,1.2);
                    \draw[dashed](axis cs:1.5708,-1.2)--(axis cs:1.5708,1.2);
                    \end{axis}
                \end{tikzpicture}
            \end{minipage}
            \hfill
            \begin{minipage}[b]{0.48\textwidth}
                \centering
                \begin{tikzpicture}
                    \begin{axis}[
                    domain=0:3.1416,samples=200,
                    xmin=-0.2,xmax=3.3,ymin=-1.2,ymax=1.2,
                    axis lines=middle,xlabel={$x$},ylabel={$y$},
                    xtick={0,1.5708,3.1416},
                    xticklabels={0,\(\tfrac{\pi}{2}\),\(\pi\)},
                    ytick={-1,0,1},title={\(\cos(x)\) en \([0,\pi]\)}
                    ]
                    \addplot+[thick]{cos(deg(x))};
                    \addplot[only marks,mark=*,mark size=1pt] coordinates {(0,1)(3.1416,-1)};
                    \draw[dashed](axis cs:0,-1.2)--(axis cs:0,1.2);
                    \draw[dashed](axis cs:3.1416,-1.2)--(axis cs:3.1416,1.2);
                    \end{axis}
                \end{tikzpicture}
            \end{minipage}

            \vspace{5mm}

            % --- TAN(x) y COT(x) ---
            \begin{minipage}[b]{0.48\textwidth}
                \centering
                \begin{tikzpicture}
                    \begin{axis}[
                    domain=-1.4:1.4,samples=200,
                    xmin=-1.6,xmax=1.6,ymin=-6,ymax=6,
                    axis lines=middle,xlabel={$x$},ylabel={$y$},
                    xtick={-1.5708,-0.7854,0,0.7854,1.5708},
                    xticklabels={\(-\tfrac{\pi}{2}\),\(-\tfrac{\pi}{4}\),0,\(\tfrac{\pi}{4}\),\(\tfrac{\pi}{2}\)},
                    title={\(\tan(x)\) en \((-\tfrac{\pi}{2},\tfrac{\pi}{2})\)}
                    ]
                    \addplot+[thick]{tan(deg(x))};
                    \draw[dashed](axis cs:-1.5708,-6)--(axis cs:-1.5708,6);
                    \draw[dashed](axis cs:1.5708,-6)--(axis cs:1.5708,6);
                    \end{axis}
                \end{tikzpicture}
            \end{minipage}
            \hfill
            \begin{minipage}[b]{0.48\textwidth}
                \centering
                \begin{tikzpicture}
                    \begin{axis}[
                    domain=0.01:3.1316,samples=200,
                    xmin=-0.1,xmax=3.2,ymin=-6,ymax=6,
                    axis lines=middle,xlabel={$x$},ylabel={$y$},
                    xtick={0,0.7854,1.5708,2.3562,3.1416},
                    xticklabels={0,\(\tfrac{\pi}{4}\),\(\tfrac{\pi}{2}\),\(\tfrac{3\pi}{4}\),\(\pi\)},
                    title={\(\cot(x)\) en \((0,\pi)\)}
                    ]
                    \addplot+[thick]{cos(deg(x))/sin(deg(x))};
                    \draw[dashed](axis cs:0,-6)--(axis cs:0,6);
                    \draw[dashed](axis cs:3.1416,-6)--(axis cs:3.1416,6);
                    \end{axis}
                \end{tikzpicture}
            \end{minipage}

            \vspace{5mm}

            % --- SEC(x) y CSC(x) ---
            \begin{minipage}[b]{0.48\textwidth}
                \centering
                \begin{tikzpicture}
                    \begin{axis}[
                    xmin=-0.1,xmax=3.2,ymin=-6,ymax=6,
                    axis lines=middle,xlabel={$x$},ylabel={$y$},
                    xtick={0,0.7854,1.5708,2.3562,3.1416},
                    xticklabels={0,\(\tfrac{\pi}{4}\),\(\tfrac{\pi}{2}\),\(\tfrac{3\pi}{4}\),\(\pi\)},
                    title={\(\sec(x)\) en \([0,\tfrac{\pi}{2})\cup(\tfrac{\pi}{2},\pi]\)}
                    ]
                    \addplot+[thick,domain=0:1.47]{1/cos(deg(x))};
                    \addplot+[thick,domain=1.671:3.1416]{1/cos(deg(x))};
                    \draw[dashed](axis cs:1.5708,-6)--(axis cs:1.5708,6);
                    \end{axis}
                \end{tikzpicture}
            \end{minipage}
            \hfill
            \begin{minipage}[b]{0.48\textwidth}
                \centering
                \begin{tikzpicture}
                    \begin{axis}[
                    domain=-1.4708:1.4708,samples=250,
                    xmin=-1.7,xmax=1.7,ymin=-6,ymax=6,
                    axis lines=middle,xlabel={$x$},ylabel={$y$},
                    xtick={-1.5708,-0.7854,0,0.7854,1.5708},
                    xticklabels={\(-\tfrac{\pi}{2}\),\(-\tfrac{\pi}{4}\),0,\(\tfrac{\pi}{4}\),\(\tfrac{\pi}{2}\)},
                    title={\(\csc(x)\) en \([-\tfrac{\pi}{2},0)\cup(0,\tfrac{\pi}{2}]\)}
                    ]
                    \addplot+[thick,domain=-1.4708:-0.01]{1/sin(deg(x))};
                    \addplot+[thick,domain=0.01:1.4708]{1/sin(deg(x))};
                    \draw[dashed](axis cs:0,-6)--(axis cs:0,6);
                    \end{axis}
                \end{tikzpicture}
            \end{minipage}

            \captionof{figure}{Gráficas de las funciones trigonométricas restringidas.}

            \vspace{1cm}
            (b) Gráficas de las funciones trigonométricas inversas.\\[2mm]

            \begin{center}
                % --- ARCSIN y ARCCOS ---
                \begin{minipage}[b]{0.48\textwidth}
                    \centering
                    \begin{tikzpicture}
                        \begin{axis}[
                        domain=-1:1,samples=200,
                        xmin=-1.2,xmax=1.2,ymin=-1.7,ymax=1.7,
                        axis lines=middle,xlabel={$x$},ylabel={$y$},
                        xtick={-1,0,1},ytick={-1.5708,0,1.5708},
                        yticklabels={\(-\tfrac{\pi}{2}\),0,\(\tfrac{\pi}{2}\)},
                        title={\(\operatorname{arcsin}(x)\)}
                        ]
                        \addplot+[thick]{asin(x)*pi/180};
                        \end{axis}
                    \end{tikzpicture}
                \end{minipage}
                \hfill
                \begin{minipage}[b]{0.48\textwidth}
                    \centering
                    \begin{tikzpicture}
                        \begin{axis}[
                        domain=-1:1,samples=200,
                        xmin=-1.2,xmax=1.2,ymin=-0.2,ymax=3.3,
                        axis lines=middle,xlabel={$x$},ylabel={$y$},
                        xtick={-1,0,1},ytick={0,1.5708,3.1416},
                        yticklabels={0,\(\tfrac{\pi}{2}\),\(\pi\)},
                        title={\(\operatorname{arccos}(x)\)}
                        ]
                        \addplot+[thick]{acos(x)*pi/180};
                        \end{axis}
                    \end{tikzpicture}
                \end{minipage}

                \vspace{5mm}

                % --- ARCTAN y ARCCOT ---
                \begin{minipage}[b]{0.48\textwidth}
                    \centering
                    \begin{tikzpicture}
                        \begin{axis}[
                        domain=-8:8,samples=400,
                        xmin=-8,xmax=8,ymin=-1.8,ymax=1.8,
                        axis lines=middle,xlabel={$x$},ylabel={$y$},
                        ytick={-1.5708,0,1.5708},
                        yticklabels={\(-\tfrac{\pi}{2}\),0,\(\tfrac{\pi}{2}\)},
                        title={\(\operatorname{arctan}(x)\)}
                        ]
                        \addplot+[thick]{atan(x)*pi/180};
                        \end{axis}
                    \end{tikzpicture}
                \end{minipage}
                \hfill
                \begin{minipage}[b]{0.48\textwidth}
                    \centering
                    \begin{tikzpicture}
                        \begin{axis}[
                        domain=-8:8,samples=400,
                        xmin=-8,xmax=8,ymin=-0.2,ymax=3.3,
                        axis lines=middle,xlabel={$x$},ylabel={$y$},
                        ytick={0,1.5708,3.1416},
                        yticklabels={0,\(\tfrac{\pi}{2}\),\(\pi\)},
                        title={\(\operatorname{arccot}(x)=\tfrac{\pi}{2}-\arctan(x)\)}
                        ]
                        \addplot+[thick]{pi/2 - atan(x)*pi/180};
                        \end{axis}
                    \end{tikzpicture}
                \end{minipage}

                \vspace{5mm}

                % --- ARCSEC y ARCCSC ---
                \begin{minipage}[b]{0.48\textwidth}
                    \centering
                    \begin{tikzpicture}
                        \begin{axis}[
                        xmin=-6,xmax=6,ymin=-0.1,ymax=3.3,
                        axis lines=middle,xlabel={$x$},ylabel={$y$},
                        xtick={-5,-2,-1,1,2,5},
                        ytick={0,0.7854,1.5708,2.3562,3.1416},
                        yticklabels={0,\(\tfrac{\pi}{4}\),\(\tfrac{\pi}{2}\),\(\tfrac{3\pi}{4}\),\(\pi\)},
                        title={\(\operatorname{arcsec}(x)=\arccos(1/x)\)}
                        ]
                        \addplot+[thick,domain=-6:-1]{acos(1/x)*pi/180};
                        \addplot+[thick,domain=1:6]{acos(1/x)*pi/180};
                        \end{axis}
                    \end{tikzpicture}
                \end{minipage}
                \hfill
                \begin{minipage}[b]{0.48\textwidth}
                    \centering
                    \begin{tikzpicture}
                        \begin{axis}[
                        xmin=-6,xmax=6,ymin=-1.7,ymax=1.7,
                        axis lines=middle,xlabel={$x$},ylabel={$y$},
                        xtick={-5,-2,-1,1,2,5},
                        ytick={-1.5708,-0.7854,0,0.7854,1.5708},
                        yticklabels={\(-\tfrac{\pi}{2}\),\(-\tfrac{\pi}{4}\),0,\(\tfrac{\pi}{4}\),\(\tfrac{\pi}{2}\)},
                        title={\(\operatorname{arccsc}(x)=\arcsin(1/x)\)}
                        ]
                        \addplot+[thick,domain=-6:-1]{asin(1/x)*pi/180};
                        \addplot+[thick,domain=1:6]{asin(1/x)*pi/180};
                        \end{axis}
                    \end{tikzpicture}
                \end{minipage}

                \captionof{figure}{Gráficas de las funciones trigonométricas inversas con sus dominios principales.}
            \end{center}
        \end{solucion}
        \vspace{1cm}
        \item Da un ejemplo de una funcion $f:\mathbb{R}\to\mathbb{R}$ que sea continua solo
        en un numero finito de puntos $p_1,\dots,p_n$.
        \item Prueba que
            \begin{align*}
                \derivative{}{x}\ln(\abs{x})=\frac{1}{x},\text{ }x\neq0.
            \end{align*}
        \item Prueba que si $f:\mathbb{R}\to\mathbb{R}$ tiene derivada acotada, entonces es
        uniformemente continua.
        \item Construye la grafica de una funcion cuya derivada siempre sea positiva.
        \item Sea $n$ un numero natural.
            \begin{enumerate}[label=(\alph*)]
                \item Prueba, utilizando los resultados de la seccion 5.8.1, que si $n$ es par,
                entonces la funcion $x^n$ es decreciente en $(-\infty,0]$ y es creciente en
                $[0,\infty)$.
                \item Prueba, utilizando los resultados de la seccion 5.8.1, que si $n$ es impar,
                entonces la funcion $x^n$ es creciente en todo $\mathbb{R}$.
            \end{enumerate}
        \item Analiza la grafica de la funcion $f(x)=x^3+3x^2$ siguiendo el procedimiento mencionado
        en la subseccion 5.8.5.
    \end{enumerate}
\end{document}