\documentclass[12pt,a4paper]{article}
\newcommand{\grupo}{}
% ------------------------------------------------------
% Idioma y codificación
% ------------------------------------------------------
\usepackage[spanish]{babel}
\usepackage[utf8]{inputenc}
\usepackage[T1]{fontenc}

% ------------------------------------------------------
% Tipografía y formato
% ------------------------------------------------------
\usepackage{lmodern}    % Fuente Latin Modern
\usepackage{microtype}  % Mejor espaciado
\usepackage{xcolor}     % Colores
\definecolor{UGblue}{RGB}{0,33,71}
\definecolor{UGgold}{RGB}{212,175,55}
\usepackage{graphicx}
\usepackage{setspace}
\usepackage{titlesec}
% Añade esto en el preámbulo (antes de \begin{document})
\usepackage{bookmark}
\bookmarksetup{open,numbered}
\usepackage{tikz}
\usetikzlibrary{calc,decorations.pathreplacing,angles,quotes}
\usepackage{pgfplots}
\pgfplotsset{compat=1.18}
\usepackage{caption}

% ------------------------------------------------------
% Matemáticas
% ------------------------------------------------------
\usepackage{amsmath,amssymb,amsthm,mathtools}
\usepackage{mathrsfs}
\usepackage{stmaryrd}
\usepackage{physics}

% ------------------------------------------------------
% Otros paquetes útiles
% ------------------------------------------------------
\usepackage{enumitem}
\usepackage{csquotes}
\usepackage{hyperref}
\hypersetup{
    colorlinks=true,
    linkcolor=UGblue,
    urlcolor=UGgold,
    citecolor=UGblue
}
\usepackage{geometry}
\geometry{margin=2.5cm}
\usepackage{multicol}

% ------------------------------------------------------
% Encabezados y pies de página
% ------------------------------------------------------
\usepackage{fancyhdr}
\setlength{\headheight}{14.49998pt}
\pagestyle{fancy}
\fancyhf{}
\lhead{Calculo Diferencial e Integral I}
\rhead{Ricardo Leon Martinez}
\cfoot{\thepage}
% ------------------------------------------------------
% Teoremas, definiciones, etc.
% ------------------------------------------------------
\theoremstyle{plain}
\newtheorem{teo}{Teorema}[section]
\newtheorem{prop}[teo]{Proposición}
\newtheorem{lema}[teo]{Lema}
\theoremstyle{definition}
\newtheorem{defi}[teo]{Definición}
\theoremstyle{remark}
\newtheorem{ejemplo}[teo]{Ejemplo}
\newtheorem{obs}[teo]{Observación}

% ------------------------------------------------------
% Comandos útiles
% ------------------------------------------------------
\newcommand{\N}{\mathbb{N}}
\newcommand{\Z}{\mathbb{Z}}
\newcommand{\Q}{\mathbb{Q}}
\newcommand{\R}{\mathbb{R}}
\newcommand{\C}{\mathbb{C}}
\newcommand{\pfrac}[2]{\left( \frac{#1}{#2} \right)} %parentesis
\newcommand{\bfrac}[2]{\left[ \frac{#1}{#2} \right]} %corchetes
\newcommand{\vecpar}[1]{\left( #1 \right)}
% En el preámbulo
\newcommand{\point}[2]{\left( #1, #2 \right)}
\newcommand{\pointfrac}[4]{\left( \frac{#1}{#2}, \frac{#3}{#4} \right)}
\newcommand{\pointmix}[4]{\left( #1, \frac{#3}{#4} \right)} % Primera coordenada normal, segunda fracción
\renewcommand{\qedsymbol}{$\blacksquare$}
\newcommand{\gt}{\ensuremath{>}}
\newcommand{\lt}{\ensuremath{<}}
\renewcommand{\div}{$÷$}

\renewcommand{\baselinestretch}{1.2}
\setlength{\parindent}{0pt}

\newenvironment{solucion}
{\par\noindent\textbf{Solución.}\ }
{\hfill$\blacktriangleleft$\par}

% ------------------------------------------------------
% Datos personales (modificar según tarea)
% ------------------------------------------------------
\newcommand{\alumno}{Ricardo León Martínez}
\newcommand{\materia}{Calculo Diferencia e Integral I}
\newcommand{\profesor}{Fernando Nuñez Medina}
\newcommand{\tarea}{Tarea 10}
\newcommand{\fecha}{31/10/2025}

\setlength{\parindent}{0pt}

\begin{document}
\begin{center}
    {\large\textbf{UNIVERSIDAD DE GUANAJUATO}}\\[0.3cm]
    {\normalsize\textbf{DIVISIÓN DE CIENCIAS NATURALES Y EXACTAS}}\\
    {\normalsize\textbf{CAMPUS GUANAJUATO}}\\[1cm]

    {\Large\textbf{\tarea\ (\materia)}}\\[1cm]
\end{center}

\textbf{Nombre:} \alumno \hfill 
\textbf{Fecha:} \fecha \hfill 
\textbf{Calificación:} \rule{3cm}{0.4pt} \\[0.3cm]

\begin{enumerate}[left=0pt, label=\textbf{\arabic*.}]
    \item \textbf{(Potencias racionales)} Prueba que el inciso (c) del ejercicio 1 de la tarea 9
    tambien se cumple si $m$ es un numero entero, es decir, prueba que si $x$ es un numero real,
    $m$ es un numero entero y $n$ es un numero natural tales que $(x^m)^{1/n}$ y $(x^{1/n})^m$
    estan definidos, entonces
        \begin{align*}
            (x^m)^{\frac{1}{n}}=(x^{\frac{1}{n}})^m.
        \end{align*}
    \begin{proof}
    Sean \( x \in \mathbb{R} \), \( m \in \mathbb{Z} \) y \( n \in \mathbb{N} \) tales que las expresiones \((x^m)^{1/n}\) y \((x^{1/n})^m\) estén definidas.

    \textbf{Caso 1: \( m \geq 0 \).}

    En este caso, \( m \) es un número natural o cero.

    Por definición de potencia entera positiva, tenemos
    \[
    x^m = \underbrace{x \cdot x \cdot \ldots \cdot x}_{m \text{ veces}}.
    \]
    Luego, aplicando la potencia racional \( 1/n \), se obtiene:
    \[
    (x^m)^{1/n} = \sqrt[n]{x^m}.
    \]
    Por otra parte,
    \[
    x^{1/n} = \sqrt[n]{x} \implies (x^{1/n})^m = (\sqrt[n]{x})^m = \underbrace{\sqrt[n]{x} \cdot \ldots \cdot \sqrt[n]{x}}_{m \text{ veces}} = \sqrt[n]{x^m}.
    \]
    Por lo tanto,
    \[
    (x^m)^{1/n} = (x^{1/n})^m \quad \text{para todo } m \ge 0.
    \]

    \textbf{Caso 2: \( m < 0 \).}

    Sea \( m = -k \) con \( k \in \mathbb{N} \).  
    Entonces:
    \[
    (x^m)^{1/n} = (x^{-k})^{1/n} = \left( \frac{1}{x^k} \right)^{1/n} = \frac{1}{(x^k)^{1/n}}.
    \]
    Por otra parte,
    \[
    (x^{1/n})^m = (x^{1/n})^{-k} = \frac{1}{(x^{1/n})^k}.
    \]
    Pero, del caso anterior (válido para \( k > 0 \)):
    \[
    (x^{1/n})^k = (x^k)^{1/n}.
    \]
    Por sustitución:
    \[
    (x^{1/n})^{-k} = \frac{1}{(x^k)^{1/n}}.
    \]
    Por tanto,
    \[
    (x^m)^{1/n} = (x^{1/n})^m.
    \]

    Esto es que, para todo \( x \in \mathbb{R} \), \( m \in \mathbb{Z} \) y \( n \in \mathbb{N} \) tales que ambas expresiones estén definidas, se cumple

    \[
    (x^m)^{1/n} = (x^{1/n})^m.
    \]
    \end{proof}
    \item Sea $f:I\to J$ una funcion sobre, donde $I$ y $J$ son intervalos. Prueba lo siguiente:
        \begin{enumerate}[label=(\alph*)]
            \item Si $f$ es creciente, entonces $f$ es invertible y $f^{-1}$ es creciente.
            \item Si $f$ es decreciente, entonces $f$ es invertible y $f^{-1}$ es decreciente.
        \end{enumerate}
    \begin{proof}
        Sea \(f:I\to J\) con \(I,J\) intervalos y \(J=f(I)\) el \emph{rango} de \(f\).

        \textbf{(a) \(f\) estrictamente creciente \(\Rightarrow\) \(f\) es invertible y \(f^{-1}\) es creciente.}

        \textit{Inyectividad.} Supongamos \(x_1,x_2\in I\) y \(f(x_1)=f(x_2)\). Si \(x_1\neq x_2\), sin pérdida de generalidad supongamos \(x_1<x_2\). Como \(f\) es creciente, de \(x_1<x_2\) se sigue \(f(x_1)<f(x_2)\), lo cual contradice \(f(x_1)=f(x_2)\). Por tanto \(x_1=x_2\). Así \(f\) es inyectiva. Como \(f\) es inyectiva y su imagen es \(J\), existe la función inversa \(f^{-1}:J\to I\).

        \textit{Monotonía de \(f^{-1}\).} Sean \(y_1,y_2\in J\) tales que \(y_1<y_2\). Definimos \(x_1=f^{-1}(y_1)\) y \(x_2=f^{-1}(y_2)\), de modo que \(f(x_1)=y_1\) y \(f(x_2)=y_2\). Supongamos, por contrapositiva, que \(x_1\ge x_2\). Si \(x_1=x_2\) entonces \(y_1=f(x_1)=f(x_2)=y_2\), contradicción con \(y_1<y_2\). Si \(x_1>x_2\), por la estricta creciente de \(f\) se tendría \(f(x_1)>f(x_2)\), es decir \(y_1>y_2\), nuevamente contradicción. Por tanto no puede suceder \(x_1\ge x_2\); necesariamente \(x_1<x_2\). Esto muestra que \(y_1<y_2\) implica \(f^{-1}(y_1)=x_1<f^{-1}(y_2)=x_2\). Es decir, \(f^{-1}\) es estrictamente creciente en \(J\).

        \textbf{(b) \(f\) estrictamente decreciente \(\Rightarrow\) \(f\) es invertible y \(f^{-1}\) es decreciente.}

        El argumento es análogo al caso (a) cambiando las desigualdades.

        \textit{Inyectividad.} Si \(x_1<x_2\) y \(f\) es decreciente entonces \(f(x_1)>f(x_2)\). Por tanto no pueden existir \(x_1\neq x_2\) con \(f(x_1)=f(x_2)\); así \(f\) es inyectiva y admite inversa \(f^{-1}:J\to I\).

        \textit{Monotonía de \(f^{-1}\).} Sean \(y_1,y_2\in J\) con \(y_1<y_2\) y \(x_1=f^{-1}(y_1),\ x_2=f^{-1}(y_2)\). Si se tuviera \(x_1\le x_2\) entonces por la decrecencia de \(f\) seguiría \(f(x_1)\ge f(x_2)\), es decir \(y_1\ge y_2\), contradicción. Por tanto necesariamente \(x_1>x_2\). Así \(y_1<y_2\) implica \(f^{-1}(y_1)>f^{-1}(y_2)\), esto es, \(f^{-1}\) es decreciente.

    \end{proof}
    \item Da un ejemplo de una funcion $f:\mathbb{R}\to\mathbb{R}$ que sea continua solo en un
    punto $x=0$,
    Definimos
    \[
    f(x)=
    \begin{cases}
    x, & x\in\mathbb{Q},\\[4pt]
    0, & x\in\mathbb{R}\setminus\mathbb{Q}.
    \end{cases}
    \]
    \begin{proof}
        Demostraremos que \(f\) es continua en \(0\) y que es discontinua en todo \(a\neq 0\).

        \textit{Continuidad en \(0\).} Sea \(\varepsilon>0\). Tomemos \(\delta=\varepsilon\). Si \(|x-0|<\delta\) entonces:

        - si \(x\in\mathbb{Q}\), \(f(x)=x\) y por tanto \(|f(x)-f(0)|=|x-0|<\delta=\varepsilon\);
        - si \(x\notin\mathbb{Q}\), \(f(x)=0\) y \(|f(x)-f(0)|=0<\varepsilon\).

        Así, para todo \(\varepsilon>0\) existe \(\delta=\varepsilon\) tal que \(|x-0|<\delta\Rightarrow |f(x)-f(0)|<\varepsilon\). Por lo tanto \(f\) es continua en \(0\).

        \textit{Discontinuidad en \(a\neq 0\).} Sea \(a\in\mathbb{R}\) con \(a\neq 0\). Consideramos dos casos:

        1. \(a\in\mathbb{Q}\). Entonces \(f(a)=a\neq 0\). Existe una sucesión de números irracionales \((y_n)\) tal que \(y_n\to a\). Para cada \(n\) se tiene \(f(y_n)=0\). Por tanto \(\lim_{n\to\infty}f(y_n)=0\neq a=f(a)\). Esto muestra que \(f\) no es continua en \(a\).

        2. \(a\notin\mathbb{Q}\). Entonces \(f(a)=0\). Existe una sucesión de números racionales \((r_n)\) con \(r_n\to a\). Para cada \(n\) se tiene \(f(r_n)=r_n\) y por tanto \(\lim_{n\to\infty}f(r_n)=\lim_{n\to\infty}r_n=a\neq 0=f(a)\). De nuevo \(f\) no es continua en \(a\).

        En ambos subcasos \(f\) es discontinua en \(a\). Por tanto \(f\) es discontinua en todo punto \(a\neq 0\).

        por lo que $f(x)$ es continua solo en $x=0$ y discontinua en todo $x\neq0$.

    \end{proof}
    \item Muestra que el rango de un polinomio de grado impar es todo $\mathbb{R}$.
    \begin{proof}
        Sea \(p(x)=a_n x^n + a_{n-1}x^{n-1}+\cdots+a_0\) un polinomio real con coeficientes \(a_i\in\mathbb{R}\) y \(\deg p=n\) impar. Es decir, \(n=2k+1\) para algún \(k\in\mathbb{N}_0\), y \(a_n\neq 0\).

        1. \textit{Comportamiento en los extremos.} Como \(n\) es impar, se tiene
        \[
        \lim_{x\to+\infty} a_n x^n = 
        \begin{cases}
        +\infty, & a_n>0,\\[4pt]
        -\infty, & a_n<0,
        \end{cases}
        \qquad
        \lim_{x\to-\infty} a_n x^n =
        \begin{cases}
        -\infty, & a_n>0,\\[4pt]
        +\infty, & a_n<0.
        \end{cases}
        \]
        Dado que los términos de menor grado son de orden inferior frente a \(a_n x^n\), se concluye que
        \[
        \lim_{x\to+\infty} p(x)=
        \begin{cases}
        +\infty, & a_n>0,\\[4pt]
        -\infty, & a_n<0,
        \end{cases}
        \qquad
        \lim_{x\to-\infty} p(x)=
        \begin{cases}
        -\infty, & a_n>0,\\[4pt]
        +\infty, & a_n<0.
        \end{cases}
        \]
        En particular, los límites en \(+\infty\) y \(-\infty\) tienen signos opuestos.

        2. \textit{Aplicación del Teorema del Valor Intermedio (TVI).} Sea \(y\in\mathbb{R}\) arbitrario. Consideremos la función continua
        \[
        q(x):=p(x)-y.
        \]
        Por la observación anterior existen \(X_1,X_2\in\mathbb{R}\) con \(X_1<X_2\) tales que \(q(X_1)\) y \(q(X_2)\) tienen signos opuestos (por ejemplo, tomar \(X_1\) suficientemente negativo y \(X_2\) suficientemente positivo). Concretamente, si \(a_n>0\) entonces \(q(X_1)<0\) para \(X_1\) muy negativo y \(q(X_2)>0\) para \(X_2\) muy positivo; el caso \(a_n<0\) es análogo intercambiando signos.

        Como \(q\) es continua en el intervalo \([X_1,X_2]\) y \(q(X_1)\cdot q(X_2)<0\), el Teorema del Valor Intermedio garantiza la existencia de \(c\in(X_1,X_2)\) tal que
        \[
        q(c)=0.
        \]
        Es decir, \(p(c)-y=0\), o bien \(p(c)=y\).

        3. \textit{Conclusión.} Dado que \(y\in\mathbb{R}\) era arbitrario, para cada \(y\) existe \(c\in\mathbb{R}\) con \(p(c)=y\). Así la imagen (rango) de \(p\) es \(\mathbb{R}\).
    \end{proof}
    \item \textbf{(Derivada de la funcion exponencial)} Como comentamos anteriormente, la funcion
    exponencial $e^{x}$ es la funcion inversa de la funcion logaritmo natural $\ln(x)$. Sabiendo
    que para $x\gt0$,
        \begin{align*}
            \ln^\prime(x)=\frac{1}{x},
        \end{align*}
    Prueba que la derivada de la funcion exponencial $e^x$ es la funcion exponencial $e^x$.
    \begin{proof}
        Sea \(e^x:\mathbb{R}\to(0,\infty)\) la función inversa de \(\ln:(0,\infty)\to\mathbb{R}\). Entonces para todo \(x\in\mathbb{R}\) se cumple
        \[
        \ln(e^x)=x.
        \]

        La función \(\ln\) es diferenciable en todo \(t>0\) y su derivada es \(\ln'(t)=\dfrac{1}{t}\). En particular, \(\ln'(e^x)=\dfrac{1}{e^x}\) y esta cantidad es distinta de cero para todo \(x\in\mathbb{R}\). Por el teorema de la derivada de la función inversa, se tiene que \(e^x\) es diferenciable en todo \(x\in\mathbb{R}\) y su derivada viene dada por
        \[
        (e^x)'=\frac{1}{\ln'(e^x)}.
        \]
        Sustituyendo \(\ln'(e^x)=\dfrac{1}{e^x}\) obtenemos
        \[
        (e^x)'=\frac{1}{\dfrac{1}{e^x}}=e^x.
        \]

        De forma equivalente, aplicando directamente la regla de la cadena a \(\ln(e^x)=x\):
        \[
        \frac{1}{e^x}\cdot (e^x)' = 1 \quad\Longrightarrow\quad (e^x)'=e^x.
        \]

        Por tanto, para todo \(x\in\mathbb{R}\),
        \[
        \dfrac{d}{dx}e^x = e^x.
        \]
    \end{proof}
    \item La recta tangente a la grafica de una funcion $f$ en el punto $(1,2)$ pasa por el
    punto $(3,4)$. Encuentra $f(1)$ y $f^\prime(1)$.
    \begin{solucion}
        Puesto que \((1,2)\) pertenece a la gráfica de \(f\), se tiene
        \[
        f(1)=2.
        \]

        La ecuación de la recta tangente a la gráfica de \(f\) en el punto \((1,2)\) viene dada por
        \[
        y-2=f'(1)(x-1).
        \]
        Dado que dicha recta pasa por \((3,4)\), sustituimos \((x,y)=(3,4)\) en la ecuación anterior y obtenemos
        \[
        4-2=f'(1)(3-1).
        \]
        Por tanto
        \[
        2=2\,f'(1)\quad\Longrightarrow\quad f'(1)=1.
        \]
        Así,
        \[
        f'(1)=1.
        \]
    \end{solucion}
    \item Realiza lo siguiente:
        \begin{enumerate}[label=(\alph*)]
            \item Dibuja la siguiente hiperbola
            \begin{align*}
                \frac{x^2}{4}-\frac{y^2}{2}=1.
            \end{align*}
            \item Usa la derivacion implicita para calcular la pendiente de la racta tangente
            $T$ a la hiperbola del inciso (a) en el punto $(3,\sqrt{5/2})$.
            \item Encuentra la ecuacion de $T$.
            \item Dibuja la grafica de $T$.
        \end{enumerate}
                    \begin{solucion}
                        \begin{enumerate}[label=(\alph*)]
                \item La grafica es\\
                \begin{center}
                \begin{tikzpicture}
                    \begin{axis}[
                        axis lines = middle,
                        xlabel = $x$, ylabel = $y$,
                        xmin = -5, xmax = 5, ymin = -4, ymax = 4
                    ]

                    % Usando parametrización
                    \addplot[blue, thick, samples=100, domain=-2:2] 
                        ({2*cosh(x)}, {sqrt(2)*sinh(x)});
                    \addplot[blue, thick, samples=100, domain=-2:2] 
                        ({2*cosh(x)}, {-sqrt(2)*sinh(x)});
                    \addplot[blue, thick, samples=100, domain=-2:2] 
                        ({-2*cosh(x)}, {sqrt(2)*sinh(x)});
                    \addplot[blue, thick, samples=100, domain=-2:2] 
                        ({-2*cosh(x)}, {-sqrt(2)*sinh(x)});

                    \end{axis}
                    \end{tikzpicture}
                    \end{center}
                \item Diferenciando implícitamente la ecuación de la hipérbola respecto de \(x\) obtenemos
                    \[
                    \frac{d}{dx}\!\left(\frac{x^2}{4}-\frac{y^2}{2}\right)=\frac{d}{dx}(1).
                    \]
                    Calculando las derivadas,
                    \[
                    \frac{2x}{4}-\frac{2y\,y'}{2}=0,
                    \]
                    de donde se simplifica a
                    \[
                    \frac{x}{2}-y\,y'=0.
                    \]
                    Despejando \(y'\),
                    \[
                    y'=\frac{x}{2y}.
                    \]
                    En el punto \(\bigl(3,\sqrt{5/2}\,\bigr)\) la pendiente es
                    \[
                    y'\Big|_{(3,\sqrt{5/2})}=\frac{3}{2\sqrt{5/2}}.
                    \]
                    Observando que \(\sqrt{5/2}=\dfrac{\sqrt{10}}{2}\), se puede simplificar:
                    \[
                    \frac{3}{2\sqrt{5/2}}=\frac{3}{\sqrt{10}}=\frac{3\sqrt{10}}{10}.
                    \]
                    Por tanto la pendiente de la tangente en el punto dado es
                    \[
                    m=\dfrac{3\sqrt{10}}{10}.
                    \]

                    \item La ecuación punto-pendiente de la recta tangente \(T\) en \(\bigl(3,\sqrt{5/2}\,\bigr)\) es
                    \[
                    y-\sqrt{\tfrac{5}{2}} \;=\; \frac{3\sqrt{10}}{10}\,(x-3).
                    \]
                    Usando \(\sqrt{5/2}=\dfrac{\sqrt{10}}{2}\) se puede reescribir \(T\) en forma explícita \(y=mx+b\). Calculamos la ordenada al origen:
                    \[
                    b=\sqrt{\tfrac{5}{2}} - m\cdot 3 = \frac{\sqrt{10}}{2} - 3\cdot\frac{3\sqrt{10}}{10}
                    = \sqrt{10}\!\left(\frac{1}{2}-\frac{9}{10}\right)
                    = \sqrt{10}\!\left(\frac{5-9}{10}\right)
                    = -\frac{2\sqrt{10}}{5}.
                    \]
                    Así, la ecuación en forma pendiente-intersección es
                    \[
                    \,y=\frac{3\sqrt{10}}{10}\,x-\frac{2\sqrt{10}}{5}\,.
                    \]
                    \item La grafica es
                    \begin{center}
                        \begin{tikzpicture}
                        \begin{axis}[
                            axis lines = middle,
                            xlabel = $x$,
                            ylabel = $y$,
                            xmin = -5, xmax = 5,
                            ymin = -4, ymax = 4,
                            grid = both
                        ]

                        % Hipérbola (azul)
                        \addplot[domain = -5:-2, samples = 100, thick, blue] 
                            {sqrt((x^2/4 - 1)*2)};
                        \addplot[domain = -5:-2, samples = 100, thick, blue] 
                            {-sqrt((x^2/4 - 1)*2)};
                        \addplot[domain = 2:5, samples = 100, thick, blue] 
                            {sqrt((x^2/4 - 1)*2)};
                        \addplot[domain = 2:5, samples = 100, thick, blue] 
                            {-sqrt((x^2/4 - 1)*2)};

                        % Recta (rojo)
                        \addplot[domain = -5:5, thick, red] 
                            {(3*sqrt(10)/10)*x - (2*sqrt(10)/5)};

                        % Leyenda
                        \node[blue] at (axis cs: -4,3) {Hipérbola};
                        \node[red] at (axis cs: 2,-2) {Recta};

                        \end{axis}
                        \end{tikzpicture}
                    \end{center}
                \end{enumerate}
            \end{solucion}
    \item Encuentra los intervalos abiertos donde la funcion $f(x)=2x^3-3x^2+1$ es creciente
    o decreciente utilizando la receta de la subseccion 5.8.1.

\end{enumerate}



\end{document}