\documentclass[12pt,a4paper]{article}
\newcommand{\grupo}{}
% ------------------------------------------------------
% Idioma y codificación
% ------------------------------------------------------
\usepackage[spanish]{babel}
\usepackage[utf8]{inputenc}
\usepackage[T1]{fontenc}

% ------------------------------------------------------
% Tipografía y formato
% ------------------------------------------------------
\usepackage{lmodern}    % Fuente Latin Modern
\usepackage{microtype}  % Mejor espaciado
\usepackage{xcolor}     % Colores
\definecolor{UGblue}{RGB}{0,33,71}
\definecolor{UGgold}{RGB}{212,175,55}
\usepackage{graphicx}
\usepackage{setspace}
\usepackage{titlesec}
% Añade esto en el preámbulo (antes de \begin{document})
\usepackage{bookmark}
\bookmarksetup{open,numbered}
\usepackage{tikz}
\usetikzlibrary{calc,decorations.pathreplacing,angles,quotes}
\usepackage{pgfplots}
\pgfplotsset{compat=1.18}
\usepackage{caption}

% ------------------------------------------------------
% Matemáticas
% ------------------------------------------------------
\usepackage{amsmath,amssymb,amsthm,mathtools}
\usepackage{mathrsfs}
\usepackage{stmaryrd}
\usepackage{physics}

% ------------------------------------------------------
% Otros paquetes útiles
% ------------------------------------------------------
\usepackage{enumitem}
\usepackage{csquotes}
\usepackage{hyperref}
\hypersetup{
    colorlinks=true,
    linkcolor=UGblue,
    urlcolor=UGgold,
    citecolor=UGblue
}
\usepackage{geometry}
\geometry{margin=2.5cm}
\usepackage{multicol}

% ------------------------------------------------------
% Encabezados y pies de página
% ------------------------------------------------------
\usepackage{fancyhdr}
\pagestyle{fancy}
\fancyhf{}
\lhead{Calculo Diferencial e Integral I}
\rhead{Ricardo Leon Martinez}
\cfoot{\thepage}
% ------------------------------------------------------
% Teoremas, definiciones, etc.
% ------------------------------------------------------
\theoremstyle{plain}
\newtheorem{teo}{Teorema}[section]
\newtheorem{prop}[teo]{Proposición}
\newtheorem{lema}[teo]{Lema}
\theoremstyle{definition}
\newtheorem{defi}[teo]{Definición}
\theoremstyle{remark}
\newtheorem{ejemplo}[teo]{Ejemplo}
\newtheorem{obs}[teo]{Observación}

% ------------------------------------------------------
% graficas
% ------------------------------------------------------
\usepackage{graphicx}
\usepackage{caption}
\usepackage{subcaption}
\usepackage{pgfplots}
\pgfplotsset{compat=1.18}
\usepackage{siunitx}
% ------------------------------------------------------
% Comandos útiles
% ------------------------------------------------------
\newcommand{\N}{\mathbb{N}}
\newcommand{\Z}{\mathbb{Z}}
\newcommand{\Q}{\mathbb{Q}}
\newcommand{\R}{\mathbb{R}}
\newcommand{\C}{\mathbb{C}}
\newcommand{\pfrac}[2]{\left( \frac{#1}{#2} \right)} %parentesis
\newcommand{\bfrac}[2]{\left[ \frac{#1}{#2} \right]} %corchetes
\newcommand{\vecpar}[1]{\left( #1 \right)}
% En el preámbulo
\newcommand{\point}[2]{\left( #1, #2 \right)}
\newcommand{\pointfrac}[4]{\left( \frac{#1}{#2}, \frac{#3}{#4} \right)}
\newcommand{\pointmix}[4]{\left( #1, \frac{#3}{#4} \right)} % Primera coordenada normal, segunda fracción
\renewcommand{\qedsymbol}{$\blacksquare$}
\newcommand{\gt}{\ensuremath{>}}
\newcommand{\lt}{\ensuremath{<}}
\renewcommand{\div}{$÷$}

\renewcommand{\baselinestretch}{1.2}
\setlength{\parindent}{0pt}

\newenvironment{solucion}
{\par\noindent\textbf{Solución.}\ }
{\hfill$\blacktriangleleft$\par}

% ------------------------------------------------------
% Datos personales (modificar según tarea)
% ------------------------------------------------------
\newcommand{\alumno}{Ricardo León Martínez}
\newcommand{\materia}{Calculo Diferencial e Integral I}
\newcommand{\profesor}{Fernando Nuñez Medina}
\newcommand{\tarea}{Tarea 12}
\newcommand{\fecha}{14/11/2025}

\setlength{\parindent}{0pt}

\begin{document}
    \begin{center}
        {\large\textbf{UNIVERSIDAD DE GUANAJUATO}}\\[0.3cm]
        {\normalsize\textbf{DIVISIÓN DE CIENCIAS NATURALES Y EXACTAS}}\\
        {\normalsize\textbf{CAMPUS GUANAJUATO}}\\[1cm]

        {\Large\textbf{\tarea\ (\materia)}}\\[1cm]
    \end{center}
    \textbf{Nombre:} \alumno \hfill 
    \textbf{Fecha:} \fecha \hfill 
    \textbf{Calificación:} \rule{3cm}{0.4pt} \\[0.3cm]
    \begin{enumerate}[left=0pt, label=\textbf{\arabic*.}]
        \item Prueba que si $f:B\to\mathbb{R}$ es continua y $A\subset B$,
        entonces $f|_{A}:A\to\mathbb{R}$ es continua.
        \item \textbf{(Las funciones continuas mandan intervalos cerrados
        y acotados en intervalos cerrados y acotados)} Prueba que si $f$ es continua en $[a,b]$,
        entonces $f([a,b])=[m,M]$, donde $m=\min\{f(x):x\in[a,b]\}$ y $m=\max\{f(x):x\in[a,b]\}$.
        \item Prueba que si $f:B\to C$ y $g:A\to B$ son funciones uniformemente continuas, entonces
        $f\circ g:A\to C$ es uniformemente continua.
        \item \textbf{(Derivada de las funciones trigonometruicas inversas)}
        Usa la proposicion 62 (derivada de una funcion inversa) para mostrar lo siguiente
            \begin{enumerate}
                \item[(a)] $\arcsin^{\prime}(x)=\frac{1}{\sqrt{1-x^2}},\text{ } -1\lt x\lt 1$. 
                \item[(b)] $\arccos^{\prime}(x)=\frac{-1}{\sqrt{1-x^2}},\text{ } -1\lt x\lt 1$.
                \item[(c)] $\arctan^{\prime}(x)=\frac{1}{1+x^2},\text{ } -\infty\lt x\lt\infty$.
                \item[(d)] $\arccot^{\prime}(x)=\frac{-1}{1+x^2},\text{ } -\infty\lt x\lt\infty$.
                \item[(e)] $\arcsec^{\prime}(x)=
                    \begin{cases*}
                        \frac{1}{x\sqrt{x^2-1}},\text{ }x\gt1,\\
                        \frac{-1}{x\sqrt{x^2-1}},\text{ }x\lt-1.
                    \end{cases*}$
                \item[(f)] $\arccsc^{\prime}(x)=
                    \begin{cases*}
                        \frac{-1}{x\sqrt{x^2-1}},\text{ }x\gt1,\\
                        \frac{-1}{x\sqrt{x^2-1}},\text{ }x\lt-1.
                    \end{cases*}$ 
            \end{enumerate}
        \item Encuentra los minimos y maximos globales de la funcion $f(x)=x^3-3x^2+1$
                en el intervalo $[-1,1]$.
        \item Supongamos que un diseñador de recipientes desea contruir un recipiente cilindrico
        de carton sin tapa. Se requiere que el cilindro tenga volumen $V$ de $27\pi cm^3$.
        ¿Cuales seran las dimensiones del cilindro que minimicen la cantidad de carton empleado
        para construirlo?
        \item Expresa los polinomios siguientes como polinomios de taylor
            \begin{enumerate}
                \item[(a)] $f(x)=5$.
                \item[(b)] $f(x)=2x+1$.
                \item[(c)] $f(x)=4x^2-9x+6$.
                \item[(d)] $f(x)=x^{10}-2x^3+3x^2+7x-11$.
            \end{enumerate}
        \item Realiza lo siguiente:
            \begin{enumerate}
                \item[(a)] Calcula los polinomios de taylor centrados en 0 de la funcion
                $\cos(x)$.
                \item[(b)] Estimna el numero $\cos(2)$ con una exactitud de tres cifras decimales.
            \end{enumerate}
    \end{enumerate}
\end{document}