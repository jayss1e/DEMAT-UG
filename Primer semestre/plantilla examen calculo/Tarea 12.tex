
\documentclass[12pt,oneside,spanish]{amsbook}
\newcommand{\grupo}{}
% ------------------------------------------------------
% Idioma y codificación
% ------------------------------------------------------
\usepackage[spanish]{babel}
\usepackage[utf8]{inputenc}
\usepackage[T1]{fontenc}

% ------------------------------------------------------
% Tipografía y formato
% ------------------------------------------------------
\usepackage{lmodern}    % Fuente Latin Modern
\usepackage{microtype}  % Mejor espaciado
\usepackage{xcolor}     % Colores
\definecolor{UGblue}{RGB}{0,33,71}
\definecolor{UGgold}{RGB}{212,175,55}
\usepackage{graphicx}
\usepackage{setspace}
\usepackage{titlesec}
% Añade esto en el preámbulo (antes de \begin{document})
\usepackage{bookmark}
\bookmarksetup{open,numbered}
\usepackage{tikz}
\usetikzlibrary{calc,decorations.pathreplacing,angles,quotes}
\usepackage{pgfplots}
\pgfplotsset{compat=1.18}
\usepackage{caption}

% ------------------------------------------------------
% Matemáticas
% ------------------------------------------------------
\usepackage{amsmath,amssymb,amsthm,mathtools}
\usepackage{mathrsfs}
\usepackage{stmaryrd}
\usepackage{physics}

% ------------------------------------------------------
% Otros paquetes útiles
% ------------------------------------------------------
\usepackage{enumitem}
\usepackage{csquotes}
\usepackage{hyperref}
\hypersetup{
    colorlinks=true,
    linkcolor=UGblue,
    urlcolor=UGgold,
    citecolor=UGblue
}
\usepackage{multicol}

% ------------------------------------------------------
% Encabezados y pies de página
% ------------------------------------------------------
% ------------------------------------------------------
% Teoremas, definiciones, etc.
% ------------------------------------------------------
\theoremstyle{plain}
\newtheorem{teo}{Teorema}[section]
\newtheorem{prop}[teo]{Proposición}
\newtheorem{lema}[teo]{Lema}
\theoremstyle{definition}
\newtheorem{defi}[teo]{Definición}
\theoremstyle{remark}
\newtheorem{ejemplo}[teo]{Ejemplo}
\newtheorem{obs}[teo]{Observación}

% ------------------------------------------------------
% Comandos útiles
% ------------------------------------------------------
\newcommand{\N}{\mathbb{N}}
\newcommand{\Z}{\mathbb{Z}}
\newcommand{\Q}{\mathbb{Q}}
\newcommand{\R}{\mathbb{R}}
\newcommand{\C}{\mathbb{C}}
\newcommand{\pfrac}[2]{\left( \frac{#1}{#2} \right)} %parentesis
\newcommand{\bfrac}[2]{\left[ \frac{#1}{#2} \right]} %corchetes
\newcommand{\vecpar}[1]{\left( #1 \right)}
% En el preámbulo
\newcommand{\point}[2]{\left( #1, #2 \right)}
\newcommand{\pointfrac}[4]{\left( \frac{#1}{#2}, \frac{#3}{#4} \right)}
\newcommand{\pointmix}[4]{\left( #1, \frac{#3}{#4} \right)} % Primera coordenada normal, segunda fracción
\renewcommand{\qedsymbol}{$\blacksquare$}
\newcommand{\gt}{\ensuremath{>}}
\newcommand{\lt}{\ensuremath{<}}
\renewcommand{\div}{|}

\renewcommand{\baselinestretch}{1.2}
\setlength{\parindent}{0pt}

\newenvironment{solucion}
{\par\noindent\textit{Solución.}\ }
{\hfill$\blacktriangleleft$\par}

% ------------------------------------------------------
% Datos personales (modificar según tarea)
% ------------------------------------------------------
\newcommand{\alumno}{Ricardo León Martínez}
\newcommand{\materia}{Elementos de Geometría}
\newcommand{\profesor}{Valentina Muñoz Porras}
\newcommand{\tarea}{Tarea 5}
\newcommand{\fecha}{22/09/2025}

\setlength{\parindent}{0pt}
\begin{document}
    \begin{center}
        \begin{tabular}{c}
            \includegraphics[width=2.5cm,height=2.5cm]{C:/Users/Radoj/Documents/logo.png}\tabularnewline
        \end{tabular}%
        \begin{tabular}{c}
            \textbf{UNIVERSIDAD DE GUANAJUATO}\tabularnewline
            \textbf{\scriptsize{}DIVISIÓN DE CIENCIAS NATURALES Y EXACTAS}\tabularnewline
            \textbf{\scriptsize{}CAMPUS GUANAJUATO}\tabularnewline
        \end{tabular}
    \par\end{center}

    \begin{center}
        \textbf{Examen Parcial 1 (Cálculo Diferencial e Integral I)}
    \par\end{center}

    \begin{tabular}{|l|l|l|}
        \hline 
        \multicolumn{3}{|l|}{\textbf{Nombre:\hspace*{10cm}}}\tabularnewline
        \hline 
        \textbf{Grupo:\hspace*{2cm}} & \textbf{Fecha:\hspace*{2cm}} & \textbf{Calificación:}\tabularnewline
        \hline 
        \multicolumn{3}{|l|}{\textbf{Profesor:} Fernando Núñez Medina.}\tabularnewline
        \hline 
    \end{tabular}

    \vspace{0.5cm}

    \textbf{Instrucciones:} Escribe limpia y ordenadamente el procedimiento
    (si lo hay) de cada ejercicio y no escribas las respuestas en la hoja
    del examen. ¡Suerte! 
    \begin{enumerate}
        \item Muestra que la ecuación $x^{5}\cos^{1000}(x)+4x-3=0$ tiene una
        solución en el intervalo $[0,1]$.
        \item Si $g(x)=7(x-1)^{3}e^{x-2}$, calcula $(g^{-1})^{\prime}(7)$. Ten
        en cuenta que $g(2)=7$.
        \item La recta tangente a la gráfica de una función $f$ en el punto $(0,3)$
        pasa por el punto $(1,5)$. Encuentra $f(0)$ y $f^{\prime}(0)$.
        \item Analiza la gráfica de la función $f(x)=-x^{4}+2x^{2}$ siguiendo el
        procedimiento que vimos en clase (el de los diez pasos).
        \item Muestra que el rectángulo de menor perimetro y área fija $A$ es
        un cuadrado.
    \end{enumerate}
\end{document}
