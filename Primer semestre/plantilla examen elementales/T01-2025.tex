\documentclass[12pt]{extarticle}
\newcommand{\grupo}{}
% ------------------------------------------------------
% Idioma y codificación
% ------------------------------------------------------
\usepackage[spanish]{babel}
\usepackage[utf8]{inputenc}
\usepackage[T1]{fontenc}

% ------------------------------------------------------
% Tipografía y formato
% ------------------------------------------------------
\usepackage{lmodern}    % Fuente Latin Modern
\usepackage{microtype}  % Mejor espaciado
\usepackage{xcolor}     % Colores
\definecolor{UGblue}{RGB}{0,33,71}
\definecolor{UGgold}{RGB}{212,175,55}
\usepackage{graphicx}
\usepackage{setspace}
\usepackage{titlesec}
% Añade esto en el preámbulo (antes de \begin{document})
\usepackage{bookmark}
\bookmarksetup{open,numbered}
\usepackage{tikz}
\usetikzlibrary{calc,decorations.pathreplacing,angles,quotes}
\usepackage{pgfplots}
\pgfplotsset{compat=1.18}
\usepackage{caption}
\usepackage{geometry}
\geometry{margin=2.5cm}

% ------------------------------------------------------
% Matemáticas
% ------------------------------------------------------
\usepackage{amsmath,amssymb,amsthm,mathtools}
\usepackage{mathrsfs}
\usepackage{stmaryrd}
\usepackage{physics}

% ------------------------------------------------------
% Otros paquetes útiles
% ------------------------------------------------------
\usepackage{enumitem}
\usepackage{csquotes}
\usepackage{hyperref}
\hypersetup{
    colorlinks=true,
    linkcolor=UGblue,
    urlcolor=UGgold,
    citecolor=UGblue
}
\usepackage{multicol}

% ------------------------------------------------------
% Encabezados y pies de página
% ------------------------------------------------------
% ------------------------------------------------------
% Teoremas, definiciones, etc.
% ------------------------------------------------------
\theoremstyle{plain}
\newtheorem{teo}{Teorema}[section]
\newtheorem{prop}[teo]{Proposición}
\newtheorem{lema}[teo]{Lema}
\theoremstyle{definition}
\newtheorem{defi}[teo]{Definición}
\theoremstyle{remark}
\newtheorem{ejemplo}[teo]{Ejemplo}
\newtheorem{obs}[teo]{Observación}

% ------------------------------------------------------
% Comandos útiles
% ------------------------------------------------------
\newcommand{\N}{\mathbb{N}}
\newcommand{\Z}{\mathbb{Z}}
\newcommand{\Q}{\mathbb{Q}}
\newcommand{\R}{\mathbb{R}}
\newcommand{\C}{\mathbb{C}}
\newcommand{\pfrac}[2]{\left( \frac{#1}{#2} \right)} %parentesis
\newcommand{\bfrac}[2]{\left[ \frac{#1}{#2} \right]} %corchetes
\newcommand{\vecpar}[1]{\left( #1 \right)}
% En el preámbulo
\newcommand{\point}[2]{\left( #1, #2 \right)}
\newcommand{\pointfrac}[4]{\left( \frac{#1}{#2}, \frac{#3}{#4} \right)}
\newcommand{\pointmix}[4]{\left( #1, \frac{#3}{#4} \right)} % Primera coordenada normal, segunda fracción
\renewcommand{\qedsymbol}{$\blacksquare$}
\newcommand{\gt}{\ensuremath{>}}
\newcommand{\lt}{\ensuremath{<}}
\renewcommand{\div}{|}

\renewcommand{\baselinestretch}{1.2}
\setlength{\parindent}{0pt}

\newenvironment{solucion}
{\par\noindent\textit{Solución.}\ }
{\hfill$\blacktriangleleft$\par}

% ------------------------------------------------------
% Datos personales (modificar según tarea)
% ------------------------------------------------------
\newcommand{\alumno}{Ricardo León Martínez}
\newcommand{\materia}{Elementos de Geometría}
\newcommand{\profesor}{Valentina Muñoz Porras}
\newcommand{\tarea}{Tarea 5}
\newcommand{\fecha}{22/09/2025}

\setlength{\parindent}{0pt}
\begin{document}
  \begin{flushright}
    \begin{minipage}{0.55\textwidth}
        \large
        Matemáticas Elementales\\
        \normalsize
        x de y de 2025\\
        Parcial 3 
    \end{minipage}%
    \begin{minipage}{0.4\textwidth}
      \begin{flushright}
        \begin{tabular}{|c|c|c|c|c|}
          \hline
          \ \ $E_1$\ \ &\ \ $E_2$\ \ &\ \ $E_3$\ \ &\ \ $E_4$\ \ &Calif.\\
          \hline
          &  &  &  &  \\
          &  &  &  &  \\
          \hline
        \end{tabular}
      \end{flushright}
    \end{minipage}
  \end{flushright}
  Resuelve los siguientes ejercicios justificando totalmente cada paso en tu
  procedimiento. La falta de alguna justificación y/o la mala redacción, serán
  severamente penalizadas.

  \vspace{0.3cm}

  El examen debe resolverse utilizando únicamente tinta azul estándar y sin escribir
  en esta hoja algo mas que tu nombre y numero de inscripción.

  \vspace{0.3cm}

  Utiliza ambos lados de las hojas de respuestas e indica claramente donde termina
  un ejercicio/inciso y comienza uno nuevo.

  \vspace{0.3cm}

  Puedes utilizar todos los resultados vistos en cursos previos siempre y cuando
  estos no sean lo que se pide demostrar. \textbf{No} puede utilizar resultados probados
  en las tareas y/o en las ayudantías. En caso de requerir alguno de tales resultados,
  deberás incluir su demostración.

  \vspace{0.3cm}

  Se penalizarán los pasos innecesarios.

  \vspace{0.3cm}

  La calificacion final del examen será el promedio de las puntuaciones obtenidas
  en cada ejercicio.

  \vspace{0.3cm}

  \begin{enumerate}
    \item (100 pts.) Demuestra que el conjunto
    \begin{align*}
      \{q\in\mathbb{Q}:q<0\}
    \end{align*}
    es una cortadura de dedekind.
    \item (100 pts.) Demuestra que el siguiente par de congruencias tiene solución
    común y halla el conjunto de todas las soluciones comunes.
    \begin{align*}
      &x\equiv 6 \pmod{10}\\
      &x\equiv 7 \pmod{11}.
    \end{align*}
    \item (100 pts.) Sea $A\subset\mathbb{N}$ infinito. Utilizando el principio del
    buen orden demuestra que $\abs{A}=\abs{\mathbb{N}}$.
    \item Demuestra que $\mathbb{Q}$ es numerable.
  \end{enumerate}
\end{document}
